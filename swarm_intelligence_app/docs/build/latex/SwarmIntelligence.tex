%% Generated by Sphinx.
\def\sphinxdocclass{report}
\documentclass[letterpaper,10pt,english]{sphinxmanual}
\ifdefined\pdfpxdimen
   \let\sphinxpxdimen\pdfpxdimen\else\newdimen\sphinxpxdimen
\fi \sphinxpxdimen=49336sp\relax

\usepackage[margin=1in,marginparwidth=0.5in]{geometry}
\usepackage[utf8]{inputenc}
\ifdefined\DeclareUnicodeCharacter
  \DeclareUnicodeCharacter{00A0}{\nobreakspace}
\fi
\usepackage{cmap}
\usepackage[T1]{fontenc}
\usepackage{amsmath,amssymb,amstext}
\usepackage{babel}
\usepackage{times}
\usepackage[Bjarne]{fncychap}
\usepackage{longtable}
\usepackage{sphinx}

\usepackage{multirow}
\usepackage{eqparbox}

% Include hyperref last.
\usepackage{hyperref}
% Fix anchor placement for figures with captions.
\usepackage{hypcap}% it must be loaded after hyperref.
% Set up styles of URL: it should be placed after hyperref.
\urlstyle{same}

\addto\captionsenglish{\renewcommand{\figurename}{Fig.\@ }}
\addto\captionsenglish{\renewcommand{\tablename}{Table }}
\addto\captionsenglish{\renewcommand{\literalblockname}{Listing }}

\addto\extrasenglish{\def\pageautorefname{page}}

\setcounter{tocdepth}{1}



\title{Swarm Intelligence}
\date{Jan 17, 2017}
\release{0.1}
\author{}
\newcommand{\sphinxlogo}{}
\renewcommand{\releasename}{Release}
\makeindex

\begin{document}

\maketitle
\sphinxtableofcontents
\phantomsection\label{\detokenize{index::doc}}



\chapter{Overview}
\label{\detokenize{overview:overview}}\label{\detokenize{overview::doc}}\label{\detokenize{overview:swarm-intelligence-platform}}
The aim of this project is to develop an open-source platform that fits well to the Holacracy® constitution (\url{http://www.holacracy.org/constitution}).

The stack of choice is Flask, a flexible Python micro framework to develop robust web applications.  Different libraries, mentionable Flask-RESTful and Flask-SQLAlchemy, add further functionality to the stack. Data persistence is handled by a MySQL database.

To foster a fast, flexible and collaborative development process the project is hosted and maintained on GitHub (\url{https://github.com/dcentralize/swarm-intelligence}). Travis CI, a continuous integration service, makes it easy to test and to coordinate the commits and increases the quality of the product.


\chapter{Installation}
\label{\detokenize{installation::doc}}\label{\detokenize{installation:installation}}

\section{Getting Started using Ubuntu}
\label{\detokenize{installation:getting-started-using-ubuntu}}
These instructions will get you a copy of the project up and running on your local machine for development and testing purposes.


\subsection{Prerequisites}
\label{\detokenize{installation:prerequisites}}
First you need to checkout the GitHub Repository using:

\begin{sphinxVerbatim}[commandchars=\\\{\}]
\PYG{n}{git} \PYG{n}{clone} \PYG{n}{https}\PYG{p}{:}\PYG{o}{/}\PYG{o}{/}\PYG{n}{github}\PYG{o}{.}\PYG{n}{com}\PYG{o}{/}\PYG{n}{dcentralize}\PYG{o}{/}\PYG{n}{swarm}\PYG{o}{\PYGZhy{}}\PYG{n}{intelligence}\PYG{o}{.}\PYG{n}{git}
\end{sphinxVerbatim}

It is highly recommended to run everything in an virtualenv. The environment can be set up using:

\begin{sphinxVerbatim}[commandchars=\\\{\}]
\PYG{n}{mkvirtualenv} \PYG{o}{\PYGZhy{}}\PYG{o}{\PYGZhy{}}\PYG{n}{python} \PYG{n}{python3}\PYG{o}{.}\PYG{l+m+mi}{4} \PYG{o}{\PYGZhy{}}\PYG{n}{a} \PYG{o}{.} \PYG{n}{si}
\end{sphinxVerbatim}

To create a local database, install Mariadb:

\begin{sphinxVerbatim}[commandchars=\\\{\}]
\PYG{n}{apt}\PYG{o}{\PYGZhy{}}\PYG{n}{get} \PYG{n}{install} \PYG{n}{mariadb}\PYG{o}{\PYGZhy{}}\PYG{n}{server}
\end{sphinxVerbatim}

In order to run or deploy the project, it is necessary to download the dependencies:

\begin{sphinxVerbatim}[commandchars=\\\{\}]
\PYG{n}{pip3} \PYG{n}{install} \PYG{o}{\PYGZhy{}}\PYG{n}{r} \PYG{n}{requirements}\PYG{o}{.}\PYG{n}{txt}
\end{sphinxVerbatim}


\subsection{Installation}
\label{\detokenize{installation:id1}}
A step by step series of examples that tell you how to get a development env running.

Starting mariadb:

\begin{sphinxVerbatim}[commandchars=\\\{\}]
\PYG{n}{service} \PYG{n}{mariadb} \PYG{n}{start}
\end{sphinxVerbatim}

Setting up the database:

\begin{sphinxVerbatim}[commandchars=\\\{\}]
\PYG{n}{mysql} \PYG{o}{\PYGZhy{}}\PYG{n}{u} \PYG{n}{root} \PYG{o}{\PYGZhy{}}\PYG{n}{e} \PYG{l+s+s1}{\PYGZsq{}}\PYG{l+s+s1}{CREATE DATABASE swarm\PYGZus{}intelligence}\PYG{l+s+s1}{\PYGZsq{}}
\end{sphinxVerbatim}

Adding the directory `swarm-intelligence' to your PYTHONPATH:

\begin{sphinxVerbatim}[commandchars=\\\{\}]
export PYTHONPATH=\PYGZdl{}PYTHONPATH:/path/of/swarm\PYGZhy{}intelligence
\end{sphinxVerbatim}

You can now navigate to the app.py and run it using:

\begin{sphinxVerbatim}[commandchars=\\\{\}]
\PYG{n}{cd} \PYG{n}{swarm}\PYG{o}{\PYGZhy{}}\PYG{n}{intelligence}
\PYG{n}{python3} \PYG{n}{swarm\PYGZus{}intelligence\PYGZus{}app}\PYG{o}{/}\PYG{n}{app}\PYG{o}{.}\PYG{n}{py}
\end{sphinxVerbatim}

You can now access the API at localhost:5000. Please not that accessing the API via 127.0.0.1:5000 will not work.


\section{Running the tests}
\label{\detokenize{installation:running-the-tests}}
Normally our tests are run using Travis-CI. In order to run the tests locally, navigate to the /tests directory and run:

\begin{sphinxVerbatim}[commandchars=\\\{\}]
\PYG{n}{py}\PYG{o}{.}\PYG{n}{test}
\end{sphinxVerbatim}


\subsection{Coding style tests}
\label{\detokenize{installation:coding-style-tests}}
Our coding style is conform to flake8, except for some minor exceptions which can be found in the tox.ini.


\section{Built With}
\label{\detokenize{installation:built-with}}\begin{itemize}
\item {} 
{[}PyCharm{]}(\url{https://www.jetbrains.com/pycharm/})

\item {} 
{[}Travis-CI{]}(\url{https://travis-ci.org/})

\item {} 
{[}Mariadb{]}(\url{https://mariadb.org/})

\item {} 
{[}Flask{]}(\url{http://flask.pocoo.org/docs/0.11/})

\item {} 
{[}Flask-Cors{]}(\url{https://github.com/corydolphin/flask-cors})

\item {} 
{[}Flask-HTTPAuth{]}(\url{https://flask-httpauth.readthedocs.io/en/latest/})

\item {} 
{[}Flask-RESTful{]}(\url{https://flask-restful-cn.readthedocs.io/en/0.3.5/})

\item {} 
{[}Flask-SQLAlchemy{]}(\url{http://flask-sqlalchemy.pocoo.org/2.1/})

\item {} 
{[}Jinja2{]}(\url{http://jinja.pocoo.org/})

\item {} 
{[}PyJWT{]}(\url{http://github.com/jpadilla/pyjwt})

\item {} 
{[}PyMySQL{]}(\url{https://media.readthedocs.org/pdf/pymysql/latest/pymysql.pdf})

\item {} 
{[}SQLAlchemy{]}(\url{http://www.sqlalchemy.org})

\item {} 
{[}SQLAlchemy-Utils{]}(\url{https://github.com/kvesteri/sqlalchemy-utils})

\item {} 
{[}Py{]}(\url{https://pypi.python.org/pypi})

\item {} 
{[}Pytest{]}(\url{http://doc.pytest.org/en/latest/})

\item {} 
{[}Pytest-Flask{]}(\url{https://pytest-flask.readthedocs.io/en/latest/})

\item {} 
{[}requests{]}(\url{http://python-requests.org})

\item {} 
{[}Tox{]}(\url{https://tox.readthedocs.io/en/latest/})

\end{itemize}


\chapter{Using the API}
\label{\detokenize{usage:using-the-api}}\label{\detokenize{usage::doc}}

\section{HTTP Methods}
\label{\detokenize{usage:http-methods}}
The API is implemented as RESTful web service and uses HTTP to access and manipulate resources. The following table shows which HTTP methods are supported:

\noindent\begin{tabulary}{\linewidth}{|L|L|}
\hline
\sphinxstylethead{\relax 
Method
\unskip}\relax &\sphinxstylethead{\relax 
Description
\unskip}\relax \\
\hline
GET
&
Used for retrieving resources.
\\
\hline
POST
&
Used for creating resources.
\\
\hline
PUT
&
Used for updating resources.
\\
\hline
DELETE
&
Used for deleting resources.
\\
\hline\end{tabulary}



\section{HTTP Status Codes}
\label{\detokenize{usage:http-status-codes}}
There are three different HTTP status codes for successful requests and five HTTP status codes to indicate client errors. The status codes are used as follows:

\sphinxstylestrong{On success}

\noindent\begin{tabulary}{\linewidth}{|L|L|}
\hline
\sphinxstylethead{\relax 
Code
\unskip}\relax &\sphinxstylethead{\relax 
Description
\unskip}\relax \\
\hline
200
&
The request has succeeded.
\\
\hline
201
&
The request has succeeded and resulted in a new resource.
\\
\hline
204
&
The request has succeeded without content being returned.
\\
\hline\end{tabulary}


\sphinxstylestrong{On client error}

\noindent\begin{tabulary}{\linewidth}{|L|L|}
\hline

Code
&
Description
\\
\hline
400
&
The request failed due to malformed syntax.
\\
\hline
401
&
The request failed due to missing or invalid token.
\\
\hline
403
&
The request failed due to missing permissions.
\\
\hline
404
&
The requested resource was not found.
\\
\hline
409
&
The request failed due to a conflict with the resource.
\\
\hline\end{tabulary}



\section{Authentication}
\label{\detokenize{usage:authentication}}
Authentication is implemented by using \href{https://jwt.io/}{JSON Web Tokens} (JWT). To authenticate through the Swarm Intelligence Platform API sent an Authorization header with each request like this:

\begin{sphinxVerbatim}[commandchars=\\\{\}]
\PYG{n}{Authorization}\PYG{p}{:} \PYG{n}{Bearer} \PYG{o}{\PYGZlt{}}\PYG{n}{JSON} \PYG{n}{Web} \PYG{n}{Token}\PYG{o}{\PYGZgt{}}
\end{sphinxVerbatim}


\section{JSON Encoded Data}
\label{\detokenize{usage:json-encoded-data}}
All reponses contain JSON encoded data. A single resource is represented by a JSON object; A collection of resources is represented by a JSON array.

\sphinxstylestrong{Single resource}

\begin{sphinxVerbatim}[commandchars=\\\{\}]
\PYG{n}{HTTP}\PYG{o}{/}\PYG{l+m+mf}{1.1} \PYG{l+m+mi}{200} \PYG{n}{OK}
\PYG{n}{Content}\PYG{o}{\PYGZhy{}}\PYG{n}{Type}\PYG{p}{:} \PYG{n}{application}\PYG{o}{/}\PYG{n}{json}

\PYG{p}{\PYGZob{}}
    \PYG{l+s+s1}{\PYGZsq{}}\PYG{l+s+s1}{key1}\PYG{l+s+s1}{\PYGZsq{}}\PYG{p}{,} \PYG{l+s+s1}{\PYGZsq{}}\PYG{l+s+s1}{value1}\PYG{l+s+s1}{\PYGZsq{}}\PYG{p}{,}
    \PYG{l+s+s1}{\PYGZsq{}}\PYG{l+s+s1}{key2}\PYG{l+s+s1}{\PYGZsq{}}\PYG{p}{,} \PYG{l+s+s1}{\PYGZsq{}}\PYG{l+s+s1}{value2}\PYG{l+s+s1}{\PYGZsq{}}
\PYG{p}{\PYGZcb{}}
\end{sphinxVerbatim}

\sphinxstylestrong{Collection of resources}

\begin{sphinxVerbatim}[commandchars=\\\{\}]
\PYG{n}{HTTP}\PYG{o}{/}\PYG{l+m+mf}{1.1} \PYG{l+m+mi}{200} \PYG{n}{OK}
\PYG{n}{Content}\PYG{o}{\PYGZhy{}}\PYG{n}{Type}\PYG{p}{:} \PYG{n}{application}\PYG{o}{/}\PYG{n}{json}

\PYG{p}{[}
    \PYG{p}{\PYGZob{}}
        \PYG{l+s+s1}{\PYGZsq{}}\PYG{l+s+s1}{key1}\PYG{l+s+s1}{\PYGZsq{}}\PYG{p}{:} \PYG{l+s+s1}{\PYGZsq{}}\PYG{l+s+s1}{value1}\PYG{l+s+s1}{\PYGZsq{}}\PYG{p}{,}
        \PYG{l+s+s1}{\PYGZsq{}}\PYG{l+s+s1}{key2}\PYG{l+s+s1}{\PYGZsq{}}\PYG{p}{:} \PYG{l+s+s1}{\PYGZsq{}}\PYG{l+s+s1}{value2}\PYG{l+s+s1}{\PYGZsq{}}
    \PYG{p}{\PYGZcb{}}\PYG{p}{,}
    \PYG{p}{\PYGZob{}}
        \PYG{l+s+s1}{\PYGZsq{}}\PYG{l+s+s1}{key1}\PYG{l+s+s1}{\PYGZsq{}}\PYG{p}{:} \PYG{l+s+s1}{\PYGZsq{}}\PYG{l+s+s1}{value1}\PYG{l+s+s1}{\PYGZsq{}}\PYG{p}{,}
        \PYG{l+s+s1}{\PYGZsq{}}\PYG{l+s+s1}{key2}\PYG{l+s+s1}{\PYGZsq{}}\PYG{p}{:} \PYG{l+s+s1}{\PYGZsq{}}\PYG{l+s+s1}{value2}\PYG{l+s+s1}{\PYGZsq{}}
    \PYG{p}{\PYGZcb{}}
\PYG{p}{]}
\end{sphinxVerbatim}


\section{Cross Origin Resource Sharing}
\label{\detokenize{usage:cross-origin-resource-sharing}}
The API supports Cross Origin Resource Sharing (CORS) for AJAX requests from any origin. You can find further information in the \href{https://www.w3.org/TR/cors/}{CORS W3C Recommendation}.


\chapter{Extending the API}
\label{\detokenize{developers:extending-the-api}}\label{\detokenize{developers::doc}}

\section{Project Structure}
\label{\detokenize{developers:project-structure}}
The main building blocks of the Swarm Intelligence App are resources and models. The following project structure shows the separation of different modules. Any resources are located in the \sphinxstyleemphasis{resources/} folder; any models in the \sphinxstyleemphasis{models/} folder. Helpers used accross the application are located in the \sphinxstyleemphasis{common/} folder. The app is configured in \sphinxstyleemphasis{config.py} and initialized in \sphinxstyleemphasis{app.py}, which is the main entry point of the application.

\begin{sphinxVerbatim}[commandchars=\\\{\}]
\PYG{n}{swarm\PYGZus{}intelligence\PYGZus{}app}\PYG{o}{/}         \PYG{c+c1}{\PYGZsh{} application root directory}
    \PYG{n}{common}\PYG{o}{/}                     \PYG{c+c1}{\PYGZsh{} any helpers and utils}
        \PYG{n}{\PYGZus{}\PYGZus{}init}\PYG{o}{.}\PYG{n}{py\PYGZus{}\PYGZus{}}
        \PYG{n}{authentication}\PYG{o}{.}\PYG{n}{py}
    \PYG{n}{docs}\PYG{o}{/}                       \PYG{c+c1}{\PYGZsh{} any documentation source files}
    \PYG{n}{models}\PYG{o}{/}                     \PYG{c+c1}{\PYGZsh{} any models}
        \PYG{n}{\PYGZus{}\PYGZus{}init\PYGZus{}\PYGZus{}}\PYG{o}{.}\PYG{n}{py}
        \PYG{n}{accountability}\PYG{o}{.}\PYG{n}{py}
        \PYG{n}{circle}\PYG{o}{.}\PYG{n}{py}
        \PYG{n}{domain}\PYG{o}{.}\PYG{n}{py}
        \PYG{n}{invitation}\PYG{o}{.}\PYG{n}{py}
        \PYG{n}{organization}\PYG{o}{.}\PYG{n}{py}
        \PYG{n}{partner}\PYG{o}{.}\PYG{n}{py}
        \PYG{n}{policy}\PYG{o}{.}\PYG{n}{py}
        \PYG{n}{role}\PYG{o}{.}\PYG{n}{py}
        \PYG{n}{role\PYGZus{}member}\PYG{o}{.}\PYG{n}{py}
        \PYG{n}{user}\PYG{o}{.}\PYG{n}{py}
    \PYG{n}{resources}\PYG{o}{/}                  \PYG{c+c1}{\PYGZsh{} any resources}
        \PYG{n}{\PYGZus{}\PYGZus{}init\PYGZus{}\PYGZus{}}\PYG{o}{.}\PYG{n}{py}
        \PYG{n}{accountability}\PYG{o}{.}\PYG{n}{py}
        \PYG{n}{circle}\PYG{o}{.}\PYG{n}{py}
        \PYG{n}{domain}\PYG{o}{.}\PYG{n}{py}
        \PYG{n}{invitation}\PYG{o}{.}\PYG{n}{py}
        \PYG{n}{organization}\PYG{o}{.}\PYG{n}{py}
        \PYG{n}{partner}\PYG{o}{.}\PYG{n}{py}
        \PYG{n}{policy}\PYG{o}{.}\PYG{n}{py}
        \PYG{n}{role}\PYG{o}{.}\PYG{n}{py}
        \PYG{n}{user}\PYG{o}{.}\PYG{n}{py}
    \PYG{n}{tests}\PYG{o}{/}                      \PYG{c+c1}{\PYGZsh{} any tests}
    \PYG{n}{\PYGZus{}\PYGZus{}init\PYGZus{}\PYGZus{}}\PYG{o}{.}\PYG{n}{py}
    \PYG{n}{app}\PYG{o}{.}\PYG{n}{py}                      \PYG{c+c1}{\PYGZsh{} application entry point}
    \PYG{n}{config}\PYG{o}{.}\PYG{n}{py}                   \PYG{c+c1}{\PYGZsh{} application configuration}
\end{sphinxVerbatim}


\section{Adding a Resource}
\label{\detokenize{developers:adding-a-resource}}
Resources are implemented with \href{http://flask-restful-cn.readthedocs.io/en/0.3.5/}{Flask-RESTful}, an extension for \href{http://flask.pocoo.org}{Flask} that adds support for building RESTful APIs. A basic CRUD resource can be defined in \sphinxstyleemphasis{resources/myresource.py} and looks like this:

\begin{sphinxVerbatim}[commandchars=\\\{\}]
\PYG{k+kn}{from} \PYG{n+nn}{flask\PYGZus{}restful} \PYG{k}{import} \PYG{n}{Resource}

\PYG{k}{class} \PYG{n+nc}{MyResource}\PYG{p}{(}\PYG{n}{Resource}\PYG{p}{)}\PYG{p}{:}
    \PYG{k}{def} \PYG{n+nf}{post}\PYG{p}{(}\PYG{n+nb+bp}{self}\PYG{p}{)}\PYG{p}{:}             \PYG{c+c1}{\PYGZsh{} create a new resource}
        \PYG{o}{.}\PYG{o}{.}\PYG{o}{.}                     \PYG{c+c1}{\PYGZsh{} insert data}
        \PYG{k}{return} \PYG{l+m+mi}{201}\PYG{p}{,} \PYG{p}{\PYGZob{}}\PYG{p}{\PYGZcb{}}          \PYG{c+c1}{\PYGZsh{} return status 201 and JSON data}

    \PYG{k}{def} \PYG{n+nf}{get}\PYG{p}{(}\PYG{n+nb+bp}{self}\PYG{p}{,} \PYG{n+nb}{id}\PYG{p}{)}\PYG{p}{:}          \PYG{c+c1}{\PYGZsh{} read a resource}
        \PYG{o}{.}\PYG{o}{.}\PYG{o}{.}                     \PYG{c+c1}{\PYGZsh{} query data}
        \PYG{k}{return} \PYG{l+m+mi}{200}\PYG{p}{,} \PYG{p}{\PYGZob{}}\PYG{p}{\PYGZcb{}}          \PYG{c+c1}{\PYGZsh{} return status 200 and JSON data}

    \PYG{k}{def} \PYG{n+nf}{put}\PYG{p}{(}\PYG{n+nb+bp}{self}\PYG{p}{,} \PYG{n+nb}{id}\PYG{p}{)}\PYG{p}{:}          \PYG{c+c1}{\PYGZsh{} update a resource}
        \PYG{o}{.}\PYG{o}{.}\PYG{o}{.}                     \PYG{c+c1}{\PYGZsh{} update data}
        \PYG{k}{return} \PYG{l+m+mi}{200}\PYG{p}{,} \PYG{p}{\PYGZob{}}\PYG{p}{\PYGZcb{}}          \PYG{c+c1}{\PYGZsh{} return status 200 and JSON data}

    \PYG{k}{def} \PYG{n+nf}{delete}\PYG{p}{(}\PYG{n+nb+bp}{self}\PYG{p}{,} \PYG{n+nb}{id}\PYG{p}{)}\PYG{p}{:}       \PYG{c+c1}{\PYGZsh{} delete a resource}
        \PYG{o}{.}\PYG{o}{.}\PYG{o}{.}                     \PYG{c+c1}{\PYGZsh{} delete data}
        \PYG{k}{return} \PYG{l+m+mi}{204}\PYG{p}{,} \PYG{k+kc}{None}        \PYG{c+c1}{\PYGZsh{} return status 204}
\end{sphinxVerbatim}

In \sphinxstyleemphasis{app.py} import your resource class

\begin{sphinxVerbatim}[commandchars=\\\{\}]
\PYG{k+kn}{from} \PYG{n+nn}{swarm\PYGZus{}intelligence\PYGZus{}app}\PYG{n+nn}{.}\PYG{n+nn}{resources}\PYG{n+nn}{.}\PYG{n+nn}{myresource} \PYG{k}{import} \PYG{n}{MyResource}
\end{sphinxVerbatim}

and add it to the API object

\begin{sphinxVerbatim}[commandchars=\\\{\}]
\PYG{k}{def} \PYG{n+nf}{create\PYGZus{}app}\PYG{p}{(}\PYG{p}{)}\PYG{p}{:}
    \PYG{o}{.}\PYG{o}{.}\PYG{o}{.}
    \PYG{n}{api}\PYG{o}{.}\PYG{n}{add\PYGZus{}resource}\PYG{p}{(}\PYG{n}{MyResource}\PYG{p}{,} \PYG{l+s+s1}{\PYGZsq{}}\PYG{l+s+s1}{/myresource}\PYG{l+s+s1}{\PYGZsq{}}\PYG{p}{)}
    \PYG{o}{.}\PYG{o}{.}\PYG{o}{.}
\end{sphinxVerbatim}


\section{Adding a Model}
\label{\detokenize{developers:adding-a-model}}
The Swarm Intelligence App uses \href{http://flask-sqlalchemy.pocoo.org/2.1/}{Flask-SQLAlchemy}, an extension that provides support for \href{http://www.sqlalchemy.org/}{SQLAlchemy}. SQLAlchemy is an SQL toolkit and Object Relational Mapper for Python. A simple model can be defined in \sphinxstyleemphasis{models/mymodel.py} and looks like this:

\begin{sphinxVerbatim}[commandchars=\\\{\}]
\PYG{k+kn}{from} \PYG{n+nn}{swarm\PYGZus{}intelligence\PYGZus{}app}\PYG{n+nn}{.}\PYG{n+nn}{models} \PYG{k}{import} \PYG{n}{db}

\PYG{k}{class} \PYG{n+nc}{MyModel}\PYG{p}{(}\PYG{n}{db}\PYG{o}{.}\PYG{n}{Model}\PYG{p}{)}\PYG{p}{:}
    \PYG{n+nb}{id} \PYG{o}{=} \PYG{n}{db}\PYG{o}{.}\PYG{n}{Column}\PYG{p}{(}\PYG{n}{db}\PYG{o}{.}\PYG{n}{Integer}\PYG{p}{,} \PYG{n}{primary\PYGZus{}key}\PYG{o}{=}\PYG{k+kc}{True}\PYG{p}{)}
    \PYG{n}{firstname} \PYG{o}{=} \PYG{n}{db}\PYG{o}{.}\PYG{n}{Column}\PYG{p}{(}\PYG{n}{db}\PYG{o}{.}\PYG{n}{String}\PYG{p}{(}\PYG{l+m+mi}{100}\PYG{p}{)}\PYG{p}{,} \PYG{n}{nullable}\PYG{o}{=}\PYG{k+kc}{False}\PYG{p}{)}
    \PYG{n}{lastname} \PYG{o}{=} \PYG{n}{db}\PYG{o}{.}\PYG{n}{Column}\PYG{p}{(}\PYG{n}{db}\PYG{o}{.}\PYG{n}{String}\PYG{p}{(}\PYG{l+m+mi}{100}\PYG{p}{)}\PYG{p}{,} \PYG{n}{nullable}\PYG{o}{=}\PYG{k+kc}{False}\PYG{p}{)}

    \PYG{k}{def} \PYG{n+nf}{\PYGZus{}\PYGZus{}init\PYGZus{}\PYGZus{}}\PYG{p}{(}\PYG{n+nb+bp}{self}\PYG{p}{,} \PYG{n}{firstname}\PYG{p}{)}\PYG{p}{:}
        \PYG{n+nb+bp}{self}\PYG{o}{.}\PYG{n}{firstname} \PYG{o}{=} \PYG{n}{firstname}
        \PYG{n+nb+bp}{self}\PYG{o}{.}\PYG{n}{lastname} \PYG{o}{=} \PYG{n}{lastname}

    \PYG{k}{def} \PYG{n+nf}{\PYGZus{}\PYGZus{}repr\PYGZus{}\PYGZus{}}\PYG{p}{(}\PYG{n+nb+bp}{self}\PYG{p}{)}\PYG{p}{:}
        \PYG{k}{return} \PYG{l+s+s1}{\PYGZsq{}}\PYG{l+s+s1}{\PYGZlt{}MyModel }\PYG{l+s+si}{\PYGZpc{}r}\PYG{l+s+s1}{\PYGZgt{}}\PYG{l+s+s1}{\PYGZsq{}} \PYG{o}{\PYGZpc{}} \PYG{n+nb+bp}{self}\PYG{o}{.}\PYG{n}{id}

    \PYG{n+nd}{@property}
    \PYG{k}{def} \PYG{n+nf}{serialize}\PYG{p}{(}\PYG{n+nb+bp}{self}\PYG{p}{)}\PYG{p}{:}
        \PYG{k}{return} \PYG{p}{\PYGZob{}}
            \PYG{l+s+s1}{\PYGZsq{}}\PYG{l+s+s1}{firstname}\PYG{l+s+s1}{\PYGZsq{}}\PYG{p}{:} \PYG{n+nb+bp}{self}\PYG{o}{.}\PYG{n}{firstname}\PYG{p}{,}
            \PYG{l+s+s1}{\PYGZsq{}}\PYG{l+s+s1}{lastname}\PYG{l+s+s1}{\PYGZsq{}}\PYG{p}{:} \PYG{n+nb+bp}{self}\PYG{o}{.}\PYG{n}{lastname}
        \PYG{p}{\PYGZcb{}}
\end{sphinxVerbatim}

Import the SQLAlchemy object and your model class and use your model as follows:

\begin{sphinxVerbatim}[commandchars=\\\{\}]
\PYG{k+kn}{from} \PYG{n+nn}{swarm\PYGZus{}intelligence\PYGZus{}app}\PYG{n+nn}{.}\PYG{n+nn}{models} \PYG{k}{import} \PYG{n}{db}
\PYG{k+kn}{from} \PYG{n+nn}{swarm\PYGZus{}intelligence\PYGZus{}app}\PYG{n+nn}{.}\PYG{n+nn}{models}\PYG{n+nn}{.}\PYG{n+nn}{mymodel} \PYG{k}{import} \PYG{n}{MyModel}

\PYG{k}{try}\PYG{p}{:}
    \PYG{c+c1}{\PYGZsh{} insert data}
    \PYG{n}{mymodel} \PYG{o}{=} \PYG{n}{MyModel}\PYG{p}{(}\PYG{l+s+s1}{\PYGZsq{}}\PYG{l+s+s1}{John}\PYG{l+s+s1}{\PYGZsq{}}\PYG{p}{,} \PYG{l+s+s1}{\PYGZsq{}}\PYG{l+s+s1}{Doe}\PYG{l+s+s1}{\PYGZsq{}}\PYG{p}{)}
    \PYG{n}{db}\PYG{o}{.}\PYG{n}{session}\PYG{o}{.}\PYG{n}{add}\PYG{p}{(}\PYG{n}{mymodel}\PYG{p}{)}
    \PYG{n}{db}\PYG{o}{.}\PYG{n}{session}\PYG{o}{.}\PYG{n}{flush}\PYG{p}{(}\PYG{p}{)}

    \PYG{c+c1}{\PYGZsh{} query data}
    \PYG{n}{mymodel} \PYG{o}{=} \PYG{n}{MyModel}\PYG{o}{.}\PYG{n}{query}\PYG{o}{.}\PYG{n}{get}\PYG{p}{(}\PYG{n}{mymodel}\PYG{o}{.}\PYG{n}{id}\PYG{p}{)}

    \PYG{c+c1}{\PYGZsh{} update data}
    \PYG{n}{mymodel}\PYG{o}{.}\PYG{n}{lastname} \PYG{o}{=} \PYG{l+s+s1}{\PYGZsq{}}\PYG{l+s+s1}{Smith}\PYG{l+s+s1}{\PYGZsq{}}

    \PYG{c+c1}{\PYGZsh{} delete data}
    \PYG{n}{db}\PYG{o}{.}\PYG{n}{session}\PYG{o}{.}\PYG{n}{delete}\PYG{p}{(}\PYG{n}{mymodel}\PYG{p}{)}

    \PYG{c+c1}{\PYGZsh{} persist data}
    \PYG{n}{db}\PYG{o}{.}\PYG{n}{session}\PYG{o}{.}\PYG{n}{commit}\PYG{p}{(}\PYG{p}{)}
\PYG{k}{except}\PYG{p}{:}
    \PYG{n}{db}\PYG{o}{.}\PYG{n}{session}\PYG{o}{.}\PYG{n}{rollback}\PYG{p}{(}\PYG{p}{)}
\end{sphinxVerbatim}


\chapter{API Reference}
\label{\detokenize{resources/index:api-reference}}\label{\detokenize{resources/index::doc}}

\section{Accountability}
\label{\detokenize{resources/accountability:accountability}}\label{\detokenize{resources/accountability::doc}}\label{\detokenize{resources/accountability:id1}}
Represents an accountability.

\noindent\begin{tabulary}{\linewidth}{|L|L|L|}
\hline
\sphinxstylethead{\relax 
Resource
\unskip}\relax &\sphinxstylethead{\relax 
Operation
\unskip}\relax &\sphinxstylethead{\relax 
Description
\unskip}\relax \\
\hline
Accountability
&
{\hyperref[\detokenize{resources/accountability:put--accountabilities-(accountability_id)}]{\emph{PUT /accountabilities/(accountability\_id)}}}
&
Update an accountability.
\\
\hline&
{\hyperref[\detokenize{resources/accountability:delete--accountabilities-(accountability_id)}]{\emph{DELETE /accountabilities/(accountability\_id)}}}
&
Delete an accountability.
\\
\hline&
{\hyperref[\detokenize{resources/accountability:get--accountabilities-(accountability_id)}]{\emph{GET /accountabilities/(accountability\_id)}}}
&
Retrieve an accountability.
\\
\hline\end{tabulary}


\begin{DUlineblock}{0em}
\item[] 
\end{DUlineblock}


\begin{fulllineitems}
\phantomsection\label{\detokenize{resources/accountability:put--accountabilities-(accountability_id)}}\pysiglinewithargsret{\sphinxbfcode{PUT~}\sphinxbfcode{/accountabilities/}}{\emph{accountability\_id}}{}~
Update an accountability.

\sphinxstylestrong{Example request}:

\begin{sphinxVerbatim}[commandchars=\\\{\}]
PUT /accountabilities/1 HTTP/1.1
Host: example.com
Authorization: Bearer \PYGZlt{}token\PYGZgt{}
Content\PYGZhy{}Type: application/json

\PYGZob{}
    \PYGZsq{}title\PYGZsq{}: \PYGZsq{}Accountability\PYGZsq{}s new title\PYGZsq{}
\PYGZcb{}
\end{sphinxVerbatim}

\sphinxstylestrong{Example response}:

\begin{sphinxVerbatim}[commandchars=\\\{\}]
HTTP/1.1 200 OK
Content\PYGZhy{}Type: application/json

\PYGZob{}
    \PYGZsq{}id\PYGZsq{}: 1,
    \PYGZsq{}title\PYGZsq{}: \PYGZsq{}Accountability\PYGZsq{}s new title\PYGZsq{},
    \PYGZsq{}role\PYGZus{}id\PYGZsq{}: 1
\PYGZcb{}
\end{sphinxVerbatim}
\begin{quote}\begin{description}
\item[{Parameters}] \leavevmode\begin{itemize}
\item {} 
\sphinxstyleliteralstrong{accountability\_id} (\sphinxstyleliteralemphasis{int}) -- the accountability to update

\end{itemize}

\item[{Request Headers}] \leavevmode\begin{itemize}
\item {} 
\href{http://tools.ietf.org/html/rfc7235\#section-4.2}{Authorization} -- JSON Web Token to authenticate

\item {} 
\href{http://tools.ietf.org/html/rfc7231\#section-3.1.1.5}{Content-Type} -- data is sent as application/json or
application/x-www-form-urlencoded

\end{itemize}

\item[{Request JSON Object}] \leavevmode\begin{itemize}
\item {} 
\sphinxstyleliteralstrong{name} (\sphinxstyleliteralemphasis{string}) -- the accountability's title

\end{itemize}

\item[{Response Headers}] \leavevmode\begin{itemize}
\item {} 
\href{http://tools.ietf.org/html/rfc7231\#section-3.1.1.5}{Content-Type} -- data is received as application/json

\end{itemize}

\item[{Response JSON Object}] \leavevmode\begin{itemize}
\item {} 
\sphinxstyleliteralstrong{id} (\sphinxstyleliteralemphasis{int}) -- the accountability's unique id

\item {} 
\sphinxstyleliteralstrong{title} (\sphinxstyleliteralemphasis{string}) -- the accountability's title

\item {} 
\sphinxstyleliteralstrong{role\_id} (\sphinxstyleliteralemphasis{int}) -- the role the accountability is related to

\end{itemize}

\item[{Status Codes}] \leavevmode\begin{itemize}
\item {} 
\href{http://www.w3.org/Protocols/rfc2616/rfc2616-sec10.html\#sec10.2.1}{200 OK} -- Accountability is updated

\item {} 
\href{http://www.w3.org/Protocols/rfc2616/rfc2616-sec10.html\#sec10.4.1}{400 Bad Request} -- Parameters are missing

\item {} 
\href{http://www.w3.org/Protocols/rfc2616/rfc2616-sec10.html\#sec10.4.1}{400 Bad Request} -- Token is not well-formed

\item {} 
\href{http://www.w3.org/Protocols/rfc2616/rfc2616-sec10.html\#sec10.4.2}{401 Unauthorized} -- Token has expired

\item {} 
\href{http://www.w3.org/Protocols/rfc2616/rfc2616-sec10.html\#sec10.4.2}{401 Unauthorized} -- User is not authorized

\item {} 
\href{http://www.w3.org/Protocols/rfc2616/rfc2616-sec10.html\#sec10.4.5}{404 Not Found} -- Accountability is not found

\end{itemize}

\end{description}\end{quote}

\end{fulllineitems}



\begin{fulllineitems}
\phantomsection\label{\detokenize{resources/accountability:delete--accountabilities-(accountability_id)}}\pysiglinewithargsret{\sphinxbfcode{DELETE~}\sphinxbfcode{/accountabilities/}}{\emph{accountability\_id}}{}~
Delete an accountability.

\sphinxstylestrong{Example request}:

\begin{sphinxVerbatim}[commandchars=\\\{\}]
\PYG{n+nf}{DELETE} \PYG{n+nn}{/accountabilities/1} \PYG{k+kr}{HTTP}\PYG{o}{/}\PYG{l+m}{1.1}
\PYG{n+na}{Host}\PYG{o}{:} \PYG{l}{example.com}
\PYG{n+na}{Authorization}\PYG{o}{:} \PYG{l}{Bearer \PYGZlt{}token\PYGZgt{}}
\end{sphinxVerbatim}

\sphinxstylestrong{Example response}:

\begin{sphinxVerbatim}[commandchars=\\\{\}]
\PYG{k+kr}{HTTP}\PYG{o}{/}\PYG{l+m}{1.1} \PYG{l+m}{204} \PYG{n+ne}{No Content}
\end{sphinxVerbatim}
\begin{quote}\begin{description}
\item[{Parameters}] \leavevmode\begin{itemize}
\item {} 
\sphinxstyleliteralstrong{accountability\_id} (\sphinxstyleliteralemphasis{int}) -- the accountability to delete

\end{itemize}

\item[{Request Headers}] \leavevmode\begin{itemize}
\item {} 
\href{http://tools.ietf.org/html/rfc7235\#section-4.2}{Authorization} -- JSON Web Token to authenticate

\end{itemize}

\item[{Status Codes}] \leavevmode\begin{itemize}
\item {} 
\href{http://www.w3.org/Protocols/rfc2616/rfc2616-sec10.html\#sec10.2.5}{204 No Content} -- Accountability is deleted

\item {} 
\href{http://www.w3.org/Protocols/rfc2616/rfc2616-sec10.html\#sec10.4.1}{400 Bad Request} -- Token is not well-formed

\item {} 
\href{http://www.w3.org/Protocols/rfc2616/rfc2616-sec10.html\#sec10.4.2}{401 Unauthorized} -- Token has expired

\item {} 
\href{http://www.w3.org/Protocols/rfc2616/rfc2616-sec10.html\#sec10.4.2}{401 Unauthorized} -- User is not authorized

\item {} 
\href{http://www.w3.org/Protocols/rfc2616/rfc2616-sec10.html\#sec10.4.5}{404 Not Found} -- Accountability is not found

\end{itemize}

\end{description}\end{quote}

\end{fulllineitems}



\begin{fulllineitems}
\phantomsection\label{\detokenize{resources/accountability:get--accountabilities-(accountability_id)}}\pysiglinewithargsret{\sphinxbfcode{GET~}\sphinxbfcode{/accountabilities/}}{\emph{accountability\_id}}{}~
Retrieve an accountability.

\sphinxstylestrong{Example request}:

\begin{sphinxVerbatim}[commandchars=\\\{\}]
\PYG{n+nf}{GET} \PYG{n+nn}{/accountabilities/1} \PYG{k+kr}{HTTP}\PYG{o}{/}\PYG{l+m}{1.1}
\PYG{n+na}{Host}\PYG{o}{:} \PYG{l}{example.com}
\PYG{n+na}{Authorization}\PYG{o}{:} \PYG{l}{Bearer \PYGZlt{}token\PYGZgt{}}
\end{sphinxVerbatim}

\sphinxstylestrong{Example response}:

\begin{sphinxVerbatim}[commandchars=\\\{\}]
HTTP/1.1 200 OK
Content\PYGZhy{}Type: application/json

\PYGZob{}
    \PYGZsq{}id\PYGZsq{}: 1,
    \PYGZsq{}title\PYGZsq{}: \PYGZsq{}Accountability\PYGZsq{}s title\PYGZsq{},
    \PYGZsq{}role\PYGZus{}id\PYGZsq{}: 1
\PYGZcb{}
\end{sphinxVerbatim}
\begin{quote}\begin{description}
\item[{Parameters}] \leavevmode\begin{itemize}
\item {} 
\sphinxstyleliteralstrong{accountability\_id} (\sphinxstyleliteralemphasis{int}) -- the accountability to retrieve

\end{itemize}

\item[{Request Headers}] \leavevmode\begin{itemize}
\item {} 
\href{http://tools.ietf.org/html/rfc7235\#section-4.2}{Authorization} -- JSON Web Token to authenticate

\end{itemize}

\item[{Response Headers}] \leavevmode\begin{itemize}
\item {} 
\href{http://tools.ietf.org/html/rfc7231\#section-3.1.1.5}{Content-Type} -- data is received as application/json

\end{itemize}

\item[{Response JSON Object}] \leavevmode\begin{itemize}
\item {} 
\sphinxstyleliteralstrong{id} (\sphinxstyleliteralemphasis{int}) -- the accountability's unique id

\item {} 
\sphinxstyleliteralstrong{title} (\sphinxstyleliteralemphasis{string}) -- the accountability's title

\item {} 
\sphinxstyleliteralstrong{role\_id} (\sphinxstyleliteralemphasis{int}) -- the role the accountability is related to

\end{itemize}

\item[{Status Codes}] \leavevmode\begin{itemize}
\item {} 
\href{http://www.w3.org/Protocols/rfc2616/rfc2616-sec10.html\#sec10.2.1}{200 OK} -- Accountability is retrieved

\item {} 
\href{http://www.w3.org/Protocols/rfc2616/rfc2616-sec10.html\#sec10.4.1}{400 Bad Request} -- Token is not well-formed

\item {} 
\href{http://www.w3.org/Protocols/rfc2616/rfc2616-sec10.html\#sec10.4.2}{401 Unauthorized} -- Token has expired

\item {} 
\href{http://www.w3.org/Protocols/rfc2616/rfc2616-sec10.html\#sec10.4.2}{401 Unauthorized} -- User is not authorized

\item {} 
\href{http://www.w3.org/Protocols/rfc2616/rfc2616-sec10.html\#sec10.4.5}{404 Not Found} -- Accountability is not found

\end{itemize}

\end{description}\end{quote}

\end{fulllineitems}



\section{Circle}
\label{\detokenize{resources/circle:circle}}\label{\detokenize{resources/circle::doc}}\label{\detokenize{resources/circle:id1}}
Represents a circle. A circle is a {\hyperref[\detokenize{resources/role:role}]{\sphinxcrossref{\DUrole{std,std-ref}{Role}}}} that is broken down into sub roles. Every circle has core roles (e.g. facilitator, secretary, lead link, rep link, cross link) as well as custom roles. A {\hyperref[\detokenize{resources/partner:partner}]{\sphinxcrossref{\DUrole{std,std-ref}{Partner}}}} can be assigned to a circle as a core member.

\noindent\begin{tabulary}{\linewidth}{|L|L|L|}
\hline
\sphinxstylethead{\relax 
Resource
\unskip}\relax &\sphinxstylethead{\relax 
Operation
\unskip}\relax &\sphinxstylethead{\relax 
Description
\unskip}\relax \\
\hline
Circle
&
{\hyperref[\detokenize{resources/circle:put--circles-(circle_id)}]{\emph{PUT /circles/(circle\_id)}}}
&
Update a circle.
\\
\hline&
{\hyperref[\detokenize{resources/circle:get--circles-(circle_id)}]{\emph{GET /circles/(circle\_id)}}}
&
Retrieve a circle.
\\
\hline\end{tabulary}


\begin{DUlineblock}{0em}
\item[] 
\end{DUlineblock}


\begin{fulllineitems}
\phantomsection\label{\detokenize{resources/circle:put--circles-(circle_id)}}\pysiglinewithargsret{\sphinxbfcode{PUT~}\sphinxbfcode{/circles/}}{\emph{circle\_id}}{}~
Update a circle.

In order to update a circle, the authenticated user must be a partner
of the organization that the circle is associated with.

\sphinxstylestrong{Example request}:

\begin{sphinxVerbatim}[commandchars=\\\{\}]
PUT /circles/6 HTTP/1.1
Host: example.com
Authorization: Bearer \PYGZlt{}token\PYGZgt{}
Content\PYGZhy{}Type: application/json

\PYGZob{}
    \PYGZsq{}name\PYGZsq{}: \PYGZsq{}My Circle\PYGZsq{}s new name\PYGZsq{},
    \PYGZsq{}purpose\PYGZsq{}: \PYGZsq{}My Circle\PYGZsq{}s new purpose\PYGZsq{},
    \PYGZsq{}strategy\PYGZsq{}: \PYGZsq{}My Circle\PYGZsq{}s new strategy\PYGZsq{}
\PYGZcb{}
\end{sphinxVerbatim}

\sphinxstylestrong{Example response}:

\begin{sphinxVerbatim}[commandchars=\\\{\}]
HTTP/1.1 200 OK
Content\PYGZhy{}Type: application/json

\PYGZob{}
    \PYGZsq{}id\PYGZsq{}: 6,
    \PYGZsq{}type\PYGZsq{}: \PYGZsq{}circle\PYGZsq{},
    \PYGZsq{}name\PYGZsq{}: \PYGZsq{}My Circle\PYGZsq{}s new name\PYGZsq{},
    \PYGZsq{}purpose\PYGZsq{}: \PYGZsq{}My Circle\PYGZsq{}s new purpose\PYGZsq{},
    \PYGZsq{}strategy\PYGZsq{}: \PYGZsq{}My Circle\PYGZsq{}s new strategy\PYGZsq{},
    \PYGZsq{}parent\PYGZus{}role\PYGZus{}id\PYGZsq{}: 1,
    \PYGZsq{}organization\PYGZus{}id\PYGZsq{}: 1
\PYGZcb{}
\end{sphinxVerbatim}
\begin{quote}\begin{description}
\item[{Parameters}] \leavevmode\begin{itemize}
\item {} 
\sphinxstyleliteralstrong{circle\_id} (\sphinxstyleliteralemphasis{int}) -- the circle to update

\end{itemize}

\item[{Request Headers}] \leavevmode\begin{itemize}
\item {} 
\href{http://tools.ietf.org/html/rfc7235\#section-4.2}{Authorization} -- JSON Web Token to authenticate

\item {} 
\href{http://tools.ietf.org/html/rfc7231\#section-3.1.1.5}{Content-Type} -- data is sent as application/json or
application/x-www-form-urlencoded

\end{itemize}

\item[{Request JSON Object}] \leavevmode\begin{itemize}
\item {} 
\sphinxstyleliteralstrong{name} (\sphinxstyleliteralemphasis{string}) -- the circle's name

\item {} 
\sphinxstyleliteralstrong{purpose} (\sphinxstyleliteralemphasis{string}) -- the circle's purpose

\item {} 
\sphinxstyleliteralstrong{strategy} (\sphinxstyleliteralemphasis{string}) -- the circle's strategy

\end{itemize}

\item[{Response Headers}] \leavevmode\begin{itemize}
\item {} 
\href{http://tools.ietf.org/html/rfc7231\#section-3.1.1.5}{Content-Type} -- data is received as application/json

\end{itemize}

\item[{Response JSON Object}] \leavevmode\begin{itemize}
\item {} 
\sphinxstyleliteralstrong{id} (\sphinxstyleliteralemphasis{int}) -- the circle's unique id

\item {} 
\sphinxstyleliteralstrong{type} (\sphinxstyleliteralemphasis{string}) -- the circle's type

\item {} 
\sphinxstyleliteralstrong{name} (\sphinxstyleliteralemphasis{string}) -- the circle's name

\item {} 
\sphinxstyleliteralstrong{purpose} (\sphinxstyleliteralemphasis{string}) -- the circle's purpose

\item {} 
\sphinxstyleliteralstrong{strategy} (\sphinxstyleliteralemphasis{string}) -- the circle's strategy

\item {} 
\sphinxstyleliteralstrong{parent\_role\_id} (\sphinxstyleliteralemphasis{int}) -- the parent role the circle is related to

\item {} 
\sphinxstyleliteralstrong{organization\_id} (\sphinxstyleliteralemphasis{int}) -- the organization the circle is related to

\end{itemize}

\item[{Status Codes}] \leavevmode\begin{itemize}
\item {} 
\href{http://www.w3.org/Protocols/rfc2616/rfc2616-sec10.html\#sec10.2.1}{200 OK} -- Circle is updated

\item {} 
\href{http://www.w3.org/Protocols/rfc2616/rfc2616-sec10.html\#sec10.4.1}{400 Bad Request} -- Parameters are missing

\item {} 
\href{http://www.w3.org/Protocols/rfc2616/rfc2616-sec10.html\#sec10.4.1}{400 Bad Request} -- Token is not well-formed

\item {} 
\href{http://www.w3.org/Protocols/rfc2616/rfc2616-sec10.html\#sec10.4.2}{401 Unauthorized} -- Token has expired

\item {} 
\href{http://www.w3.org/Protocols/rfc2616/rfc2616-sec10.html\#sec10.4.2}{401 Unauthorized} -- User is not authorized

\item {} 
\href{http://www.w3.org/Protocols/rfc2616/rfc2616-sec10.html\#sec10.4.5}{404 Not Found} -- Circle is not found

\end{itemize}

\end{description}\end{quote}

\end{fulllineitems}



\begin{fulllineitems}
\phantomsection\label{\detokenize{resources/circle:get--circles-(circle_id)}}\pysiglinewithargsret{\sphinxbfcode{GET~}\sphinxbfcode{/circles/}}{\emph{circle\_id}}{}~
Retrieve a circle.

In order to retrieve a circle, the authenticated user must be a
partner of the organization that the circle is associated with.

\sphinxstylestrong{Example request}:

\begin{sphinxVerbatim}[commandchars=\\\{\}]
\PYG{n+nf}{GET} \PYG{n+nn}{/circles/6} \PYG{k+kr}{HTTP}\PYG{o}{/}\PYG{l+m}{1.1}
\PYG{n+na}{Host}\PYG{o}{:} \PYG{l}{example.com}
\PYG{n+na}{Authorization}\PYG{o}{:} \PYG{l}{Bearer \PYGZlt{}token\PYGZgt{}}
\end{sphinxVerbatim}

\sphinxstylestrong{Example response}:

\begin{sphinxVerbatim}[commandchars=\\\{\}]
HTTP/1.1 200 OK
Content\PYGZhy{}Type: application/json

\PYGZob{}
    \PYGZsq{}id\PYGZsq{}: 6,
    \PYGZsq{}type\PYGZsq{}: \PYGZsq{}circle\PYGZsq{},
    \PYGZsq{}name\PYGZsq{}: \PYGZsq{}Circle\PYGZsq{}s name\PYGZsq{},
    \PYGZsq{}purpose\PYGZsq{}: \PYGZsq{}Circle\PYGZsq{}s purpose\PYGZsq{},
    \PYGZsq{}strategy\PYGZsq{}: \PYGZsq{}Circle\PYGZsq{}s strategy\PYGZsq{},
    \PYGZsq{}parent\PYGZus{}role\PYGZus{}id\PYGZsq{}: 1,
    \PYGZsq{}organization\PYGZus{}id\PYGZsq{}: 1
\PYGZcb{}
\end{sphinxVerbatim}
\begin{quote}\begin{description}
\item[{Parameters}] \leavevmode\begin{itemize}
\item {} 
\sphinxstyleliteralstrong{circle\_id} (\sphinxstyleliteralemphasis{int}) -- the circle to retrieve

\end{itemize}

\item[{Request Headers}] \leavevmode\begin{itemize}
\item {} 
\href{http://tools.ietf.org/html/rfc7235\#section-4.2}{Authorization} -- JSON Web Token to authenticate

\end{itemize}

\item[{Response Headers}] \leavevmode\begin{itemize}
\item {} 
\href{http://tools.ietf.org/html/rfc7231\#section-3.1.1.5}{Content-Type} -- data is received as application/json

\end{itemize}

\item[{Response JSON Object}] \leavevmode\begin{itemize}
\item {} 
\sphinxstyleliteralstrong{id} (\sphinxstyleliteralemphasis{int}) -- the circle's unique id

\item {} 
\sphinxstyleliteralstrong{type} (\sphinxstyleliteralemphasis{string}) -- the circle's type

\item {} 
\sphinxstyleliteralstrong{name} (\sphinxstyleliteralemphasis{string}) -- the circle's name

\item {} 
\sphinxstyleliteralstrong{purpose} (\sphinxstyleliteralemphasis{string}) -- the circle's purpose

\item {} 
\sphinxstyleliteralstrong{strategy} (\sphinxstyleliteralemphasis{string}) -- the circle's optional strategy

\item {} 
\sphinxstyleliteralstrong{parent\_role\_id} (\sphinxstyleliteralemphasis{int}) -- the parent role the circle is related to

\item {} 
\sphinxstyleliteralstrong{organization\_id} (\sphinxstyleliteralemphasis{int}) -- the organization the circle is related to

\end{itemize}

\item[{Status Codes}] \leavevmode\begin{itemize}
\item {} 
\href{http://www.w3.org/Protocols/rfc2616/rfc2616-sec10.html\#sec10.2.1}{200 OK} -- Circle is retrieved

\item {} 
\href{http://www.w3.org/Protocols/rfc2616/rfc2616-sec10.html\#sec10.4.1}{400 Bad Request} -- Token is not well-formed

\item {} 
\href{http://www.w3.org/Protocols/rfc2616/rfc2616-sec10.html\#sec10.4.2}{401 Unauthorized} -- Token has expired

\item {} 
\href{http://www.w3.org/Protocols/rfc2616/rfc2616-sec10.html\#sec10.4.2}{401 Unauthorized} -- User is not authorized

\item {} 
\href{http://www.w3.org/Protocols/rfc2616/rfc2616-sec10.html\#sec10.4.5}{404 Not Found} -- Circle is not found

\end{itemize}

\end{description}\end{quote}

\end{fulllineitems}



\subsection{Members}
\label{\detokenize{resources/circle:members}}
Represents the members of a circle. See {\hyperref[\detokenize{resources/partner:partner}]{\sphinxcrossref{\DUrole{std,std-ref}{Partner}}}} for a description of a single member.

\noindent\begin{tabulary}{\linewidth}{|L|L|L|}
\hline
\sphinxstylethead{\relax 
Resource
\unskip}\relax &\sphinxstylethead{\relax 
Operation
\unskip}\relax &\sphinxstylethead{\relax 
Description
\unskip}\relax \\
\hline
Circle Members
&
{\hyperref[\detokenize{resources/circle:get--circles-(circle_id)-members}]{\emph{GET /circles/(circle\_id)/members}}}
&
List members of a circle.
\\
\hline&
{\hyperref[\detokenize{resources/circle:put--circles-(circle_id)-members-(partner_id)}]{\emph{PUT /circles/(circle\_id)/members/(partner\_id)}}}
&
Assign a partner to a circle.
\\
\hline&
{\hyperref[\detokenize{resources/circle:delete--circles-(circle_id)-members-(partner_id)}]{\emph{DELETE /circles/(circle\_id)/members/(partner\_id)}}}
&
Unassign a partner from a circle.
\\
\hline\end{tabulary}


\begin{DUlineblock}{0em}
\item[] 
\end{DUlineblock}


\begin{fulllineitems}
\phantomsection\label{\detokenize{resources/circle:get--circles-(circle_id)-members}}\pysiglinewithargsret{\sphinxbfcode{GET~}\sphinxbfcode{/circles/}}{\emph{circle\_id}}{\sphinxbfcode{/members}}~
List members of a circle.

In order to list the members of a circle, the authenticated user must
be a partner of the organization that the circle is associated with.

\sphinxstylestrong{Example request}:

\begin{sphinxVerbatim}[commandchars=\\\{\}]
\PYG{n+nf}{GET} \PYG{n+nn}{/circles/1/members} \PYG{k+kr}{HTTP}\PYG{o}{/}\PYG{l+m}{1.1}
\PYG{n+na}{Host}\PYG{o}{:} \PYG{l}{example.com}
\PYG{n+na}{Authorization}\PYG{o}{:} \PYG{l}{Bearer \PYGZlt{}token\PYGZgt{}}
\end{sphinxVerbatim}

\sphinxstylestrong{Example response}:

\begin{sphinxVerbatim}[commandchars=\\\{\}]
HTTP/1.1 200 OK
Content\PYGZhy{}Type: application/json

[
    \PYGZob{}
        \PYGZsq{}id\PYGZsq{}: 1,
        \PYGZsq{}type\PYGZsq{}: \PYGZsq{}admin\PYGZsq{},
        \PYGZsq{}firstname\PYGZsq{}: \PYGZsq{}John\PYGZsq{},
        \PYGZsq{}lastname\PYGZsq{}: \PYGZsq{}Doe\PYGZsq{},
        \PYGZsq{}email\PYGZsq{}: \PYGZsq{}john@example.org\PYGZsq{},
        \PYGZsq{}is\PYGZus{}active\PYGZsq{}: True,
        \PYGZsq{}user\PYGZus{}id\PYGZsq{}: 1,
        \PYGZsq{}organization\PYGZus{}id\PYGZsq{}: 1,
        \PYGZsq{}invitation\PYGZus{}id\PYGZsq{}: null
    \PYGZcb{}
]
\end{sphinxVerbatim}
\begin{quote}\begin{description}
\item[{Parameters}] \leavevmode\begin{itemize}
\item {} 
\sphinxstyleliteralstrong{circle\_id} (\sphinxstyleliteralemphasis{int}) -- the circle the members are listed for

\end{itemize}

\item[{Request Headers}] \leavevmode\begin{itemize}
\item {} 
\href{http://tools.ietf.org/html/rfc7235\#section-4.2}{Authorization} -- JSON Web Token to authenticate

\end{itemize}

\item[{Response Headers}] \leavevmode\begin{itemize}
\item {} 
\href{http://tools.ietf.org/html/rfc7231\#section-3.1.1.5}{Content-Type} -- data is received as application/json

\end{itemize}

\item[{Response JSON Array of Objects}] \leavevmode\begin{itemize}
\item {} 
\sphinxstyleliteralstrong{id} (\sphinxstyleliteralemphasis{int}) -- the partner's unique id

\item {} 
\sphinxstyleliteralstrong{type} (\sphinxstyleliteralemphasis{string}) -- the partner's type

\item {} 
\sphinxstyleliteralstrong{firstname} (\sphinxstyleliteralemphasis{string}) -- the partner's firstname

\item {} 
\sphinxstyleliteralstrong{lastname} (\sphinxstyleliteralemphasis{string}) -- the partner's lastname

\item {} 
\sphinxstyleliteralstrong{email} (\sphinxstyleliteralemphasis{string}) -- the partner's email address

\item {} 
\sphinxstyleliteralstrong{is\_active} (\sphinxstyleliteralemphasis{boolean}) -- the partner's status

\item {} 
\sphinxstyleliteralstrong{user\_id} (\sphinxstyleliteralemphasis{int}) -- the user account the partner is related to

\item {} 
\sphinxstyleliteralstrong{organization\_id} (\sphinxstyleliteralemphasis{int}) -- the organization the partner is
related to

\item {} 
\sphinxstyleliteralstrong{invitation\_id} (\sphinxstyleliteralemphasis{int}) -- the invitation the partner is related to

\end{itemize}

\item[{Status Codes}] \leavevmode\begin{itemize}
\item {} 
\href{http://www.w3.org/Protocols/rfc2616/rfc2616-sec10.html\#sec10.2.1}{200 OK} -- Members are listed

\item {} 
\href{http://www.w3.org/Protocols/rfc2616/rfc2616-sec10.html\#sec10.4.1}{400 Bad Request} -- Token is not well-formed

\item {} 
\href{http://www.w3.org/Protocols/rfc2616/rfc2616-sec10.html\#sec10.4.2}{401 Unauthorized} -- Token has expired

\item {} 
\href{http://www.w3.org/Protocols/rfc2616/rfc2616-sec10.html\#sec10.4.2}{401 Unauthorized} -- User is not authorized

\item {} 
\href{http://www.w3.org/Protocols/rfc2616/rfc2616-sec10.html\#sec10.4.5}{404 Not Found} -- Circle is not found

\end{itemize}

\end{description}\end{quote}

\end{fulllineitems}



\begin{fulllineitems}
\phantomsection\label{\detokenize{resources/circle:put--circles-(circle_id)-members-(partner_id)}}\pysiglinewithargsret{\sphinxbfcode{PUT~}\sphinxbfcode{/circles/}}{\emph{circle\_id}}{\sphinxbfcode{/members/}}{\emph{partner\_id}}{}~
Assign a partner to a circle.

In order to assign a partner to a circle, the authenticated user must
be an admin of the organization that the circle is associated with.

\sphinxstylestrong{Example request}:

\begin{sphinxVerbatim}[commandchars=\\\{\}]
\PYG{n+nf}{PUT} \PYG{n+nn}{/circles/1/members/1} \PYG{k+kr}{HTTP}\PYG{o}{/}\PYG{l+m}{1.1}
\PYG{n+na}{Host}\PYG{o}{:} \PYG{l}{example.com}
\PYG{n+na}{Authorization}\PYG{o}{:} \PYG{l}{Bearer \PYGZlt{}token\PYGZgt{}}
\end{sphinxVerbatim}

\sphinxstylestrong{Example response}:

\begin{sphinxVerbatim}[commandchars=\\\{\}]
\PYG{k+kr}{HTTP}\PYG{o}{/}\PYG{l+m}{1.1} \PYG{l+m}{204} \PYG{n+ne}{No Content}
\end{sphinxVerbatim}
\begin{quote}\begin{description}
\item[{Parameters}] \leavevmode\begin{itemize}
\item {} 
\sphinxstyleliteralstrong{circle\_id} (\sphinxstyleliteralemphasis{int}) -- the circle the partner is assigned to

\item {} 
\sphinxstyleliteralstrong{partner\_id} (\sphinxstyleliteralemphasis{int}) -- the partner who is assigned to the circle

\end{itemize}

\item[{Request Headers}] \leavevmode\begin{itemize}
\item {} 
\href{http://tools.ietf.org/html/rfc7235\#section-4.2}{Authorization} -- JSON Web Token to authenticate

\end{itemize}

\item[{Status Codes}] \leavevmode\begin{itemize}
\item {} 
\href{http://www.w3.org/Protocols/rfc2616/rfc2616-sec10.html\#sec10.2.5}{204 No Content} -- Partner is assigned to circle

\item {} 
\href{http://www.w3.org/Protocols/rfc2616/rfc2616-sec10.html\#sec10.4.1}{400 Bad Request} -- Token is not well-formed

\item {} 
\href{http://www.w3.org/Protocols/rfc2616/rfc2616-sec10.html\#sec10.4.2}{401 Unauthorized} -- Token has expired

\item {} 
\href{http://www.w3.org/Protocols/rfc2616/rfc2616-sec10.html\#sec10.4.2}{401 Unauthorized} -- User is not authorized

\item {} 
\href{http://www.w3.org/Protocols/rfc2616/rfc2616-sec10.html\#sec10.4.5}{404 Not Found} -- Circle is not found

\item {} 
\href{http://www.w3.org/Protocols/rfc2616/rfc2616-sec10.html\#sec10.4.5}{404 Not Found} -- Partner is not found

\item {} 
\href{http://www.w3.org/Protocols/rfc2616/rfc2616-sec10.html\#sec10.4.10}{409 Conflict} -- Circle is not associated with partner's organization

\end{itemize}

\end{description}\end{quote}

\end{fulllineitems}



\begin{fulllineitems}
\phantomsection\label{\detokenize{resources/circle:delete--circles-(circle_id)-members-(partner_id)}}\pysiglinewithargsret{\sphinxbfcode{DELETE~}\sphinxbfcode{/circles/}}{\emph{circle\_id}}{\sphinxbfcode{/members/}}{\emph{partner\_id}}{}~
Unassign a partner from a circle.

In order to unassign a partner from a circle, the authenticated user
must be an admin of the organization that the circle is associated
with.

\sphinxstylestrong{Example request}:

\begin{sphinxVerbatim}[commandchars=\\\{\}]
\PYG{n+nf}{DELETE} \PYG{n+nn}{/circles/1/members/1} \PYG{k+kr}{HTTP}\PYG{o}{/}\PYG{l+m}{1.1}
\PYG{n+na}{Host}\PYG{o}{:} \PYG{l}{example.com}
\PYG{n+na}{Authorization}\PYG{o}{:} \PYG{l}{Bearer \PYGZlt{}token\PYGZgt{}}
\end{sphinxVerbatim}

\sphinxstylestrong{Example response}:

\begin{sphinxVerbatim}[commandchars=\\\{\}]
\PYG{k+kr}{HTTP}\PYG{o}{/}\PYG{l+m}{1.1} \PYG{l+m}{204} \PYG{n+ne}{No Content}
\end{sphinxVerbatim}
\begin{quote}\begin{description}
\item[{Parameters}] \leavevmode\begin{itemize}
\item {} 
\sphinxstyleliteralstrong{circle\_id} (\sphinxstyleliteralemphasis{int}) -- the circle the partner is unassigned from

\item {} 
\sphinxstyleliteralstrong{partner\_id} (\sphinxstyleliteralemphasis{int}) -- the partner who is unassigned from the circle

\end{itemize}

\item[{Request Headers}] \leavevmode\begin{itemize}
\item {} 
\href{http://tools.ietf.org/html/rfc7235\#section-4.2}{Authorization} -- JSON Web Token to authenticate

\end{itemize}

\item[{Status Codes}] \leavevmode\begin{itemize}
\item {} 
\href{http://www.w3.org/Protocols/rfc2616/rfc2616-sec10.html\#sec10.2.5}{204 No Content} -- Partner is unassigned from circle

\item {} 
\href{http://www.w3.org/Protocols/rfc2616/rfc2616-sec10.html\#sec10.4.1}{400 Bad Request} -- Token is not well-formed

\item {} 
\href{http://www.w3.org/Protocols/rfc2616/rfc2616-sec10.html\#sec10.4.2}{401 Unauthorized} -- Token has expired

\item {} 
\href{http://www.w3.org/Protocols/rfc2616/rfc2616-sec10.html\#sec10.4.2}{401 Unauthorized} -- User is not authorized

\item {} 
\href{http://www.w3.org/Protocols/rfc2616/rfc2616-sec10.html\#sec10.4.5}{404 Not Found} -- Circle is not found

\item {} 
\href{http://www.w3.org/Protocols/rfc2616/rfc2616-sec10.html\#sec10.4.5}{404 Not Found} -- Partner is not found

\end{itemize}

\end{description}\end{quote}

\end{fulllineitems}



\subsection{Roles}
\label{\detokenize{resources/circle:roles}}
Represents the roles of a circle. See {\hyperref[\detokenize{resources/role:role}]{\sphinxcrossref{\DUrole{std,std-ref}{Role}}}} for a description of a single role.

\noindent\begin{tabulary}{\linewidth}{|L|L|L|}
\hline
\sphinxstylethead{\relax 
Resource
\unskip}\relax &\sphinxstylethead{\relax 
Operation
\unskip}\relax &\sphinxstylethead{\relax 
Description
\unskip}\relax \\
\hline
Circle Roles
&
{\hyperref[\detokenize{resources/circle:post--circles-(circle_id)-roles}]{\emph{POST /circles/(circle\_id)/roles}}}
&
Add a role to a circle.
\\
\hline&
{\hyperref[\detokenize{resources/circle:get--circles-(circle_id)-roles}]{\emph{GET /circles/(circle\_id)/roles}}}
&
List roles of a circle.
\\
\hline\end{tabulary}


\begin{DUlineblock}{0em}
\item[] 
\end{DUlineblock}


\begin{fulllineitems}
\phantomsection\label{\detokenize{resources/circle:post--circles-(circle_id)-roles}}\pysiglinewithargsret{\sphinxbfcode{POST~}\sphinxbfcode{/circles/}}{\emph{circle\_id}}{\sphinxbfcode{/roles}}~
Add a role to a circle.

\sphinxstylestrong{Example request}:

\begin{sphinxVerbatim}[commandchars=\\\{\}]
POST /circles/1/roles HTTP/1.1
Host: example.com
Authorization: Bearer \PYGZlt{}token\PYGZgt{}
Content\PYGZhy{}Type: application/json

\PYGZob{}
    \PYGZsq{}name\PYGZsq{}: \PYGZsq{}Role\PYGZsq{}s name\PYGZsq{},
    \PYGZsq{}purpose\PYGZsq{}: \PYGZsq{}Role\PYGZsq{}s purpose\PYGZsq{}
\PYGZcb{}
\end{sphinxVerbatim}

\sphinxstylestrong{Example response}:

\begin{sphinxVerbatim}[commandchars=\\\{\}]
HTTP/1.1 201 Created
Content\PYGZhy{}Type: application/json

\PYGZob{}
    \PYGZsq{}id\PYGZsq{}: 5,
    \PYGZsq{}type\PYGZsq{}: \PYGZsq{}custom\PYGZsq{},
    \PYGZsq{}name\PYGZsq{}: \PYGZsq{}My Role\PYGZsq{}s name\PYGZsq{},
    \PYGZsq{}purpose\PYGZsq{}: \PYGZsq{}My Role\PYGZsq{}s purpose\PYGZsq{},
    \PYGZsq{}parent\PYGZus{}role\PYGZus{}id\PYGZsq{}: 1,
    \PYGZsq{}organization\PYGZus{}id\PYGZsq{}: 1
\PYGZcb{}
\end{sphinxVerbatim}
\begin{quote}\begin{description}
\item[{Parameters}] \leavevmode\begin{itemize}
\item {} 
\sphinxstyleliteralstrong{circle\_id} (\sphinxstyleliteralemphasis{int}) -- the circle the role is added to

\end{itemize}

\item[{Request Headers}] \leavevmode\begin{itemize}
\item {} 
\href{http://tools.ietf.org/html/rfc7235\#section-4.2}{Authorization} -- JSON Web Token to authenticate

\item {} 
\href{http://tools.ietf.org/html/rfc7231\#section-3.1.1.5}{Content-Type} -- data is sent as application/json or
application/x-www-form-urlencoded

\end{itemize}

\item[{Request JSON Object}] \leavevmode\begin{itemize}
\item {} 
\sphinxstyleliteralstrong{name} (\sphinxstyleliteralemphasis{string}) -- the role's name

\item {} 
\sphinxstyleliteralstrong{purpose} (\sphinxstyleliteralemphasis{string}) -- the role's purpose

\end{itemize}

\item[{Response Headers}] \leavevmode\begin{itemize}
\item {} 
\href{http://tools.ietf.org/html/rfc7231\#section-3.1.1.5}{Content-Type} -- data is received as application/json

\end{itemize}

\item[{Response JSON Object}] \leavevmode\begin{itemize}
\item {} 
\sphinxstyleliteralstrong{id} (\sphinxstyleliteralemphasis{int}) -- the role's unique id

\item {} 
\sphinxstyleliteralstrong{type} (\sphinxstyleliteralemphasis{string}) -- the role's type

\item {} 
\sphinxstyleliteralstrong{name} (\sphinxstyleliteralemphasis{string}) -- the role's name

\item {} 
\sphinxstyleliteralstrong{purpose} (\sphinxstyleliteralemphasis{string}) -- the role's purpose

\item {} 
\sphinxstyleliteralstrong{parent\_role\_id} (\sphinxstyleliteralemphasis{int}) -- the parent role the role is related to

\item {} 
\sphinxstyleliteralstrong{organization\_id} (\sphinxstyleliteralemphasis{int}) -- the organization the role is related to

\end{itemize}

\item[{Status Codes}] \leavevmode\begin{itemize}
\item {} 
\href{http://www.w3.org/Protocols/rfc2616/rfc2616-sec10.html\#sec10.2.2}{201 Created} -- Role is added

\item {} 
\href{http://www.w3.org/Protocols/rfc2616/rfc2616-sec10.html\#sec10.4.1}{400 Bad Request} -- Parameters are missing

\item {} 
\href{http://www.w3.org/Protocols/rfc2616/rfc2616-sec10.html\#sec10.4.1}{400 Bad Request} -- Token is not well-formed

\item {} 
\href{http://www.w3.org/Protocols/rfc2616/rfc2616-sec10.html\#sec10.4.2}{401 Unauthorized} -- Token has expired

\item {} 
\href{http://www.w3.org/Protocols/rfc2616/rfc2616-sec10.html\#sec10.4.2}{401 Unauthorized} -- User is not authorized

\item {} 
\href{http://www.w3.org/Protocols/rfc2616/rfc2616-sec10.html\#sec10.4.5}{404 Not Found} -- Circle is not found

\end{itemize}

\end{description}\end{quote}

\end{fulllineitems}



\begin{fulllineitems}
\phantomsection\label{\detokenize{resources/circle:get--circles-(circle_id)-roles}}\pysiglinewithargsret{\sphinxbfcode{GET~}\sphinxbfcode{/circles/}}{\emph{circle\_id}}{\sphinxbfcode{/roles}}~
List roles of a circle.

\sphinxstylestrong{Example request}:

\begin{sphinxVerbatim}[commandchars=\\\{\}]
\PYG{n+nf}{GET} \PYG{n+nn}{/circles/1/roles} \PYG{k+kr}{HTTP}\PYG{o}{/}\PYG{l+m}{1.1}
\PYG{n+na}{Host}\PYG{o}{:} \PYG{l}{example.com}
\PYG{n+na}{Authorization}\PYG{o}{:} \PYG{l}{Bearer \PYGZlt{}token\PYGZgt{}}
\end{sphinxVerbatim}

\sphinxstylestrong{Example response}:

\begin{sphinxVerbatim}[commandchars=\\\{\}]
HTTP/1.1 200 OK
Content\PYGZhy{}Type: application/json

[
    \PYGZob{}
        \PYGZsq{}id\PYGZsq{}: 2,
        \PYGZsq{}type\PYGZsq{}: \PYGZsq{}lead\PYGZus{}link\PYGZsq{},
        \PYGZsq{}name\PYGZsq{}: \PYGZsq{}Lead Link\PYGZsq{}s name\PYGZsq{},
        \PYGZsq{}purpose\PYGZsq{}: \PYGZsq{}Lead Link\PYGZsq{}s purpose\PYGZsq{},
        \PYGZsq{}parent\PYGZus{}role\PYGZus{}id\PYGZsq{}: 1,
        \PYGZsq{}organization\PYGZus{}id\PYGZsq{}: 1
    \PYGZcb{},
    \PYGZob{}
        \PYGZsq{}id\PYGZsq{}: 3,
        \PYGZsq{}type\PYGZsq{}: \PYGZsq{}secretary\PYGZsq{},
        \PYGZsq{}name\PYGZsq{}: \PYGZsq{}Secretary\PYGZsq{}s name\PYGZsq{},
        \PYGZsq{}purpose\PYGZsq{}: \PYGZsq{}Secretary\PYGZsq{}s purpose\PYGZsq{},
        \PYGZsq{}parent\PYGZus{}role\PYGZus{}id\PYGZsq{}: 1,
        \PYGZsq{}organization\PYGZus{}id\PYGZsq{}: 1
    \PYGZcb{},
    \PYGZob{}
        \PYGZsq{}id\PYGZsq{}: 4,
        \PYGZsq{}type\PYGZsq{}: \PYGZsq{}facilitator\PYGZsq{},
        \PYGZsq{}name\PYGZsq{}: \PYGZsq{}Facilitator\PYGZsq{}s name\PYGZsq{},
        \PYGZsq{}purpose\PYGZsq{}: \PYGZsq{}Facilitator\PYGZsq{}s purpose\PYGZsq{},
        \PYGZsq{}parent\PYGZus{}role\PYGZus{}id\PYGZsq{}: 1,
        \PYGZsq{}organization\PYGZus{}id\PYGZsq{}: 1
    \PYGZcb{},
    \PYGZob{}
        \PYGZsq{}id\PYGZsq{}: 5,
        \PYGZsq{}type\PYGZsq{}: \PYGZsq{}custom\PYGZsq{},
        \PYGZsq{}name\PYGZsq{}: \PYGZsq{}My Role\PYGZsq{}s name\PYGZsq{},
        \PYGZsq{}purpose\PYGZsq{}: \PYGZsq{}My Role\PYGZsq{}s purpose\PYGZsq{},
        \PYGZsq{}parent\PYGZus{}role\PYGZus{}id\PYGZsq{}: 1,
        \PYGZsq{}organization\PYGZus{}id\PYGZsq{}: 1
    \PYGZcb{},
    \PYGZob{}
        \PYGZsq{}id\PYGZsq{}: 6,
        \PYGZsq{}type\PYGZsq{}: \PYGZsq{}circle\PYGZsq{},
        \PYGZsq{}name\PYGZsq{}: \PYGZsq{}My Circle\PYGZsq{}s name\PYGZsq{},
        \PYGZsq{}purpose\PYGZsq{}: \PYGZsq{}My Circle\PYGZsq{}s purpose\PYGZsq{},
        \PYGZsq{}parent\PYGZus{}role\PYGZus{}id\PYGZsq{}: 1,
        \PYGZsq{}organization\PYGZus{}id\PYGZsq{}: 1
    \PYGZcb{}
]
\end{sphinxVerbatim}
\begin{quote}\begin{description}
\item[{Parameters}] \leavevmode\begin{itemize}
\item {} 
\sphinxstyleliteralstrong{circle\_id} (\sphinxstyleliteralemphasis{int}) -- the circle the roles are listed for

\end{itemize}

\item[{Request Headers}] \leavevmode\begin{itemize}
\item {} 
\href{http://tools.ietf.org/html/rfc7235\#section-4.2}{Authorization} -- JSON Web Token to authenticate

\end{itemize}

\item[{Response Headers}] \leavevmode\begin{itemize}
\item {} 
\href{http://tools.ietf.org/html/rfc7231\#section-3.1.1.5}{Content-Type} -- data is received as application/json

\end{itemize}

\item[{Response JSON Array of Objects}] \leavevmode\begin{itemize}
\item {} 
\sphinxstyleliteralstrong{id} (\sphinxstyleliteralemphasis{int}) -- the role's unique id

\item {} 
\sphinxstyleliteralstrong{type} (\sphinxstyleliteralemphasis{string}) -- the role's type

\item {} 
\sphinxstyleliteralstrong{name} (\sphinxstyleliteralemphasis{string}) -- the role's name

\item {} 
\sphinxstyleliteralstrong{purpose} (\sphinxstyleliteralemphasis{string}) -- the role's purpose

\item {} 
\sphinxstyleliteralstrong{parent\_role\_id} (\sphinxstyleliteralemphasis{int}) -- the parent role the role is related to

\item {} 
\sphinxstyleliteralstrong{organization\_id} (\sphinxstyleliteralemphasis{int}) -- the organization the role is related to

\end{itemize}

\item[{Status Codes}] \leavevmode\begin{itemize}
\item {} 
\href{http://www.w3.org/Protocols/rfc2616/rfc2616-sec10.html\#sec10.2.1}{200 OK} -- Roles are listed

\item {} 
\href{http://www.w3.org/Protocols/rfc2616/rfc2616-sec10.html\#sec10.4.1}{400 Bad Request} -- Token is not well-formed

\item {} 
\href{http://www.w3.org/Protocols/rfc2616/rfc2616-sec10.html\#sec10.4.2}{401 Unauthorized} -- Token has expired

\item {} 
\href{http://www.w3.org/Protocols/rfc2616/rfc2616-sec10.html\#sec10.4.2}{401 Unauthorized} -- User is not authorized

\item {} 
\href{http://www.w3.org/Protocols/rfc2616/rfc2616-sec10.html\#sec10.4.5}{404 Not Found} -- Circle is not found

\end{itemize}

\end{description}\end{quote}

\end{fulllineitems}



\section{Domain}
\label{\detokenize{resources/domain:domain}}\label{\detokenize{resources/domain::doc}}\label{\detokenize{resources/domain:id1}}
Represents a domain.

\noindent\begin{tabulary}{\linewidth}{|L|L|L|}
\hline
\sphinxstylethead{\relax 
Resource
\unskip}\relax &\sphinxstylethead{\relax 
Operation
\unskip}\relax &\sphinxstylethead{\relax 
Description
\unskip}\relax \\
\hline
Domain
&
{\hyperref[\detokenize{resources/domain:delete--domains-(domain_id)}]{\emph{DELETE /domains/(domain\_id)}}}
&
Delete a domain.
\\
\hline&
{\hyperref[\detokenize{resources/domain:get--domains-(domain_id)}]{\emph{GET /domains/(domain\_id)}}}
&
Retrieve a domain.
\\
\hline
Role
&
{\hyperref[\detokenize{resources/domain:put--domains-(domain_id)}]{\emph{PUT /domains/(domain\_id)}}}
&
Update a domain.
\\
\hline\end{tabulary}


\begin{DUlineblock}{0em}
\item[] 
\end{DUlineblock}


\begin{fulllineitems}
\phantomsection\label{\detokenize{resources/domain:put--domains-(domain_id)}}\pysiglinewithargsret{\sphinxbfcode{PUT~}\sphinxbfcode{/domains/}}{\emph{domain\_id}}{}~
Update a domain.

\sphinxstylestrong{Example request}:

\begin{sphinxVerbatim}[commandchars=\\\{\}]
PUT /domains/1 HTTP/1.1
Host: example.com
Authorization: Bearer \PYGZlt{}token\PYGZgt{}
Content\PYGZhy{}Type: application/json

\PYGZob{}
    \PYGZsq{}title\PYGZsq{}: \PYGZsq{}Domain\PYGZsq{}s new title\PYGZsq{}
\PYGZcb{}
\end{sphinxVerbatim}

\sphinxstylestrong{Example response}:

\begin{sphinxVerbatim}[commandchars=\\\{\}]
HTTP/1.1 200 OK
Content\PYGZhy{}Type: application/json

\PYGZob{}
    \PYGZsq{}id\PYGZsq{}: 1,
    \PYGZsq{}title\PYGZsq{}: \PYGZsq{}Domain\PYGZsq{}s new title\PYGZsq{},
    \PYGZsq{}role\PYGZus{}id\PYGZsq{}: 1
\PYGZcb{}
\end{sphinxVerbatim}
\begin{quote}\begin{description}
\item[{Parameters}] \leavevmode\begin{itemize}
\item {} 
\sphinxstyleliteralstrong{domain\_id} (\sphinxstyleliteralemphasis{int}) -- the domain to update

\end{itemize}

\item[{Request Headers}] \leavevmode\begin{itemize}
\item {} 
\href{http://tools.ietf.org/html/rfc7235\#section-4.2}{Authorization} -- JSON Web Token to authenticate

\item {} 
\href{http://tools.ietf.org/html/rfc7231\#section-3.1.1.5}{Content-Type} -- data is sent as application/json or
application/x-www-form-urlencoded

\end{itemize}

\item[{Request JSON Object}] \leavevmode\begin{itemize}
\item {} 
\sphinxstyleliteralstrong{name} (\sphinxstyleliteralemphasis{string}) -- the domain's title

\end{itemize}

\item[{Response Headers}] \leavevmode\begin{itemize}
\item {} 
\href{http://tools.ietf.org/html/rfc7231\#section-3.1.1.5}{Content-Type} -- data is received as application/json

\end{itemize}

\item[{Response JSON Object}] \leavevmode\begin{itemize}
\item {} 
\sphinxstyleliteralstrong{id} (\sphinxstyleliteralemphasis{int}) -- the domain's unique id

\item {} 
\sphinxstyleliteralstrong{title} (\sphinxstyleliteralemphasis{string}) -- the domain's title

\item {} 
\sphinxstyleliteralstrong{role\_id} (\sphinxstyleliteralemphasis{int}) -- the domain the role is related to

\end{itemize}

\item[{Status Codes}] \leavevmode\begin{itemize}
\item {} 
\href{http://www.w3.org/Protocols/rfc2616/rfc2616-sec10.html\#sec10.2.1}{200 OK} -- Domain is updated

\item {} 
\href{http://www.w3.org/Protocols/rfc2616/rfc2616-sec10.html\#sec10.4.1}{400 Bad Request} -- Parameters are missing

\item {} 
\href{http://www.w3.org/Protocols/rfc2616/rfc2616-sec10.html\#sec10.4.1}{400 Bad Request} -- Token is not well-formed

\item {} 
\href{http://www.w3.org/Protocols/rfc2616/rfc2616-sec10.html\#sec10.4.2}{401 Unauthorized} -- Token has expired

\item {} 
\href{http://www.w3.org/Protocols/rfc2616/rfc2616-sec10.html\#sec10.4.2}{401 Unauthorized} -- User is not authorized

\item {} 
\href{http://www.w3.org/Protocols/rfc2616/rfc2616-sec10.html\#sec10.4.5}{404 Not Found} -- Domain is not found

\end{itemize}

\end{description}\end{quote}

\end{fulllineitems}



\begin{fulllineitems}
\phantomsection\label{\detokenize{resources/domain:delete--domains-(domain_id)}}\pysiglinewithargsret{\sphinxbfcode{DELETE~}\sphinxbfcode{/domains/}}{\emph{domain\_id}}{}~
Delete a domain.

\sphinxstylestrong{Example request}:

\begin{sphinxVerbatim}[commandchars=\\\{\}]
\PYG{n+nf}{DELETE} \PYG{n+nn}{/domain/1} \PYG{k+kr}{HTTP}\PYG{o}{/}\PYG{l+m}{1.1}
\PYG{n+na}{Host}\PYG{o}{:} \PYG{l}{example.com}
\PYG{n+na}{Authorization}\PYG{o}{:} \PYG{l}{Bearer \PYGZlt{}token\PYGZgt{}}
\end{sphinxVerbatim}

\sphinxstylestrong{Example response}:

\begin{sphinxVerbatim}[commandchars=\\\{\}]
\PYG{k+kr}{HTTP}\PYG{o}{/}\PYG{l+m}{1.1} \PYG{l+m}{204} \PYG{n+ne}{No Content}
\end{sphinxVerbatim}
\begin{quote}\begin{description}
\item[{Parameters}] \leavevmode\begin{itemize}
\item {} 
\sphinxstyleliteralstrong{domain\_id} (\sphinxstyleliteralemphasis{int}) -- the domain to delete

\end{itemize}

\item[{Request Headers}] \leavevmode\begin{itemize}
\item {} 
\href{http://tools.ietf.org/html/rfc7235\#section-4.2}{Authorization} -- JSON Web Token to authenticate

\end{itemize}

\item[{Status Codes}] \leavevmode\begin{itemize}
\item {} 
\href{http://www.w3.org/Protocols/rfc2616/rfc2616-sec10.html\#sec10.2.5}{204 No Content} -- Domain is deleted

\item {} 
\href{http://www.w3.org/Protocols/rfc2616/rfc2616-sec10.html\#sec10.4.1}{400 Bad Request} -- Token is not well-formed

\item {} 
\href{http://www.w3.org/Protocols/rfc2616/rfc2616-sec10.html\#sec10.4.2}{401 Unauthorized} -- Token has expired

\item {} 
\href{http://www.w3.org/Protocols/rfc2616/rfc2616-sec10.html\#sec10.4.2}{401 Unauthorized} -- User is not authorized

\item {} 
\href{http://www.w3.org/Protocols/rfc2616/rfc2616-sec10.html\#sec10.4.5}{404 Not Found} -- Domain is not found

\end{itemize}

\end{description}\end{quote}

\end{fulllineitems}



\begin{fulllineitems}
\phantomsection\label{\detokenize{resources/domain:get--domains-(domain_id)}}\pysiglinewithargsret{\sphinxbfcode{GET~}\sphinxbfcode{/domains/}}{\emph{domain\_id}}{}~
Retrieve a domain.

\sphinxstylestrong{Example request}:

\begin{sphinxVerbatim}[commandchars=\\\{\}]
\PYG{n+nf}{GET} \PYG{n+nn}{/domains/1} \PYG{k+kr}{HTTP}\PYG{o}{/}\PYG{l+m}{1.1}
\PYG{n+na}{Host}\PYG{o}{:} \PYG{l}{example.com}
\PYG{n+na}{Authorization}\PYG{o}{:} \PYG{l}{Bearer \PYGZlt{}token\PYGZgt{}}
\end{sphinxVerbatim}

\sphinxstylestrong{Example response}:

\begin{sphinxVerbatim}[commandchars=\\\{\}]
HTTP/1.1 200 OK
Content\PYGZhy{}Type: application/json

\PYGZob{}
    \PYGZsq{}id\PYGZsq{}: 1,
    \PYGZsq{}title\PYGZsq{}: \PYGZsq{}Domain\PYGZsq{}s title\PYGZsq{},
    \PYGZsq{}role\PYGZus{}id\PYGZsq{}: 1
\PYGZcb{}
\end{sphinxVerbatim}
\begin{quote}\begin{description}
\item[{Parameters}] \leavevmode\begin{itemize}
\item {} 
\sphinxstyleliteralstrong{domain\_id} (\sphinxstyleliteralemphasis{int}) -- the domain to retrieve

\end{itemize}

\item[{Request Headers}] \leavevmode\begin{itemize}
\item {} 
\href{http://tools.ietf.org/html/rfc7235\#section-4.2}{Authorization} -- JSON Web Token to authenticate

\end{itemize}

\item[{Response Headers}] \leavevmode\begin{itemize}
\item {} 
\href{http://tools.ietf.org/html/rfc7231\#section-3.1.1.5}{Content-Type} -- data is received as application/json

\end{itemize}

\item[{Response JSON Object}] \leavevmode\begin{itemize}
\item {} 
\sphinxstyleliteralstrong{id} (\sphinxstyleliteralemphasis{int}) -- the domain's unique id

\item {} 
\sphinxstyleliteralstrong{title} (\sphinxstyleliteralemphasis{string}) -- the domain's title

\item {} 
\sphinxstyleliteralstrong{role\_id} (\sphinxstyleliteralemphasis{int}) -- the role the domain is related to

\end{itemize}

\item[{Status Codes}] \leavevmode\begin{itemize}
\item {} 
\href{http://www.w3.org/Protocols/rfc2616/rfc2616-sec10.html\#sec10.2.1}{200 OK} -- Domain is retrieved

\item {} 
\href{http://www.w3.org/Protocols/rfc2616/rfc2616-sec10.html\#sec10.4.1}{400 Bad Request} -- Token is not well-formed

\item {} 
\href{http://www.w3.org/Protocols/rfc2616/rfc2616-sec10.html\#sec10.4.2}{401 Unauthorized} -- Token has expired

\item {} 
\href{http://www.w3.org/Protocols/rfc2616/rfc2616-sec10.html\#sec10.4.2}{401 Unauthorized} -- User is not authorized

\item {} 
\href{http://www.w3.org/Protocols/rfc2616/rfc2616-sec10.html\#sec10.4.5}{404 Not Found} -- Domain is not found

\end{itemize}

\end{description}\end{quote}

\end{fulllineitems}



\subsection{Policies}
\label{\detokenize{resources/domain:policies}}
Represents the policies of a domain. See {\hyperref[\detokenize{resources/policy:policy}]{\sphinxcrossref{\DUrole{std,std-ref}{Policy}}}} for a description of a single policy.

\noindent\begin{tabulary}{\linewidth}{|L|L|L|}
\hline
\sphinxstylethead{\relax 
Resource
\unskip}\relax &\sphinxstylethead{\relax 
Operation
\unskip}\relax &\sphinxstylethead{\relax 
Description
\unskip}\relax \\
\hline
Domain Policies
&
{\hyperref[\detokenize{resources/domain:post--domains-(domain_id)-policies}]{\emph{POST /domains/(domain\_id)/policies}}}
&
Add a policy to a domain.
\\
\hline&
{\hyperref[\detokenize{resources/domain:get--domains-(domain_id)-policies}]{\emph{GET /domains/(domain\_id)/policies}}}
&
List policies of a domain.
\\
\hline\end{tabulary}


\begin{DUlineblock}{0em}
\item[] 
\end{DUlineblock}


\begin{fulllineitems}
\phantomsection\label{\detokenize{resources/domain:post--domains-(domain_id)-policies}}\pysiglinewithargsret{\sphinxbfcode{POST~}\sphinxbfcode{/domains/}}{\emph{domain\_id}}{\sphinxbfcode{/policies}}~
Add a policy to a domain.

\sphinxstylestrong{Example request}:

\begin{sphinxVerbatim}[commandchars=\\\{\}]
POST /domains/1/policies HTTP/1.1
Host: example.com
Authorization: Bearer \PYGZlt{}token\PYGZgt{}
Content\PYGZhy{}Type: application/json

\PYGZob{}
    \PYGZsq{}title\PYGZsq{}: \PYGZsq{}Policy\PYGZsq{}s title\PYGZsq{}
\PYGZcb{}
\end{sphinxVerbatim}

\sphinxstylestrong{Example response}:

\begin{sphinxVerbatim}[commandchars=\\\{\}]
HTTP/1.1 201 Created
Content\PYGZhy{}Type: application/json

\PYGZob{}
    \PYGZsq{}id\PYGZsq{}: 1,
    \PYGZsq{}title\PYGZsq{}: \PYGZsq{}Policy\PYGZsq{}s title\PYGZsq{},
    \PYGZsq{}domain\PYGZus{}id\PYGZsq{}: 1
\PYGZcb{}
\end{sphinxVerbatim}
\begin{quote}\begin{description}
\item[{Parameters}] \leavevmode\begin{itemize}
\item {} 
\sphinxstyleliteralstrong{organization\_id} (\sphinxstyleliteralemphasis{int}) -- the domain the policy is added to

\end{itemize}

\item[{Request Headers}] \leavevmode\begin{itemize}
\item {} 
\href{http://tools.ietf.org/html/rfc7235\#section-4.2}{Authorization} -- JSON Web Token to authenticate

\item {} 
\href{http://tools.ietf.org/html/rfc7231\#section-3.1.1.5}{Content-Type} -- data is sent as application/json or
application/x-www-form-urlencoded

\end{itemize}

\item[{Request JSON Object}] \leavevmode\begin{itemize}
\item {} 
\sphinxstyleliteralstrong{title} (\sphinxstyleliteralemphasis{string}) -- the policy's title

\end{itemize}

\item[{Response Headers}] \leavevmode\begin{itemize}
\item {} 
\href{http://tools.ietf.org/html/rfc7231\#section-3.1.1.5}{Content-Type} -- data is received as application/json

\end{itemize}

\item[{Response JSON Object}] \leavevmode\begin{itemize}
\item {} 
\sphinxstyleliteralstrong{id} (\sphinxstyleliteralemphasis{int}) -- the policy's unique id

\item {} 
\sphinxstyleliteralstrong{title} (\sphinxstyleliteralemphasis{string}) -- the policy's title

\item {} 
\sphinxstyleliteralstrong{domain\_id} (\sphinxstyleliteralemphasis{int}) -- the domain the policy is related to

\end{itemize}

\item[{Status Codes}] \leavevmode\begin{itemize}
\item {} 
\href{http://www.w3.org/Protocols/rfc2616/rfc2616-sec10.html\#sec10.2.2}{201 Created} -- Policy is added to domain

\item {} 
\href{http://www.w3.org/Protocols/rfc2616/rfc2616-sec10.html\#sec10.4.1}{400 Bad Request} -- Parameters are missing

\item {} 
\href{http://www.w3.org/Protocols/rfc2616/rfc2616-sec10.html\#sec10.4.1}{400 Bad Request} -- Token is not well-formed

\item {} 
\href{http://www.w3.org/Protocols/rfc2616/rfc2616-sec10.html\#sec10.4.2}{401 Unauthorized} -- Token has expired

\item {} 
\href{http://www.w3.org/Protocols/rfc2616/rfc2616-sec10.html\#sec10.4.2}{401 Unauthorized} -- User is not authorized

\item {} 
\href{http://www.w3.org/Protocols/rfc2616/rfc2616-sec10.html\#sec10.4.5}{404 Not Found} -- Domain is not found

\end{itemize}

\end{description}\end{quote}

\end{fulllineitems}



\begin{fulllineitems}
\phantomsection\label{\detokenize{resources/domain:get--domains-(domain_id)-policies}}\pysiglinewithargsret{\sphinxbfcode{GET~}\sphinxbfcode{/domains/}}{\emph{domain\_id}}{\sphinxbfcode{/policies}}~
List policies of a domain.

\sphinxstylestrong{Example request}:

\begin{sphinxVerbatim}[commandchars=\\\{\}]
\PYG{n+nf}{GET} \PYG{n+nn}{/domains/1/policies} \PYG{k+kr}{HTTP}\PYG{o}{/}\PYG{l+m}{1.1}
\PYG{n+na}{Host}\PYG{o}{:} \PYG{l}{example.com}
\PYG{n+na}{Authorization}\PYG{o}{:} \PYG{l}{Bearer \PYGZlt{}token\PYGZgt{}}
\end{sphinxVerbatim}

\sphinxstylestrong{Example response}:

\begin{sphinxVerbatim}[commandchars=\\\{\}]
HTTP/1.1 200 OK
Content\PYGZhy{}Type: application/json

[
    \PYGZob{}
        \PYGZsq{}id\PYGZsq{}: 1,
        \PYGZsq{}title\PYGZsq{}: \PYGZsq{}Policy\PYGZsq{}s title\PYGZsq{},
        \PYGZsq{}domain\PYGZus{}id\PYGZsq{}: 1
    \PYGZcb{}
]
\end{sphinxVerbatim}
\begin{quote}\begin{description}
\item[{Parameters}] \leavevmode\begin{itemize}
\item {} 
\sphinxstyleliteralstrong{domain\_id} (\sphinxstyleliteralemphasis{int}) -- the domain the policies are listed for

\end{itemize}

\item[{Request Headers}] \leavevmode\begin{itemize}
\item {} 
\href{http://tools.ietf.org/html/rfc7235\#section-4.2}{Authorization} -- JSON Web Token to authenticate

\end{itemize}

\item[{Response Headers}] \leavevmode\begin{itemize}
\item {} 
\href{http://tools.ietf.org/html/rfc7231\#section-3.1.1.5}{Content-Type} -- data is received as application/json

\end{itemize}

\item[{Response JSON Array of Objects}] \leavevmode\begin{itemize}
\item {} 
\sphinxstyleliteralstrong{id} (\sphinxstyleliteralemphasis{int}) -- the policy's unique id

\item {} 
\sphinxstyleliteralstrong{title} (\sphinxstyleliteralemphasis{string}) -- the policy's title

\item {} 
\sphinxstyleliteralstrong{domain\_id} (\sphinxstyleliteralemphasis{int}) -- the domain the policy is related to

\end{itemize}

\item[{Status Codes}] \leavevmode\begin{itemize}
\item {} 
\href{http://www.w3.org/Protocols/rfc2616/rfc2616-sec10.html\#sec10.2.1}{200 OK} -- Policies are listed

\item {} 
\href{http://www.w3.org/Protocols/rfc2616/rfc2616-sec10.html\#sec10.4.1}{400 Bad Request} -- Token is not well-formed

\item {} 
\href{http://www.w3.org/Protocols/rfc2616/rfc2616-sec10.html\#sec10.4.2}{401 Unauthorized} -- Token has expired

\item {} 
\href{http://www.w3.org/Protocols/rfc2616/rfc2616-sec10.html\#sec10.4.2}{401 Unauthorized} -- User is not authorized

\item {} 
\href{http://www.w3.org/Protocols/rfc2616/rfc2616-sec10.html\#sec10.4.5}{404 Not Found} -- Domain is not found

\end{itemize}

\end{description}\end{quote}

\end{fulllineitems}



\section{Invitation}
\label{\detokenize{resources/invitation:invitation}}\label{\detokenize{resources/invitation::doc}}\label{\detokenize{resources/invitation:id1}}
Represents an invitation.

\noindent\begin{tabulary}{\linewidth}{|L|L|L|}
\hline
\sphinxstylethead{\relax 
Resource
\unskip}\relax &\sphinxstylethead{\relax 
Operation
\unskip}\relax &\sphinxstylethead{\relax 
Description
\unskip}\relax \\
\hline
Invitation
&
{\hyperref[\detokenize{resources/invitation:get--invitations-(invitation_id)}]{\emph{GET /invitations/(invitation\_id)}}}
&
Retrieve an invitation.
\\
\hline&
{\hyperref[\detokenize{resources/invitation:get--invitations-(code)-accept}]{\emph{GET /invitations/(code)/accept}}}
&
Accept an invitation.
\\
\hline&
{\hyperref[\detokenize{resources/invitation:put--invitations-(invitation_id)-cancel}]{\emph{PUT /invitations/(invitation\_id)/cancel}}}
&
Cancel an invitation.
\\
\hline\end{tabulary}


\begin{DUlineblock}{0em}
\item[] 
\end{DUlineblock}


\begin{fulllineitems}
\phantomsection\label{\detokenize{resources/invitation:get--invitations-(invitation_id)}}\pysiglinewithargsret{\sphinxbfcode{GET~}\sphinxbfcode{/invitations/}}{\emph{invitation\_id}}{}~
Retrieve an invitation.

In order to retrieve an invitation, the authenticated user must be a
partner of the organization that the invitation is associated with.

\sphinxstylestrong{Example request}:

\begin{sphinxVerbatim}[commandchars=\\\{\}]
\PYG{n+nf}{GET} \PYG{n+nn}{/invitations/1} \PYG{k+kr}{HTTP}\PYG{o}{/}\PYG{l+m}{1.1}
\PYG{n+na}{Host}\PYG{o}{:} \PYG{l}{example.com}
\PYG{n+na}{Authorization}\PYG{o}{:} \PYG{l}{Bearer \PYGZlt{}token\PYGZgt{}}
\end{sphinxVerbatim}

\sphinxstylestrong{Example response}:

\begin{sphinxVerbatim}[commandchars=\\\{\}]
HTTP/1.1 200 OK
Content\PYGZhy{}Type: application/json

\PYGZob{}
    \PYGZsq{}id\PYGZsq{}: 1,
    \PYGZsq{}code\PYGZsq{}: \PYGZsq{}12345678\PYGZhy{}1234\PYGZhy{}1234\PYGZhy{}1234\PYGZhy{}123456789012\PYGZsq{},
    \PYGZsq{}email\PYGZsq{}: \PYGZsq{}john@example.org\PYGZsq{},
    \PYGZsq{}status\PYGZsq{}: \PYGZsq{}pending\PYGZsq{},
    \PYGZsq{}organization\PYGZus{}id\PYGZsq{}: 1
\PYGZcb{}
\end{sphinxVerbatim}
\begin{quote}\begin{description}
\item[{Parameters}] \leavevmode\begin{itemize}
\item {} 
\sphinxstyleliteralstrong{invitation\_id} (\sphinxstyleliteralemphasis{int}) -- the invitation to retrieve

\end{itemize}

\item[{Request Headers}] \leavevmode\begin{itemize}
\item {} 
\href{http://tools.ietf.org/html/rfc7235\#section-4.2}{Authorization} -- JSON Web Token to authenticate

\end{itemize}

\item[{Response Headers}] \leavevmode\begin{itemize}
\item {} 
\href{http://tools.ietf.org/html/rfc7231\#section-3.1.1.5}{Content-Type} -- data is received as application/json

\end{itemize}

\item[{Response JSON Object}] \leavevmode\begin{itemize}
\item {} 
\sphinxstyleliteralstrong{id} (\sphinxstyleliteralemphasis{int}) -- the invitation's unique id

\item {} 
\sphinxstyleliteralstrong{code} (\sphinxstyleliteralemphasis{string}) -- the invitation's unique code

\item {} 
\sphinxstyleliteralstrong{email} (\sphinxstyleliteralemphasis{string}) -- the email address the invitation is sent to

\item {} 
\sphinxstyleliteralstrong{status} (\sphinxstyleliteralemphasis{string}) -- the invitation's status

\item {} 
\sphinxstyleliteralstrong{organization\_id} (\sphinxstyleliteralemphasis{int}) -- the organization the invitation is related
to

\end{itemize}

\item[{Status Codes}] \leavevmode\begin{itemize}
\item {} 
\href{http://www.w3.org/Protocols/rfc2616/rfc2616-sec10.html\#sec10.2.1}{200 OK} -- Invitation is retrieved

\item {} 
\href{http://www.w3.org/Protocols/rfc2616/rfc2616-sec10.html\#sec10.4.1}{400 Bad Request} -- Token is not well-formed

\item {} 
\href{http://www.w3.org/Protocols/rfc2616/rfc2616-sec10.html\#sec10.4.2}{401 Unauthorized} -- Token has expired

\item {} 
\href{http://www.w3.org/Protocols/rfc2616/rfc2616-sec10.html\#sec10.4.2}{401 Unauthorized} -- User is not authorized

\item {} 
\href{http://www.w3.org/Protocols/rfc2616/rfc2616-sec10.html\#sec10.4.5}{404 Not Found} -- Invitation is not found

\end{itemize}

\end{description}\end{quote}

\end{fulllineitems}



\begin{fulllineitems}
\phantomsection\label{\detokenize{resources/invitation:get--invitations-(code)-accept}}\pysiglinewithargsret{\sphinxbfcode{GET~}\sphinxbfcode{/invitations/}}{\emph{code}}{\sphinxbfcode{/accept}}~
Accept an invitation.

If an invitation's state is `pending', this endpoint will set the
invitation's state to `accepted' and the authenticated user will be
added as a partner to the associated organization. If an invitation's
state is `accepted' or `cancelled', the invitation cannot be
accepted again or accepted at all. In order to accept an invitation,
the user must be an authenticated user.

\sphinxstylestrong{Example request}:

\begin{sphinxVerbatim}[commandchars=\\\{\}]
GET /invitations/12345678\PYGZhy{}1234\PYGZhy{}1234\PYGZhy{}1234\PYGZhy{}123456789012/accept
HTTP/1.1
Host: example.com
Authorization: Bearer \PYGZlt{}token\PYGZgt{}
\end{sphinxVerbatim}

\sphinxstylestrong{Example response}:

\begin{sphinxVerbatim}[commandchars=\\\{\}]
HTTP/1.1 200 OK
Content\PYGZhy{}Type: application/json

\PYGZob{}
    \PYGZsq{}id\PYGZsq{}: 1,
    \PYGZsq{}code\PYGZsq{}: \PYGZsq{}12345678\PYGZhy{}1234\PYGZhy{}1234\PYGZhy{}1234\PYGZhy{}123456789012\PYGZsq{},
    \PYGZsq{}email\PYGZsq{}: \PYGZsq{}john@example.org\PYGZsq{},
    \PYGZsq{}status\PYGZsq{}: \PYGZsq{}accepted\PYGZsq{},
    \PYGZsq{}organization\PYGZus{}id\PYGZsq{}: 1
\PYGZcb{}
\end{sphinxVerbatim}
\begin{quote}\begin{description}
\item[{Parameters}] \leavevmode\begin{itemize}
\item {} 
\sphinxstyleliteralstrong{invitation\_id} (\sphinxstyleliteralemphasis{int}) -- the invitation to accept

\end{itemize}

\item[{Request Headers}] \leavevmode\begin{itemize}
\item {} 
\href{http://tools.ietf.org/html/rfc7235\#section-4.2}{Authorization} -- JSON Web Token to authenticate

\end{itemize}

\item[{Response Headers}] \leavevmode\begin{itemize}
\item {} 
\href{http://tools.ietf.org/html/rfc7231\#section-3.1.1.5}{Content-Type} -- data is received as application/json

\end{itemize}

\item[{Response JSON Object}] \leavevmode\begin{itemize}
\item {} 
\sphinxstyleliteralstrong{id} (\sphinxstyleliteralemphasis{int}) -- the invitation's unique id

\item {} 
\sphinxstyleliteralstrong{code} (\sphinxstyleliteralemphasis{string}) -- the invitation's unique code

\item {} 
\sphinxstyleliteralstrong{email} (\sphinxstyleliteralemphasis{string}) -- the email address the invitation is sent to

\item {} 
\sphinxstyleliteralstrong{status} (\sphinxstyleliteralemphasis{string}) -- the invitation's status

\item {} 
\sphinxstyleliteralstrong{organization\_id} (\sphinxstyleliteralemphasis{int}) -- the organization the invitation is related
to

\end{itemize}

\item[{Status Codes}] \leavevmode\begin{itemize}
\item {} 
\href{http://www.w3.org/Protocols/rfc2616/rfc2616-sec10.html\#sec10.2.1}{200 OK} -- Invitation is accepted

\item {} 
\href{http://www.w3.org/Protocols/rfc2616/rfc2616-sec10.html\#sec10.4.1}{400 Bad Request} -- Token is not well-formed

\item {} 
\href{http://www.w3.org/Protocols/rfc2616/rfc2616-sec10.html\#sec10.4.2}{401 Unauthorized} -- Token has expired

\item {} 
\href{http://www.w3.org/Protocols/rfc2616/rfc2616-sec10.html\#sec10.4.2}{401 Unauthorized} -- User is not authorized

\item {} 
\href{http://www.w3.org/Protocols/rfc2616/rfc2616-sec10.html\#sec10.4.5}{404 Not Found} -- Invitation is not found

\item {} 
\href{http://www.w3.org/Protocols/rfc2616/rfc2616-sec10.html\#sec10.4.10}{409 Conflict} -- Invitation status is cancelled

\end{itemize}

\end{description}\end{quote}

\end{fulllineitems}



\begin{fulllineitems}
\phantomsection\label{\detokenize{resources/invitation:put--invitations-(invitation_id)-cancel}}\pysiglinewithargsret{\sphinxbfcode{PUT~}\sphinxbfcode{/invitations/}}{\emph{invitation\_id}}{\sphinxbfcode{/cancel}}~
Cancel an invitation.

If an invitation's state is `pending', this endpoint will set the
invitation's state to `cancelled'. If an invitation's state is
`accepted' or `cancelled', the invitation cannot be cancelled at all or
cancelled again. In order to cancel an invitation, the authenticated
user must be an admin of the organization that the invitation is
associated with.

\sphinxstylestrong{Example request}:

\begin{sphinxVerbatim}[commandchars=\\\{\}]
\PYG{n+nf}{GET} \PYG{n+nn}{/invitations/1/cancel} \PYG{k+kr}{HTTP}\PYG{o}{/}\PYG{l+m}{1.1}
\PYG{n+na}{Host}\PYG{o}{:} \PYG{l}{example.com}
\PYG{n+na}{Authorization}\PYG{o}{:} \PYG{l}{Bearer \PYGZlt{}token\PYGZgt{}}
\end{sphinxVerbatim}

\sphinxstylestrong{Example response}:

\begin{sphinxVerbatim}[commandchars=\\\{\}]
HTTP/1.1 200 OK
Content\PYGZhy{}Type: application/json

\PYGZob{}
    \PYGZsq{}id\PYGZsq{}: 1,
    \PYGZsq{}code\PYGZsq{}: \PYGZsq{}12345678\PYGZhy{}1234\PYGZhy{}1234\PYGZhy{}1234\PYGZhy{}123456789012\PYGZsq{},
    \PYGZsq{}email\PYGZsq{}: \PYGZsq{}john@example.org\PYGZsq{},
    \PYGZsq{}status\PYGZsq{}: \PYGZsq{}cancelled\PYGZsq{},
    \PYGZsq{}organization\PYGZus{}id\PYGZsq{}: 1
\PYGZcb{}
\end{sphinxVerbatim}
\begin{quote}\begin{description}
\item[{Parameters}] \leavevmode\begin{itemize}
\item {} 
\sphinxstyleliteralstrong{invitation\_id} (\sphinxstyleliteralemphasis{int}) -- the invitation to cancel

\end{itemize}

\item[{Request Headers}] \leavevmode\begin{itemize}
\item {} 
\href{http://tools.ietf.org/html/rfc7235\#section-4.2}{Authorization} -- JSON Web Token to authenticate

\end{itemize}

\item[{Response Headers}] \leavevmode\begin{itemize}
\item {} 
\href{http://tools.ietf.org/html/rfc7231\#section-3.1.1.5}{Content-Type} -- data is received as application/json

\end{itemize}

\item[{Response JSON Object}] \leavevmode\begin{itemize}
\item {} 
\sphinxstyleliteralstrong{id} (\sphinxstyleliteralemphasis{int}) -- the invitation's unique id

\item {} 
\sphinxstyleliteralstrong{code} (\sphinxstyleliteralemphasis{string}) -- the invitation's unique code

\item {} 
\sphinxstyleliteralstrong{email} (\sphinxstyleliteralemphasis{string}) -- the email address the invitation is sent to

\item {} 
\sphinxstyleliteralstrong{status} (\sphinxstyleliteralemphasis{string}) -- the invitation's status

\item {} 
\sphinxstyleliteralstrong{organization\_id} (\sphinxstyleliteralemphasis{int}) -- the organization the invitation is related
to

\end{itemize}

\item[{Status Codes}] \leavevmode\begin{itemize}
\item {} 
\href{http://www.w3.org/Protocols/rfc2616/rfc2616-sec10.html\#sec10.2.1}{200 OK} -- Invitation is cancelled

\item {} 
\href{http://www.w3.org/Protocols/rfc2616/rfc2616-sec10.html\#sec10.4.1}{400 Bad Request} -- Token is not well-formed

\item {} 
\href{http://www.w3.org/Protocols/rfc2616/rfc2616-sec10.html\#sec10.4.2}{401 Unauthorized} -- Token has expired

\item {} 
\href{http://www.w3.org/Protocols/rfc2616/rfc2616-sec10.html\#sec10.4.2}{401 Unauthorized} -- User is not authorized

\item {} 
\href{http://www.w3.org/Protocols/rfc2616/rfc2616-sec10.html\#sec10.4.5}{404 Not Found} -- Invitation is not found

\item {} 
\href{http://www.w3.org/Protocols/rfc2616/rfc2616-sec10.html\#sec10.4.10}{409 Conflict} -- Invitation status is accepted

\end{itemize}

\end{description}\end{quote}

\end{fulllineitems}



\section{Organization}
\label{\detokenize{resources/organization:organization}}\label{\detokenize{resources/organization::doc}}\label{\detokenize{resources/organization:id1}}
Represents an organization.

\noindent\begin{tabulary}{\linewidth}{|L|L|L|}
\hline
\sphinxstylethead{\relax 
Resource
\unskip}\relax &\sphinxstylethead{\relax 
Operation
\unskip}\relax &\sphinxstylethead{\relax 
Description
\unskip}\relax \\
\hline
Organization
&
{\hyperref[\detokenize{resources/organization:put--organizations-(organization_id)}]{\emph{PUT /organizations/(organization\_id)}}}
&
Update an organization.
\\
\hline&
{\hyperref[\detokenize{resources/organization:delete--organizations-(organization_id)}]{\emph{DELETE /organizations/(organization\_id)}}}
&
Delete an organization.
\\
\hline&
{\hyperref[\detokenize{resources/organization:get--organizations-(organization_id)}]{\emph{GET /organizations/(organization\_id)}}}
&
Retrieve an Organization.
\\
\hline\end{tabulary}


\begin{DUlineblock}{0em}
\item[] 
\end{DUlineblock}


\begin{fulllineitems}
\phantomsection\label{\detokenize{resources/organization:put--organizations-(organization_id)}}\pysiglinewithargsret{\sphinxbfcode{PUT~}\sphinxbfcode{/organizations/}}{\emph{organization\_id}}{}~
Update an organization.

In order to update an organization, the authenticated user must be an
admin of the organization.

\sphinxstylestrong{Example request}:

\begin{sphinxVerbatim}[commandchars=\\\{\}]
PUT /organizations/1 HTTP/1.1
Host: example.com
Authorization: Bearer \PYGZlt{}token\PYGZgt{}
Content\PYGZhy{}Type: application/json

\PYGZob{}
    \PYGZsq{}name\PYGZsq{}: \PYGZsq{}My Organization\PYGZsq{}
\PYGZcb{}
\end{sphinxVerbatim}

\sphinxstylestrong{Example response}:

\begin{sphinxVerbatim}[commandchars=\\\{\}]
HTTP/1.1 200 OK
Content\PYGZhy{}Type: application/json

\PYGZob{}
    \PYGZsq{}id\PYGZsq{}: 1,
    \PYGZsq{}name\PYGZsq{}: \PYGZsq{}My Organization\PYGZsq{}
\PYGZcb{}
\end{sphinxVerbatim}
\begin{quote}\begin{description}
\item[{Parameters}] \leavevmode\begin{itemize}
\item {} 
\sphinxstyleliteralstrong{organization\_id} (\sphinxstyleliteralemphasis{int}) -- the organization to update

\end{itemize}

\item[{Request Headers}] \leavevmode\begin{itemize}
\item {} 
\href{http://tools.ietf.org/html/rfc7235\#section-4.2}{Authorization} -- JSON Web Token to authenticate

\item {} 
\href{http://tools.ietf.org/html/rfc7231\#section-3.1.1.5}{Content-Type} -- data is sent as application/json or
application/x-www-form-urlencoded

\end{itemize}

\item[{Request JSON Object}] \leavevmode\begin{itemize}
\item {} 
\sphinxstyleliteralstrong{name} (\sphinxstyleliteralemphasis{string}) -- the organization's name

\end{itemize}

\item[{Response Headers}] \leavevmode\begin{itemize}
\item {} 
\href{http://tools.ietf.org/html/rfc7231\#section-3.1.1.5}{Content-Type} -- data is received as application/json

\end{itemize}

\item[{Response JSON Object}] \leavevmode\begin{itemize}
\item {} 
\sphinxstyleliteralstrong{id} (\sphinxstyleliteralemphasis{int}) -- the organization's unique id

\item {} 
\sphinxstyleliteralstrong{name} (\sphinxstyleliteralemphasis{string}) -- the organization's name

\end{itemize}

\item[{Status Codes}] \leavevmode\begin{itemize}
\item {} 
\href{http://www.w3.org/Protocols/rfc2616/rfc2616-sec10.html\#sec10.2.1}{200 OK} -- Organization is updated

\item {} 
\href{http://www.w3.org/Protocols/rfc2616/rfc2616-sec10.html\#sec10.4.1}{400 Bad Request} -- Parameters are missing

\item {} 
\href{http://www.w3.org/Protocols/rfc2616/rfc2616-sec10.html\#sec10.4.1}{400 Bad Request} -- Token is not well-formed

\item {} 
\href{http://www.w3.org/Protocols/rfc2616/rfc2616-sec10.html\#sec10.4.2}{401 Unauthorized} -- Token has expired

\item {} 
\href{http://www.w3.org/Protocols/rfc2616/rfc2616-sec10.html\#sec10.4.2}{401 Unauthorized} -- User is not authorized

\item {} 
\href{http://www.w3.org/Protocols/rfc2616/rfc2616-sec10.html\#sec10.4.5}{404 Not Found} -- Organization is not found

\end{itemize}

\end{description}\end{quote}

\end{fulllineitems}



\begin{fulllineitems}
\phantomsection\label{\detokenize{resources/organization:delete--organizations-(organization_id)}}\pysiglinewithargsret{\sphinxbfcode{DELETE~}\sphinxbfcode{/organizations/}}{\emph{organization\_id}}{}~
Delete an organization.

In order to delete an organization, the authenticated user must be an
admin of the organization.

\sphinxstylestrong{Example request}:

\begin{sphinxVerbatim}[commandchars=\\\{\}]
\PYG{n+nf}{DELETE} \PYG{n+nn}{/organizations/1} \PYG{k+kr}{HTTP}\PYG{o}{/}\PYG{l+m}{1.1}
\PYG{n+na}{Host}\PYG{o}{:} \PYG{l}{example.com}
\PYG{n+na}{Authorization}\PYG{o}{:} \PYG{l}{Bearer \PYGZlt{}token\PYGZgt{}}
\end{sphinxVerbatim}

\sphinxstylestrong{Example response}:

\begin{sphinxVerbatim}[commandchars=\\\{\}]
\PYG{k+kr}{HTTP}\PYG{o}{/}\PYG{l+m}{1.1} \PYG{l+m}{204} \PYG{n+ne}{No Content}
\end{sphinxVerbatim}
\begin{quote}\begin{description}
\item[{Parameters}] \leavevmode\begin{itemize}
\item {} 
\sphinxstyleliteralstrong{organization\_id} (\sphinxstyleliteralemphasis{int}) -- the organization to delete

\end{itemize}

\item[{Request Headers}] \leavevmode\begin{itemize}
\item {} 
\href{http://tools.ietf.org/html/rfc7235\#section-4.2}{Authorization} -- JSON Web Token to authenticate

\end{itemize}

\item[{Status Codes}] \leavevmode\begin{itemize}
\item {} 
\href{http://www.w3.org/Protocols/rfc2616/rfc2616-sec10.html\#sec10.2.5}{204 No Content} -- Organization is deleted

\item {} 
\href{http://www.w3.org/Protocols/rfc2616/rfc2616-sec10.html\#sec10.4.1}{400 Bad Request} -- Token is not well-formed

\item {} 
\href{http://www.w3.org/Protocols/rfc2616/rfc2616-sec10.html\#sec10.4.2}{401 Unauthorized} -- Token has expired

\item {} 
\href{http://www.w3.org/Protocols/rfc2616/rfc2616-sec10.html\#sec10.4.2}{401 Unauthorized} -- User is not authorized

\item {} 
\href{http://www.w3.org/Protocols/rfc2616/rfc2616-sec10.html\#sec10.4.5}{404 Not Found} -- Organization is not found

\end{itemize}

\end{description}\end{quote}

\end{fulllineitems}



\begin{fulllineitems}
\phantomsection\label{\detokenize{resources/organization:get--organizations-(organization_id)}}\pysiglinewithargsret{\sphinxbfcode{GET~}\sphinxbfcode{/organizations/}}{\emph{organization\_id}}{}~
Retrieve an organization.

In order to retrieve an organization, the authenticated user must be a
member or an admin of the organization.

\sphinxstylestrong{Example request}:

\begin{sphinxVerbatim}[commandchars=\\\{\}]
\PYG{n+nf}{GET} \PYG{n+nn}{/organizations/1} \PYG{k+kr}{HTTP}\PYG{o}{/}\PYG{l+m}{1.1}
\PYG{n+na}{Host}\PYG{o}{:} \PYG{l}{example.com}
\PYG{n+na}{Authorization}\PYG{o}{:} \PYG{l}{Bearer \PYGZlt{}token\PYGZgt{}}
\end{sphinxVerbatim}

\sphinxstylestrong{Example response}:

\begin{sphinxVerbatim}[commandchars=\\\{\}]
HTTP/1.1 200 OK
Content\PYGZhy{}Type: application/json

\PYGZob{}
    \PYGZsq{}id\PYGZsq{}: 1,
    \PYGZsq{}name\PYGZsq{}: \PYGZsq{}My Organization\PYGZsq{}
\PYGZcb{}
\end{sphinxVerbatim}
\begin{quote}\begin{description}
\item[{Parameters}] \leavevmode\begin{itemize}
\item {} 
\sphinxstyleliteralstrong{organization\_id} (\sphinxstyleliteralemphasis{int}) -- the organization to retrieve

\end{itemize}

\item[{Request Headers}] \leavevmode\begin{itemize}
\item {} 
\href{http://tools.ietf.org/html/rfc7235\#section-4.2}{Authorization} -- JSON Web Token to authenticate

\end{itemize}

\item[{Response Headers}] \leavevmode\begin{itemize}
\item {} 
\href{http://tools.ietf.org/html/rfc7231\#section-3.1.1.5}{Content-Type} -- data is received as application/json

\end{itemize}

\item[{Response JSON Object}] \leavevmode\begin{itemize}
\item {} 
\sphinxstyleliteralstrong{id} (\sphinxstyleliteralemphasis{int}) -- the organization's unique id

\item {} 
\sphinxstyleliteralstrong{name} (\sphinxstyleliteralemphasis{string}) -- the organization's name

\end{itemize}

\item[{Status Codes}] \leavevmode\begin{itemize}
\item {} 
\href{http://www.w3.org/Protocols/rfc2616/rfc2616-sec10.html\#sec10.2.1}{200 OK} -- Organization is retrieved

\item {} 
\href{http://www.w3.org/Protocols/rfc2616/rfc2616-sec10.html\#sec10.4.1}{400 Bad Request} -- Token is not well-formed

\item {} 
\href{http://www.w3.org/Protocols/rfc2616/rfc2616-sec10.html\#sec10.4.2}{401 Unauthorized} -- Token has expired

\item {} 
\href{http://www.w3.org/Protocols/rfc2616/rfc2616-sec10.html\#sec10.4.2}{401 Unauthorized} -- User is not authorized

\item {} 
\href{http://www.w3.org/Protocols/rfc2616/rfc2616-sec10.html\#sec10.4.5}{404 Not Found} -- Organization is not found

\end{itemize}

\end{description}\end{quote}

\end{fulllineitems}



\subsection{Anchor Circle}
\label{\detokenize{resources/organization:anchor-circle}}
Represents the anchor circle of an organization. See {\hyperref[\detokenize{resources/circle:circle}]{\sphinxcrossref{\DUrole{std,std-ref}{Circle}}}} for a description of a circle.

\noindent\begin{tabulary}{\linewidth}{|L|L|L|}
\hline
\sphinxstylethead{\relax 
Resource
\unskip}\relax &\sphinxstylethead{\relax 
Operation
\unskip}\relax &\sphinxstylethead{\relax 
Description
\unskip}\relax \\
\hline
Organization Anchor Circle
&
{\hyperref[\detokenize{resources/organization:get--organizations-(organization_id)-anchor_circle}]{\emph{GET /organizations/(organization\_id)/anchor\_circle}}}
&
Retrieve the anchor circle.
\\
\hline\end{tabulary}


\begin{DUlineblock}{0em}
\item[] 
\end{DUlineblock}


\begin{fulllineitems}
\phantomsection\label{\detokenize{resources/organization:get--organizations-(organization_id)-anchor_circle}}\pysiglinewithargsret{\sphinxbfcode{GET~}\sphinxbfcode{/organizations/}}{\emph{organization\_id}}{\sphinxbfcode{/anchor\_circle}}~
Retrieve the anchor circle of an organization.

This endpoint retrieves the anchor circle of an organization. Each
organization has exactly one circle as its anchor circle. In order to
retrieve the anchor circle of an organization, the authenticated user
must be a member or an admin of the organization.

\sphinxstylestrong{Example request}:

\begin{sphinxVerbatim}[commandchars=\\\{\}]
\PYG{n+nf}{GET} \PYG{n+nn}{/organizations/1/anchor\PYGZus{}circle} \PYG{k+kr}{HTTP}\PYG{o}{/}\PYG{l+m}{1.1}
\PYG{n+na}{Host}\PYG{o}{:} \PYG{l}{example.com}
\PYG{n+na}{Authorization}\PYG{o}{:} \PYG{l}{Bearer \PYGZlt{}token\PYGZgt{}}
\end{sphinxVerbatim}

\sphinxstylestrong{Example response}:

\begin{sphinxVerbatim}[commandchars=\\\{\}]
HTTP/1.1 200 OK
Content\PYGZhy{}Type: application/json

\PYGZob{}
    \PYGZsq{}id\PYGZsq{}: 1,
    \PYGZsq{}type\PYGZsq{}: \PYGZsq{}circle\PYGZsq{},
    \PYGZsq{}name\PYGZsq{}: \PYGZsq{}My Organization\PYGZsq{},
    \PYGZsq{}pupose\PYGZsq{}: \PYGZsq{}My Organization\PYGZsq{}s purpose\PYGZsq{},
    \PYGZsq{}strategy\PYGZsq{}: \PYGZsq{}My Organizations\PYGZsq{}s strategy\PYGZsq{},
    \PYGZsq{}parent\PYGZus{}circle\PYGZus{}id\PYGZsq{}: null,
    \PYGZsq{}organization\PYGZus{}id\PYGZsq{}: 1
\PYGZcb{}
\end{sphinxVerbatim}
\begin{quote}\begin{description}
\item[{Parameters}] \leavevmode\begin{itemize}
\item {} 
\sphinxstyleliteralstrong{organization\_id} (\sphinxstyleliteralemphasis{int}) -- the organization to retrieve the anchor
circle of

\end{itemize}

\item[{Request Headers}] \leavevmode\begin{itemize}
\item {} 
\href{http://tools.ietf.org/html/rfc7235\#section-4.2}{Authorization} -- JSON Web Token to authenticate

\end{itemize}

\item[{Response Headers}] \leavevmode\begin{itemize}
\item {} 
\href{http://tools.ietf.org/html/rfc7231\#section-3.1.1.5}{Content-Type} -- data is received as application/json

\end{itemize}

\item[{Response JSON Object}] \leavevmode\begin{itemize}
\item {} 
\sphinxstyleliteralstrong{id} (\sphinxstyleliteralemphasis{int}) -- the anchor circle's unique id

\item {} 
\sphinxstyleliteralstrong{type} (\sphinxstyleliteralemphasis{string}) -- the anchor circle's type

\item {} 
\sphinxstyleliteralstrong{name} (\sphinxstyleliteralemphasis{string}) -- the anchor circle's name

\item {} 
\sphinxstyleliteralstrong{purpose} (\sphinxstyleliteralemphasis{string}) -- the anchor circle's purpose

\item {} 
\sphinxstyleliteralstrong{strategy} (\sphinxstyleliteralemphasis{string}) -- the anchor circle's strategy

\item {} 
\sphinxstyleliteralstrong{parent\_role\_id} (\sphinxstyleliteralemphasis{int}) -- the role the anchor circle is a child of

\item {} 
\sphinxstyleliteralstrong{organization\_id} (\sphinxstyleliteralemphasis{int}) -- the organization the anchor circle is
related to

\end{itemize}

\item[{Status Codes}] \leavevmode\begin{itemize}
\item {} 
\href{http://www.w3.org/Protocols/rfc2616/rfc2616-sec10.html\#sec10.2.1}{200 OK} -- Anchor circle is retrieved

\item {} 
\href{http://www.w3.org/Protocols/rfc2616/rfc2616-sec10.html\#sec10.4.1}{400 Bad Request} -- Token is not well-formed

\item {} 
\href{http://www.w3.org/Protocols/rfc2616/rfc2616-sec10.html\#sec10.4.2}{401 Unauthorized} -- Token has expired

\item {} 
\href{http://www.w3.org/Protocols/rfc2616/rfc2616-sec10.html\#sec10.4.2}{401 Unauthorized} -- User is not authorized

\item {} 
\href{http://www.w3.org/Protocols/rfc2616/rfc2616-sec10.html\#sec10.4.5}{404 Not Found} -- Organization is not found

\end{itemize}

\end{description}\end{quote}

\end{fulllineitems}



\subsection{Invitations}
\label{\detokenize{resources/organization:invitations}}
Represents the invitations to an organization. See {\hyperref[\detokenize{resources/invitation:invitation}]{\sphinxcrossref{\DUrole{std,std-ref}{Invitation}}}} for a description of a single invitation.

\noindent\begin{tabulary}{\linewidth}{|L|L|L|}
\hline
\sphinxstylethead{\relax 
Resource
\unskip}\relax &\sphinxstylethead{\relax 
Operation
\unskip}\relax &\sphinxstylethead{\relax 
Description
\unskip}\relax \\
\hline
Organization Invitations
&
{\hyperref[\detokenize{resources/organization:post--organizations-(organization_id)-invitations}]{\emph{POST /organizations/(organization\_id)/invitations}}}
&
Invite a user to an
\\
\hline&
{\hyperref[\detokenize{resources/organization:get--organizations-(organization_id)-invitations}]{\emph{GET /organizations/(organization\_id)/invitations}}}
&
List invitations to an
\\
\hline\end{tabulary}


\begin{DUlineblock}{0em}
\item[] 
\end{DUlineblock}


\begin{fulllineitems}
\phantomsection\label{\detokenize{resources/organization:post--organizations-(organization_id)-invitations}}\pysiglinewithargsret{\sphinxbfcode{POST~}\sphinxbfcode{/organizations/}}{\emph{organization\_id}}{\sphinxbfcode{/invitations}}~
Invite a user to an organization.

This endpoint will send an invitation to a given email address. The
newly-created invitation will be in the `pending' state until the user
accepts the invitation. At this point the invitation will transition
to the `accepted' state and the user will be added as a new partner to
the organization. In order to invite a user to an organization, the
authenticated user must be an admin of the organization.

\sphinxstylestrong{Example request}:

\begin{sphinxVerbatim}[commandchars=\\\{\}]
POST /organizations/1/invitations HTTP/1.1
Host: example.com
Authorization: Bearer \PYGZlt{}token\PYGZgt{}
Content\PYGZhy{}Type: application/json

\PYGZob{}
    \PYGZsq{}email\PYGZsq{}: \PYGZsq{}john@example.org\PYGZsq{}
\PYGZcb{}
\end{sphinxVerbatim}

\sphinxstylestrong{Example response}:

\begin{sphinxVerbatim}[commandchars=\\\{\}]
HTTP/1.1 201 Created
Content\PYGZhy{}Type: application/json

\PYGZob{}
    \PYGZsq{}id\PYGZsq{}: 1,
    \PYGZsq{}code\PYGZsq{}: \PYGZsq{}12345678\PYGZhy{}1234\PYGZhy{}1234\PYGZhy{}1234\PYGZhy{}123456789012\PYGZsq{},
    \PYGZsq{}email\PYGZsq{}: \PYGZsq{}john@example.org\PYGZsq{},
    \PYGZsq{}status\PYGZsq{}: \PYGZsq{}pending\PYGZsq{},
    \PYGZsq{}organization\PYGZus{}id\PYGZsq{}: 1
\PYGZcb{}
\end{sphinxVerbatim}
\begin{quote}\begin{description}
\item[{Parameters}] \leavevmode\begin{itemize}
\item {} 
\sphinxstyleliteralstrong{organization\_id} (\sphinxstyleliteralemphasis{int}) -- the organization the invitation is created
for

\end{itemize}

\item[{Request Headers}] \leavevmode\begin{itemize}
\item {} 
\href{http://tools.ietf.org/html/rfc7235\#section-4.2}{Authorization} -- JSON Web Token to authenticate

\item {} 
\href{http://tools.ietf.org/html/rfc7231\#section-3.1.1.5}{Content-Type} -- data is sent as application/json or
application/x-www-form-urlencoded

\end{itemize}

\item[{Request JSON Object}] \leavevmode\begin{itemize}
\item {} 
\sphinxstyleliteralstrong{email} (\sphinxstyleliteralemphasis{string}) -- the email address the invitation is sent to

\end{itemize}

\item[{Response Headers}] \leavevmode\begin{itemize}
\item {} 
\href{http://tools.ietf.org/html/rfc7231\#section-3.1.1.5}{Content-Type} -- data is received as application/json

\end{itemize}

\item[{Response JSON Object}] \leavevmode\begin{itemize}
\item {} 
\sphinxstyleliteralstrong{id} (\sphinxstyleliteralemphasis{int}) -- the invitation's unique id

\item {} 
\sphinxstyleliteralstrong{code} (\sphinxstyleliteralemphasis{string}) -- the invitation's unique code

\item {} 
\sphinxstyleliteralstrong{email} (\sphinxstyleliteralemphasis{string}) -- the email address the invitation is sent to

\item {} 
\sphinxstyleliteralstrong{status} (\sphinxstyleliteralemphasis{string}) -- the invitation's status

\item {} 
\sphinxstyleliteralstrong{organization\_id} (\sphinxstyleliteralemphasis{int}) -- the organization the invitation is related
to

\end{itemize}

\item[{Status Codes}] \leavevmode\begin{itemize}
\item {} 
\href{http://www.w3.org/Protocols/rfc2616/rfc2616-sec10.html\#sec10.2.2}{201 Created} -- Invitation is created

\item {} 
\href{http://www.w3.org/Protocols/rfc2616/rfc2616-sec10.html\#sec10.4.1}{400 Bad Request} -- Parameters are missing

\item {} 
\href{http://www.w3.org/Protocols/rfc2616/rfc2616-sec10.html\#sec10.4.1}{400 Bad Request} -- Token is not well-formed

\item {} 
\href{http://www.w3.org/Protocols/rfc2616/rfc2616-sec10.html\#sec10.4.2}{401 Unauthorized} -- Token has expired

\item {} 
\href{http://www.w3.org/Protocols/rfc2616/rfc2616-sec10.html\#sec10.4.2}{401 Unauthorized} -- User is not authorized

\item {} 
\href{http://www.w3.org/Protocols/rfc2616/rfc2616-sec10.html\#sec10.4.5}{404 Not Found} -- Organization is not found

\end{itemize}

\end{description}\end{quote}

\end{fulllineitems}



\begin{fulllineitems}
\phantomsection\label{\detokenize{resources/organization:get--organizations-(organization_id)-invitations}}\pysiglinewithargsret{\sphinxbfcode{GET~}\sphinxbfcode{/organizations/}}{\emph{organization\_id}}{\sphinxbfcode{/invitations}}~
List invitations to an organization.

This endpoint lists all `pending', `accepted' and `cancelled'
invitations to an organization. In order to list invitations to an
organization, the authenticated user must be a member or an admin of
the organization.

\sphinxstylestrong{Example request}:

\begin{sphinxVerbatim}[commandchars=\\\{\}]
\PYG{n+nf}{GET} \PYG{n+nn}{/organizations/1/invitations} \PYG{k+kr}{HTTP}\PYG{o}{/}\PYG{l+m}{1.1}
\PYG{n+na}{Host}\PYG{o}{:} \PYG{l}{example.com}
\PYG{n+na}{Authorization}\PYG{o}{:} \PYG{l}{Bearer \PYGZlt{}token\PYGZgt{}}
\end{sphinxVerbatim}

\sphinxstylestrong{Example response}:

\begin{sphinxVerbatim}[commandchars=\\\{\}]
HTTP/1.1 200 OK
Content\PYGZhy{}Type: application/json

[
    \PYGZob{}
        \PYGZsq{}id\PYGZsq{}: 1,
        \PYGZsq{}code\PYGZsq{}: \PYGZsq{}12345678\PYGZhy{}1234\PYGZhy{}1234\PYGZhy{}1234\PYGZhy{}123456789012\PYGZsq{},
        \PYGZsq{}email\PYGZsq{}: \PYGZsq{}john@example.org\PYGZsq{},
        \PYGZsq{}status\PYGZsq{}: \PYGZsq{}pending\PYGZsq{},
        \PYGZsq{}organization\PYGZus{}id\PYGZsq{}: 1
    \PYGZcb{}
]
\end{sphinxVerbatim}
\begin{quote}\begin{description}
\item[{Parameters}] \leavevmode\begin{itemize}
\item {} 
\sphinxstyleliteralstrong{organization\_id} (\sphinxstyleliteralemphasis{int}) -- the organization the invitations are
listed for

\end{itemize}

\item[{Request Headers}] \leavevmode\begin{itemize}
\item {} 
\href{http://tools.ietf.org/html/rfc7235\#section-4.2}{Authorization} -- JSON Web Token to authenticate

\end{itemize}

\item[{Response Headers}] \leavevmode\begin{itemize}
\item {} 
\href{http://tools.ietf.org/html/rfc7231\#section-3.1.1.5}{Content-Type} -- data is received as application/json

\end{itemize}

\item[{Response JSON Array of Objects}] \leavevmode\begin{itemize}
\item {} 
\sphinxstyleliteralstrong{id} (\sphinxstyleliteralemphasis{int}) -- the invitation's unique id

\item {} 
\sphinxstyleliteralstrong{code} (\sphinxstyleliteralemphasis{string}) -- the invitation's unique code

\item {} 
\sphinxstyleliteralstrong{email} (\sphinxstyleliteralemphasis{string}) -- the email address the invitation is sent to

\item {} 
\sphinxstyleliteralstrong{status} (\sphinxstyleliteralemphasis{string}) -- the invitation's status

\item {} 
\sphinxstyleliteralstrong{organization\_id} (\sphinxstyleliteralemphasis{int}) -- the organization the invitation is
related to

\end{itemize}

\item[{Status Codes}] \leavevmode\begin{itemize}
\item {} 
\href{http://www.w3.org/Protocols/rfc2616/rfc2616-sec10.html\#sec10.2.1}{200 OK} -- Invitations are listed

\item {} 
\href{http://www.w3.org/Protocols/rfc2616/rfc2616-sec10.html\#sec10.4.1}{400 Bad Request} -- Token is not well-formed

\item {} 
\href{http://www.w3.org/Protocols/rfc2616/rfc2616-sec10.html\#sec10.4.2}{401 Unauthorized} -- Token has expired

\item {} 
\href{http://www.w3.org/Protocols/rfc2616/rfc2616-sec10.html\#sec10.4.2}{401 Unauthorized} -- User is not authorized

\item {} 
\href{http://www.w3.org/Protocols/rfc2616/rfc2616-sec10.html\#sec10.4.5}{404 Not Found} -- Organization is not found

\end{itemize}

\end{description}\end{quote}

\end{fulllineitems}



\subsection{Members}
\label{\detokenize{resources/organization:members}}
Represents the members of an organization. See {\hyperref[\detokenize{resources/partner:partner}]{\sphinxcrossref{\DUrole{std,std-ref}{Partner}}}} for a description of a single member.

\noindent\begin{tabulary}{\linewidth}{|L|L|L|}
\hline
\sphinxstylethead{\relax 
Resource
\unskip}\relax &\sphinxstylethead{\relax 
Operation
\unskip}\relax &\sphinxstylethead{\relax 
Description
\unskip}\relax \\
\hline
Organization Members
&
{\hyperref[\detokenize{resources/organization:get--organizations-(organization_id)-members}]{\emph{GET /organizations/(organization\_id)/members}}}
&
List members of an organization.
\\
\hline\end{tabulary}


\begin{DUlineblock}{0em}
\item[] 
\end{DUlineblock}


\begin{fulllineitems}
\phantomsection\label{\detokenize{resources/organization:get--organizations-(organization_id)-members}}\pysiglinewithargsret{\sphinxbfcode{GET~}\sphinxbfcode{/organizations/}}{\emph{organization\_id}}{\sphinxbfcode{/members}}~
List partners of an organization.

This endpoint lists all members of an organization, whether their
status is `active' or not. In order to list the members of an
organization, the authenticated user must be a members of the
organization.

\sphinxstylestrong{Example request}:

\begin{sphinxVerbatim}[commandchars=\\\{\}]
\PYG{n+nf}{GET} \PYG{n+nn}{/organizations/1/members} \PYG{k+kr}{HTTP}\PYG{o}{/}\PYG{l+m}{1.1}
\PYG{n+na}{Host}\PYG{o}{:} \PYG{l}{example.com}
\PYG{n+na}{Authorization}\PYG{o}{:} \PYG{l}{Bearer \PYGZlt{}token\PYGZgt{}}
\end{sphinxVerbatim}

\sphinxstylestrong{Example response}:

\begin{sphinxVerbatim}[commandchars=\\\{\}]
HTTP/1.1 200 OK
Content\PYGZhy{}Type: application/json

[
    \PYGZob{}
        \PYGZsq{}id\PYGZsq{}: 1,
        \PYGZsq{}type\PYGZsq{}: \PYGZsq{}member\PYGZsq{},
        \PYGZsq{}firstname\PYGZsq{}: \PYGZsq{}John\PYGZsq{},
        \PYGZsq{}lastname\PYGZsq{}: \PYGZsq{}Doe\PYGZsq{},
        \PYGZsq{}email\PYGZsq{}: \PYGZsq{}john@example.org\PYGZsq{},
        \PYGZsq{}is\PYGZus{}active\PYGZsq{}: True,
        \PYGZsq{}user\PYGZus{}id\PYGZsq{}: 1,
        \PYGZsq{}organization\PYGZus{}id\PYGZsq{}: 1,
        \PYGZsq{}invitation\PYGZus{}id\PYGZsq{}: null
    \PYGZcb{}
]
\end{sphinxVerbatim}
\begin{quote}\begin{description}
\item[{Parameters}] \leavevmode\begin{itemize}
\item {} 
\sphinxstyleliteralstrong{organization\_id} (\sphinxstyleliteralemphasis{int}) -- the organization the members are listed
for

\end{itemize}

\item[{Request Headers}] \leavevmode\begin{itemize}
\item {} 
\href{http://tools.ietf.org/html/rfc7235\#section-4.2}{Authorization} -- JSON Web Token to authenticate

\end{itemize}

\item[{Response Headers}] \leavevmode\begin{itemize}
\item {} 
\href{http://tools.ietf.org/html/rfc7231\#section-3.1.1.5}{Content-Type} -- data is received as application/json

\end{itemize}

\item[{Response JSON Array of Objects}] \leavevmode\begin{itemize}
\item {} 
\sphinxstyleliteralstrong{id} (\sphinxstyleliteralemphasis{int}) -- the member's unique id

\item {} 
\sphinxstyleliteralstrong{type} (\sphinxstyleliteralemphasis{string}) -- the member's type

\item {} 
\sphinxstyleliteralstrong{firstname} (\sphinxstyleliteralemphasis{string}) -- the member's firstname

\item {} 
\sphinxstyleliteralstrong{lastname} (\sphinxstyleliteralemphasis{string}) -- the member's lastname

\item {} 
\sphinxstyleliteralstrong{email} (\sphinxstyleliteralemphasis{string}) -- the member's email address

\item {} 
\sphinxstyleliteralstrong{is\_active} (\sphinxstyleliteralemphasis{boolean}) -- the member's status

\item {} 
\sphinxstyleliteralstrong{user\_id} (\sphinxstyleliteralemphasis{int}) -- the user account the member is related to

\item {} 
\sphinxstyleliteralstrong{organization\_id} (\sphinxstyleliteralemphasis{int}) -- the organization the member is
related to

\item {} 
\sphinxstyleliteralstrong{invitation\_id} (\sphinxstyleliteralemphasis{int}) -- the invitation the member is related to

\end{itemize}

\item[{Status Codes}] \leavevmode\begin{itemize}
\item {} 
\href{http://www.w3.org/Protocols/rfc2616/rfc2616-sec10.html\#sec10.2.1}{200 OK} -- Members are listed

\item {} 
\href{http://www.w3.org/Protocols/rfc2616/rfc2616-sec10.html\#sec10.4.1}{400 Bad Request} -- Token is not well-formed

\item {} 
\href{http://www.w3.org/Protocols/rfc2616/rfc2616-sec10.html\#sec10.4.2}{401 Unauthorized} -- Token has expired

\item {} 
\href{http://www.w3.org/Protocols/rfc2616/rfc2616-sec10.html\#sec10.4.2}{401 Unauthorized} -- User is not authorized

\item {} 
\href{http://www.w3.org/Protocols/rfc2616/rfc2616-sec10.html\#sec10.4.5}{404 Not Found} -- Organization is not found

\end{itemize}

\end{description}\end{quote}

\end{fulllineitems}



\section{Partner}
\label{\detokenize{resources/partner::doc}}\label{\detokenize{resources/partner:id1}}\label{\detokenize{resources/partner:partner}}
Represents the relationship of a {\hyperref[\detokenize{resources/user:user}]{\sphinxcrossref{\DUrole{std,std-ref}{User}}}} with an {\hyperref[\detokenize{resources/organization:organization}]{\sphinxcrossref{\DUrole{std,std-ref}{Organization}}}}.

\noindent\begin{tabulary}{\linewidth}{|L|L|L|}
\hline
\sphinxstylethead{\relax 
Resource
\unskip}\relax &\sphinxstylethead{\relax 
Operation
\unskip}\relax &\sphinxstylethead{\relax 
Description
\unskip}\relax \\
\hline
Partner
&
{\hyperref[\detokenize{resources/partner:put--partners-(partner_id)}]{\emph{PUT /partners/(partner\_id)}}}
&
Update a partner.
\\
\hline&
{\hyperref[\detokenize{resources/partner:delete--partners-(partner_id)}]{\emph{DELETE /partners/(partner\_id)}}}
&
Delete a partner.
\\
\hline&
{\hyperref[\detokenize{resources/partner:get--partners-(partner_id)}]{\emph{GET /partners/(partner\_id)}}}
&
Retrieve a partner.
\\
\hline\end{tabulary}


\begin{DUlineblock}{0em}
\item[] 
\end{DUlineblock}


\begin{fulllineitems}
\phantomsection\label{\detokenize{resources/partner:put--partners-(partner_id)}}\pysiglinewithargsret{\sphinxbfcode{PUT~}\sphinxbfcode{/partners/}}{\emph{partner\_id}}{}~
Update a partner.

In order to update a partner, the authenticated user must be a partner
with admin access of the organization that the partner is associated
with.

\sphinxstylestrong{Example request}:

\begin{sphinxVerbatim}[commandchars=\\\{\}]
PUT /partners/1 HTTP/1.1
Host: example.com
Authorization: Bearer \PYGZlt{}token\PYGZgt{}
Content\PYGZhy{}Type: application/json

\PYGZob{}
    \PYGZsq{}firstname\PYGZsq{}: \PYGZsq{}John\PYGZsq{},
    \PYGZsq{}lastname\PYGZsq{}: \PYGZsq{}Doe\PYGZsq{},
    \PYGZsq{}email\PYGZsq{}: \PYGZsq{}john@example.org\PYGZsq{}
\PYGZcb{}
\end{sphinxVerbatim}

\sphinxstylestrong{Example response}:

\begin{sphinxVerbatim}[commandchars=\\\{\}]
HTTP/1.1 200 OK
Content\PYGZhy{}Type: application/json

\PYGZob{}
    \PYGZsq{}id\PYGZsq{}: 1,
    \PYGZsq{}type\PYGZsq{}: \PYGZsq{}member\PYGZsq{},
    \PYGZsq{}firstname\PYGZsq{}: \PYGZsq{}John\PYGZsq{},
    \PYGZsq{}lastname\PYGZsq{}: \PYGZsq{}Doe\PYGZsq{},
    \PYGZsq{}email\PYGZsq{}: \PYGZsq{}john@example.org\PYGZsq{},
    \PYGZsq{}is\PYGZus{}active\PYGZsq{}: True,
    \PYGZsq{}user\PYGZus{}id\PYGZsq{}: 1,
    \PYGZsq{}organization\PYGZus{}id\PYGZsq{}: 1,
    \PYGZsq{}invitation\PYGZus{}id\PYGZsq{}: null
\PYGZcb{}
\end{sphinxVerbatim}
\begin{quote}\begin{description}
\item[{Parameters}] \leavevmode\begin{itemize}
\item {} 
\sphinxstyleliteralstrong{partner\_id} (\sphinxstyleliteralemphasis{int}) -- the partner to update

\end{itemize}

\item[{Request Headers}] \leavevmode\begin{itemize}
\item {} 
\href{http://tools.ietf.org/html/rfc7235\#section-4.2}{Authorization} -- JSON Web Token to authenticate

\item {} 
\href{http://tools.ietf.org/html/rfc7231\#section-3.1.1.5}{Content-Type} -- data is sent as application/json or
application/x-www-form-urlencoded

\end{itemize}

\item[{Request JSON Object}] \leavevmode\begin{itemize}
\item {} 
\sphinxstyleliteralstrong{firstname} (\sphinxstyleliteralemphasis{string}) -- the partner's firstname

\item {} 
\sphinxstyleliteralstrong{lastname} (\sphinxstyleliteralemphasis{string}) -- the partner's lastname

\item {} 
\sphinxstyleliteralstrong{email} (\sphinxstyleliteralemphasis{string}) -- the partner's email address

\end{itemize}

\item[{Response Headers}] \leavevmode\begin{itemize}
\item {} 
\href{http://tools.ietf.org/html/rfc7231\#section-3.1.1.5}{Content-Type} -- data is received as application/json

\end{itemize}

\item[{Response JSON Object}] \leavevmode\begin{itemize}
\item {} 
\sphinxstyleliteralstrong{id} (\sphinxstyleliteralemphasis{int}) -- the partner's unique id

\item {} 
\sphinxstyleliteralstrong{type} (\sphinxstyleliteralemphasis{string}) -- the partner's type

\item {} 
\sphinxstyleliteralstrong{firstname} (\sphinxstyleliteralemphasis{string}) -- the partner's firstname

\item {} 
\sphinxstyleliteralstrong{lastname} (\sphinxstyleliteralemphasis{string}) -- the partner's lastname

\item {} 
\sphinxstyleliteralstrong{email} (\sphinxstyleliteralemphasis{string}) -- the partner's email address

\item {} 
\sphinxstyleliteralstrong{is\_active} (\sphinxstyleliteralemphasis{boolean}) -- the partner's status

\item {} 
\sphinxstyleliteralstrong{user\_id} (\sphinxstyleliteralemphasis{int}) -- the user account the partner is related to

\item {} 
\sphinxstyleliteralstrong{organization\_id} (\sphinxstyleliteralemphasis{int}) -- the organization the partner is related to

\item {} 
\sphinxstyleliteralstrong{invitation\_id} (\sphinxstyleliteralemphasis{int}) -- the invitation the partner is related to

\end{itemize}

\item[{Status Codes}] \leavevmode\begin{itemize}
\item {} 
\href{http://www.w3.org/Protocols/rfc2616/rfc2616-sec10.html\#sec10.2.1}{200 OK} -- Partner is updated

\item {} 
\href{http://www.w3.org/Protocols/rfc2616/rfc2616-sec10.html\#sec10.4.1}{400 Bad Request} -- Parameters are missing

\item {} 
\href{http://www.w3.org/Protocols/rfc2616/rfc2616-sec10.html\#sec10.4.1}{400 Bad Request} -- Token is not well-formed

\item {} 
\href{http://www.w3.org/Protocols/rfc2616/rfc2616-sec10.html\#sec10.4.2}{401 Unauthorized} -- Token has expired

\item {} 
\href{http://www.w3.org/Protocols/rfc2616/rfc2616-sec10.html\#sec10.4.2}{401 Unauthorized} -- User is not authorized

\item {} 
\href{http://www.w3.org/Protocols/rfc2616/rfc2616-sec10.html\#sec10.4.5}{404 Not Found} -- Partner is not found

\end{itemize}

\end{description}\end{quote}

\end{fulllineitems}



\begin{fulllineitems}
\phantomsection\label{\detokenize{resources/partner:delete--partners-(partner_id)}}\pysiglinewithargsret{\sphinxbfcode{DELETE~}\sphinxbfcode{/partners/}}{\emph{partner\_id}}{}~
Delete a partner.

In order to delete a partner, the authenticated user must be a partner
with admin access of the organization that the partner is associated
with.

\sphinxstylestrong{Example request}:

\begin{sphinxVerbatim}[commandchars=\\\{\}]
\PYG{n+nf}{DELETE} \PYG{n+nn}{/partners/1} \PYG{k+kr}{HTTP}\PYG{o}{/}\PYG{l+m}{1.1}
\PYG{n+na}{Host}\PYG{o}{:} \PYG{l}{example.com}
\PYG{n+na}{Authorization}\PYG{o}{:} \PYG{l}{Bearer \PYGZlt{}token\PYGZgt{}}
\end{sphinxVerbatim}

\sphinxstylestrong{Example response}:

\begin{sphinxVerbatim}[commandchars=\\\{\}]
\PYG{k+kr}{HTTP}\PYG{o}{/}\PYG{l+m}{1.1} \PYG{l+m}{204} \PYG{n+ne}{No Content}
\end{sphinxVerbatim}
\begin{quote}\begin{description}
\item[{Parameters}] \leavevmode\begin{itemize}
\item {} 
\sphinxstyleliteralstrong{partner\_id} (\sphinxstyleliteralemphasis{int}) -- the partner to delete

\end{itemize}

\item[{Request Headers}] \leavevmode\begin{itemize}
\item {} 
\href{http://tools.ietf.org/html/rfc7235\#section-4.2}{Authorization} -- JSON Web Token to authenticate

\end{itemize}

\item[{Status Codes}] \leavevmode\begin{itemize}
\item {} 
\href{http://www.w3.org/Protocols/rfc2616/rfc2616-sec10.html\#sec10.2.5}{204 No Content} -- Partner is deleted

\item {} 
\href{http://www.w3.org/Protocols/rfc2616/rfc2616-sec10.html\#sec10.4.1}{400 Bad Request} -- Token is not well-formed

\item {} 
\href{http://www.w3.org/Protocols/rfc2616/rfc2616-sec10.html\#sec10.4.2}{401 Unauthorized} -- Token has expired

\item {} 
\href{http://www.w3.org/Protocols/rfc2616/rfc2616-sec10.html\#sec10.4.2}{401 Unauthorized} -- User is not authorized

\item {} 
\href{http://www.w3.org/Protocols/rfc2616/rfc2616-sec10.html\#sec10.4.5}{404 Not Found} -- Partner is not found

\item {} 
\href{http://www.w3.org/Protocols/rfc2616/rfc2616-sec10.html\#sec10.4.10}{409 Conflict} -- Partner is the only admin of an organization

\end{itemize}

\end{description}\end{quote}

\end{fulllineitems}



\begin{fulllineitems}
\phantomsection\label{\detokenize{resources/partner:get--partners-(partner_id)}}\pysiglinewithargsret{\sphinxbfcode{GET~}\sphinxbfcode{/partners/}}{\emph{partner\_id}}{}~
Retrieve a partner.

In order to retrieve a partner, the authenticated user must be a
partner of the organization that the partner is associated with.

\sphinxstylestrong{Example request}:

\begin{sphinxVerbatim}[commandchars=\\\{\}]
\PYG{n+nf}{GET} \PYG{n+nn}{/partners/1} \PYG{k+kr}{HTTP}\PYG{o}{/}\PYG{l+m}{1.1}
\PYG{n+na}{Host}\PYG{o}{:} \PYG{l}{example.com}
\PYG{n+na}{Authorization}\PYG{o}{:} \PYG{l}{Bearer \PYGZlt{}token\PYGZgt{}}
\end{sphinxVerbatim}

\sphinxstylestrong{Example response}:

\begin{sphinxVerbatim}[commandchars=\\\{\}]
HTTP/1.1 200 OK
Content\PYGZhy{}Type: application/json

\PYGZob{}
    \PYGZsq{}id\PYGZsq{}: 1,
    \PYGZsq{}type\PYGZsq{}: \PYGZsq{}member\PYGZsq{},
    \PYGZsq{}firstname\PYGZsq{}: \PYGZsq{}John\PYGZsq{},
    \PYGZsq{}lastname\PYGZsq{}: \PYGZsq{}Doe\PYGZsq{},
    \PYGZsq{}email\PYGZsq{}: \PYGZsq{}john@example.org\PYGZsq{},
    \PYGZsq{}is\PYGZus{}active\PYGZsq{}: True,
    \PYGZsq{}user\PYGZus{}id\PYGZsq{}: 1,
    \PYGZsq{}organization\PYGZus{}id\PYGZsq{}: 1,
    \PYGZsq{}invitation\PYGZus{}id\PYGZsq{}: null
\PYGZcb{}
\end{sphinxVerbatim}
\begin{quote}\begin{description}
\item[{Parameters}] \leavevmode\begin{itemize}
\item {} 
\sphinxstyleliteralstrong{partner\_id} (\sphinxstyleliteralemphasis{int}) -- the partner to retrieve

\end{itemize}

\item[{Request Headers}] \leavevmode\begin{itemize}
\item {} 
\href{http://tools.ietf.org/html/rfc7235\#section-4.2}{Authorization} -- JSON Web Token to authenticate

\end{itemize}

\item[{Response Headers}] \leavevmode\begin{itemize}
\item {} 
\href{http://tools.ietf.org/html/rfc7231\#section-3.1.1.5}{Content-Type} -- data is received as application/json

\end{itemize}

\item[{Response JSON Object}] \leavevmode\begin{itemize}
\item {} 
\sphinxstyleliteralstrong{id} (\sphinxstyleliteralemphasis{int}) -- the partner's unique id

\item {} 
\sphinxstyleliteralstrong{type} (\sphinxstyleliteralemphasis{string}) -- the partner's type

\item {} 
\sphinxstyleliteralstrong{firstname} (\sphinxstyleliteralemphasis{string}) -- the partner's firstname

\item {} 
\sphinxstyleliteralstrong{lastname} (\sphinxstyleliteralemphasis{string}) -- the partner's lastname

\item {} 
\sphinxstyleliteralstrong{email} (\sphinxstyleliteralemphasis{string}) -- the partner's email address

\item {} 
\sphinxstyleliteralstrong{is\_active} (\sphinxstyleliteralemphasis{boolean}) -- the partner's status

\item {} 
\sphinxstyleliteralstrong{user\_id} (\sphinxstyleliteralemphasis{int}) -- the user account the partner is related to

\item {} 
\sphinxstyleliteralstrong{organization\_id} (\sphinxstyleliteralemphasis{int}) -- the organization the partner is related to

\item {} 
\sphinxstyleliteralstrong{invitation\_id} (\sphinxstyleliteralemphasis{int}) -- the invitation the partner is related to

\end{itemize}

\item[{Status Codes}] \leavevmode\begin{itemize}
\item {} 
\href{http://www.w3.org/Protocols/rfc2616/rfc2616-sec10.html\#sec10.2.1}{200 OK} -- Partner is retrieved

\item {} 
\href{http://www.w3.org/Protocols/rfc2616/rfc2616-sec10.html\#sec10.4.1}{400 Bad Request} -- Token is not well-formed

\item {} 
\href{http://www.w3.org/Protocols/rfc2616/rfc2616-sec10.html\#sec10.4.2}{401 Unauthorized} -- Token has expired

\item {} 
\href{http://www.w3.org/Protocols/rfc2616/rfc2616-sec10.html\#sec10.4.2}{401 Unauthorized} -- User is not authorized

\item {} 
\href{http://www.w3.org/Protocols/rfc2616/rfc2616-sec10.html\#sec10.4.5}{404 Not Found} -- Partner is not found

\end{itemize}

\end{description}\end{quote}

\end{fulllineitems}



\subsection{Memberships}
\label{\detokenize{resources/partner:memberships}}
Represents the relationships of a partner with roles and circles. See {\hyperref[\detokenize{resources/role:role}]{\sphinxcrossref{\DUrole{std,std-ref}{Role}}}} or {\hyperref[\detokenize{resources/circle:circle}]{\sphinxcrossref{\DUrole{std,std-ref}{Circle}}}} for a description of a single role or a single circle.

\noindent\begin{tabulary}{\linewidth}{|L|L|L|}
\hline
\sphinxstylethead{\relax 
Resource
\unskip}\relax &\sphinxstylethead{\relax 
Operation
\unskip}\relax &\sphinxstylethead{\relax 
Description
\unskip}\relax \\
\hline
Partner Memberships
&
{\hyperref[\detokenize{resources/partner:get--partners-(partner_id)-memberships}]{\emph{GET /partners/(partner\_id)/memberships}}}
&
List memberships of a partner.
\\
\hline\end{tabulary}


\begin{DUlineblock}{0em}
\item[] 
\end{DUlineblock}


\begin{fulllineitems}
\phantomsection\label{\detokenize{resources/partner:get--partners-(partner_id)-memberships}}\pysiglinewithargsret{\sphinxbfcode{GET~}\sphinxbfcode{/partners/}}{\emph{partner\_id}}{\sphinxbfcode{/memberships}}~
List memberships of a partner.

In order to list the memberships of a partner, the authenticated user
must be a partner of the organization that the partner is associated
with.

\sphinxstylestrong{Example request}:

\begin{sphinxVerbatim}[commandchars=\\\{\}]
\PYG{n+nf}{GET} \PYG{n+nn}{/partners/1/memberships} \PYG{k+kr}{HTTP}\PYG{o}{/}\PYG{l+m}{1.1}
\PYG{n+na}{Host}\PYG{o}{:} \PYG{l}{example.com}
\PYG{n+na}{Authorization}\PYG{o}{:} \PYG{l}{Bearer \PYGZlt{}token\PYGZgt{}}
\end{sphinxVerbatim}

\sphinxstylestrong{Example response}:

\begin{sphinxVerbatim}[commandchars=\\\{\}]
HTTP/1.1 200 OK
Content\PYGZhy{}Type: application/json

[
    \PYGZob{}
        \PYGZsq{}id\PYGZsq{}: 1,
        \PYGZsq{}type\PYGZsq{}: \PYGZsq{}circle\PYGZsq{},
        \PYGZsq{}name\PYGZsq{}: \PYGZsq{}My Organization\PYGZsq{},
        \PYGZsq{}pupose\PYGZsq{}: \PYGZsq{}My Organization\PYGZsq{}s purpose\PYGZsq{},
        \PYGZsq{}parent\PYGZus{}role\PYGZus{}id\PYGZsq{}: null,
        \PYGZsq{}organization\PYGZus{}id\PYGZsq{}: 1
    \PYGZcb{},
    \PYGZob{}
        \PYGZsq{}id\PYGZsq{}: 2,
        \PYGZsq{}type\PYGZsq{}: \PYGZsq{}lead\PYGZus{}link\PYGZsq{},
        \PYGZsq{}name\PYGZsq{}: \PYGZsq{}Lead Link\PYGZsq{}s name\PYGZsq{},
        \PYGZsq{}purpose\PYGZsq{}: \PYGZsq{}Lead Link\PYGZsq{}s purpose\PYGZsq{},
        \PYGZsq{}parent\PYGZus{}role\PYGZus{}id\PYGZsq{}: 1,
        \PYGZsq{}organization\PYGZus{}id\PYGZsq{}: 1
    \PYGZcb{},
    \PYGZob{}
        \PYGZsq{}id\PYGZsq{}: 3,
        \PYGZsq{}type\PYGZsq{}: \PYGZsq{}secretary\PYGZsq{},
        \PYGZsq{}name\PYGZsq{}: \PYGZsq{}Secretary\PYGZsq{}s name\PYGZsq{},
        \PYGZsq{}purpose\PYGZsq{}: \PYGZsq{}Secretary\PYGZsq{}s purpose\PYGZsq{},
        \PYGZsq{}parent\PYGZus{}role\PYGZus{}id\PYGZsq{}: 1,
        \PYGZsq{}organization\PYGZus{}id\PYGZsq{}: 1
    \PYGZcb{},
    \PYGZob{}
        \PYGZsq{}id\PYGZsq{}: 4,
        \PYGZsq{}type\PYGZsq{}: \PYGZsq{}facilitator\PYGZsq{},
        \PYGZsq{}name\PYGZsq{}: \PYGZsq{}Facilitator\PYGZsq{}s name\PYGZsq{},
        \PYGZsq{}purpose\PYGZsq{}: \PYGZsq{}Facilitator\PYGZsq{}s purpose\PYGZsq{},
        \PYGZsq{}parent\PYGZus{}role\PYGZus{}id\PYGZsq{}: 1,
        \PYGZsq{}organization\PYGZus{}id\PYGZsq{}: 1
    \PYGZcb{},
    \PYGZob{}
        \PYGZsq{}id\PYGZsq{}: 5,
        \PYGZsq{}type\PYGZsq{}: \PYGZsq{}custom\PYGZsq{},
        \PYGZsq{}name\PYGZsq{}: \PYGZsq{}My Role\PYGZsq{}s name\PYGZsq{},
        \PYGZsq{}purpose\PYGZsq{}: \PYGZsq{}My Role\PYGZsq{}s purpose\PYGZsq{},
        \PYGZsq{}parent\PYGZus{}role\PYGZus{}id\PYGZsq{}: 1,
        \PYGZsq{}organization\PYGZus{}id\PYGZsq{}: 1
    \PYGZcb{},
    \PYGZob{}
        \PYGZsq{}id\PYGZsq{}: 6,
        \PYGZsq{}type\PYGZsq{}: \PYGZsq{}circle\PYGZsq{},
        \PYGZsq{}name\PYGZsq{}: \PYGZsq{}My Circle\PYGZsq{}s name\PYGZsq{},
        \PYGZsq{}purpose\PYGZsq{}: \PYGZsq{}My Circle\PYGZsq{}s purpose\PYGZsq{},
        \PYGZsq{}parent\PYGZus{}role\PYGZus{}id\PYGZsq{}: 1,
        \PYGZsq{}organization\PYGZus{}id\PYGZsq{}: 1
    \PYGZcb{}
]
\end{sphinxVerbatim}
\begin{quote}\begin{description}
\item[{Parameters}] \leavevmode\begin{itemize}
\item {} 
\sphinxstyleliteralstrong{partner\_id} (\sphinxstyleliteralemphasis{int}) -- the partner the memberships are listed for

\end{itemize}

\item[{Request Headers}] \leavevmode\begin{itemize}
\item {} 
\href{http://tools.ietf.org/html/rfc7235\#section-4.2}{Authorization} -- JSON Web Token to authenticate

\end{itemize}

\item[{Response Headers}] \leavevmode\begin{itemize}
\item {} 
\href{http://tools.ietf.org/html/rfc7231\#section-3.1.1.5}{Content-Type} -- data is received as application/json

\end{itemize}

\item[{Response JSON Array of Objects}] \leavevmode\begin{itemize}
\item {} 
\sphinxstyleliteralstrong{id} (\sphinxstyleliteralemphasis{int}) -- the role's unique id

\item {} 
\sphinxstyleliteralstrong{type} (\sphinxstyleliteralemphasis{string}) -- the role's type

\item {} 
\sphinxstyleliteralstrong{name} (\sphinxstyleliteralemphasis{string}) -- the role's name

\item {} 
\sphinxstyleliteralstrong{purpose} (\sphinxstyleliteralemphasis{string}) -- the role's purpose

\item {} 
\sphinxstyleliteralstrong{parent\_role\_id} (\sphinxstyleliteralemphasis{int}) -- the parent role the role is related to

\item {} 
\sphinxstyleliteralstrong{organization\_id} (\sphinxstyleliteralemphasis{int}) -- the organization the role is related to

\end{itemize}

\item[{Status Codes}] \leavevmode\begin{itemize}
\item {} 
\href{http://www.w3.org/Protocols/rfc2616/rfc2616-sec10.html\#sec10.2.1}{200 OK} -- Memberships are listed

\item {} 
\href{http://www.w3.org/Protocols/rfc2616/rfc2616-sec10.html\#sec10.4.1}{400 Bad Request} -- Token is not well-formed

\item {} 
\href{http://www.w3.org/Protocols/rfc2616/rfc2616-sec10.html\#sec10.4.2}{401 Unauthorized} -- Token has expired

\item {} 
\href{http://www.w3.org/Protocols/rfc2616/rfc2616-sec10.html\#sec10.4.2}{401 Unauthorized} -- User is not authorized

\item {} 
\href{http://www.w3.org/Protocols/rfc2616/rfc2616-sec10.html\#sec10.4.5}{404 Not Found} -- Partner is not found

\end{itemize}

\end{description}\end{quote}

\end{fulllineitems}



\section{Policy}
\label{\detokenize{resources/policy::doc}}\label{\detokenize{resources/policy:id1}}\label{\detokenize{resources/policy:policy}}
Represents a policy.

\noindent\begin{tabulary}{\linewidth}{|L|L|L|}
\hline
\sphinxstylethead{\relax 
Resource
\unskip}\relax &\sphinxstylethead{\relax 
Operation
\unskip}\relax &\sphinxstylethead{\relax 
Description
\unskip}\relax \\
\hline
Policy
&
{\hyperref[\detokenize{resources/policy:put--policies-(policy_id)}]{\emph{PUT /policies/(policy\_id)}}}
&
Update a policy.
\\
\hline&
{\hyperref[\detokenize{resources/policy:delete--policies-(policy_id)}]{\emph{DELETE /policies/(policy\_id)}}}
&
Delete a policy.
\\
\hline&
{\hyperref[\detokenize{resources/policy:get--policies-(policy_id)}]{\emph{GET /policies/(policy\_id)}}}
&
Retrieve a policy.
\\
\hline\end{tabulary}


\begin{DUlineblock}{0em}
\item[] 
\end{DUlineblock}


\begin{fulllineitems}
\phantomsection\label{\detokenize{resources/policy:put--policies-(policy_id)}}\pysiglinewithargsret{\sphinxbfcode{PUT~}\sphinxbfcode{/policies/}}{\emph{policy\_id}}{}~
Update a policy.

\sphinxstylestrong{Example request}:

\begin{sphinxVerbatim}[commandchars=\\\{\}]
PUT /policies/1 HTTP/1.1
Host: example.com
Authorization: Bearer \PYGZlt{}token\PYGZgt{}
Content\PYGZhy{}Type: application/json

\PYGZob{}
    \PYGZsq{}title\PYGZsq{}: \PYGZsq{}Policy\PYGZsq{}s new title\PYGZsq{}
\PYGZcb{}
\end{sphinxVerbatim}

\sphinxstylestrong{Example response}:

\begin{sphinxVerbatim}[commandchars=\\\{\}]
HTTP/1.1 200 OK
Content\PYGZhy{}Type: application/json

\PYGZob{}
    \PYGZsq{}id\PYGZsq{}: 1,
    \PYGZsq{}title\PYGZsq{}: \PYGZsq{}Policy\PYGZsq{}s new title\PYGZsq{},
    \PYGZsq{}domain\PYGZus{}id\PYGZsq{}: 1
\PYGZcb{}
\end{sphinxVerbatim}
\begin{quote}\begin{description}
\item[{Parameters}] \leavevmode\begin{itemize}
\item {} 
\sphinxstyleliteralstrong{policy\_id} (\sphinxstyleliteralemphasis{int}) -- the policy to update

\end{itemize}

\item[{Request Headers}] \leavevmode\begin{itemize}
\item {} 
\href{http://tools.ietf.org/html/rfc7235\#section-4.2}{Authorization} -- JSON Web Token to authenticate

\item {} 
\href{http://tools.ietf.org/html/rfc7231\#section-3.1.1.5}{Content-Type} -- data is sent as application/json or
application/x-www-form-urlencoded

\end{itemize}

\item[{Request JSON Object}] \leavevmode\begin{itemize}
\item {} 
\sphinxstyleliteralstrong{name} (\sphinxstyleliteralemphasis{string}) -- the policy's title

\end{itemize}

\item[{Response Headers}] \leavevmode\begin{itemize}
\item {} 
\href{http://tools.ietf.org/html/rfc7231\#section-3.1.1.5}{Content-Type} -- data is received as application/json

\end{itemize}

\item[{Response JSON Object}] \leavevmode\begin{itemize}
\item {} 
\sphinxstyleliteralstrong{id} (\sphinxstyleliteralemphasis{int}) -- the policy's unique id

\item {} 
\sphinxstyleliteralstrong{title} (\sphinxstyleliteralemphasis{string}) -- the policy's title

\item {} 
\sphinxstyleliteralstrong{domain\_id} (\sphinxstyleliteralemphasis{int}) -- the domain the policy is related to

\end{itemize}

\item[{Status Codes}] \leavevmode\begin{itemize}
\item {} 
\href{http://www.w3.org/Protocols/rfc2616/rfc2616-sec10.html\#sec10.2.1}{200 OK} -- Policy is updated

\item {} 
\href{http://www.w3.org/Protocols/rfc2616/rfc2616-sec10.html\#sec10.4.1}{400 Bad Request} -- Parameters are missing

\item {} 
\href{http://www.w3.org/Protocols/rfc2616/rfc2616-sec10.html\#sec10.4.1}{400 Bad Request} -- Token is not well-formed

\item {} 
\href{http://www.w3.org/Protocols/rfc2616/rfc2616-sec10.html\#sec10.4.2}{401 Unauthorized} -- Token has expired

\item {} 
\href{http://www.w3.org/Protocols/rfc2616/rfc2616-sec10.html\#sec10.4.2}{401 Unauthorized} -- User is not authorized

\item {} 
\href{http://www.w3.org/Protocols/rfc2616/rfc2616-sec10.html\#sec10.4.5}{404 Not Found} -- Policy is not found

\end{itemize}

\end{description}\end{quote}

\end{fulllineitems}



\begin{fulllineitems}
\phantomsection\label{\detokenize{resources/policy:delete--policies-(policy_id)}}\pysiglinewithargsret{\sphinxbfcode{DELETE~}\sphinxbfcode{/policies/}}{\emph{policy\_id}}{}~
Delete a policy.

\sphinxstylestrong{Example request}:

\begin{sphinxVerbatim}[commandchars=\\\{\}]
\PYG{n+nf}{DELETE} \PYG{n+nn}{/policies/1} \PYG{k+kr}{HTTP}\PYG{o}{/}\PYG{l+m}{1.1}
\PYG{n+na}{Host}\PYG{o}{:} \PYG{l}{example.com}
\PYG{n+na}{Authorization}\PYG{o}{:} \PYG{l}{Bearer \PYGZlt{}token\PYGZgt{}}
\end{sphinxVerbatim}

\sphinxstylestrong{Example response}:

\begin{sphinxVerbatim}[commandchars=\\\{\}]
\PYG{k+kr}{HTTP}\PYG{o}{/}\PYG{l+m}{1.1} \PYG{l+m}{204} \PYG{n+ne}{No Content}
\end{sphinxVerbatim}
\begin{quote}\begin{description}
\item[{Parameters}] \leavevmode\begin{itemize}
\item {} 
\sphinxstyleliteralstrong{policy\_id} (\sphinxstyleliteralemphasis{int}) -- the policy to delete

\end{itemize}

\item[{Request Headers}] \leavevmode\begin{itemize}
\item {} 
\href{http://tools.ietf.org/html/rfc7235\#section-4.2}{Authorization} -- JSON Web Token to authenticate

\end{itemize}

\item[{Status Codes}] \leavevmode\begin{itemize}
\item {} 
\href{http://www.w3.org/Protocols/rfc2616/rfc2616-sec10.html\#sec10.2.5}{204 No Content} -- Policy is deleted

\item {} 
\href{http://www.w3.org/Protocols/rfc2616/rfc2616-sec10.html\#sec10.4.1}{400 Bad Request} -- Token is not well-formed

\item {} 
\href{http://www.w3.org/Protocols/rfc2616/rfc2616-sec10.html\#sec10.4.2}{401 Unauthorized} -- Token has expired

\item {} 
\href{http://www.w3.org/Protocols/rfc2616/rfc2616-sec10.html\#sec10.4.2}{401 Unauthorized} -- User is not authorized

\item {} 
\href{http://www.w3.org/Protocols/rfc2616/rfc2616-sec10.html\#sec10.4.5}{404 Not Found} -- Policy is not found

\end{itemize}

\end{description}\end{quote}

\end{fulllineitems}



\begin{fulllineitems}
\phantomsection\label{\detokenize{resources/policy:get--policies-(policy_id)}}\pysiglinewithargsret{\sphinxbfcode{GET~}\sphinxbfcode{/policies/}}{\emph{policy\_id}}{}~
Retrieve a policy.

\sphinxstylestrong{Example request}:

\begin{sphinxVerbatim}[commandchars=\\\{\}]
\PYG{n+nf}{GET} \PYG{n+nn}{/policies/1} \PYG{k+kr}{HTTP}\PYG{o}{/}\PYG{l+m}{1.1}
\PYG{n+na}{Host}\PYG{o}{:} \PYG{l}{example.com}
\PYG{n+na}{Authorization}\PYG{o}{:} \PYG{l}{Bearer \PYGZlt{}token\PYGZgt{}}
\end{sphinxVerbatim}

\sphinxstylestrong{Example response}:

\begin{sphinxVerbatim}[commandchars=\\\{\}]
HTTP/1.1 200 OK
Content\PYGZhy{}Type: application/json

\PYGZob{}
    \PYGZsq{}id\PYGZsq{}: 1,
    \PYGZsq{}title\PYGZsq{}: \PYGZsq{}Policy\PYGZsq{}s title\PYGZsq{},
    \PYGZsq{}domain\PYGZus{}id\PYGZsq{}: 1
\PYGZcb{}
\end{sphinxVerbatim}
\begin{quote}\begin{description}
\item[{Parameters}] \leavevmode\begin{itemize}
\item {} 
\sphinxstyleliteralstrong{policy\_id} (\sphinxstyleliteralemphasis{int}) -- the policy to retrieve

\end{itemize}

\item[{Request Headers}] \leavevmode\begin{itemize}
\item {} 
\href{http://tools.ietf.org/html/rfc7235\#section-4.2}{Authorization} -- JSON Web Token to authenticate

\end{itemize}

\item[{Response Headers}] \leavevmode\begin{itemize}
\item {} 
\href{http://tools.ietf.org/html/rfc7231\#section-3.1.1.5}{Content-Type} -- data is received as application/json

\end{itemize}

\item[{Response JSON Object}] \leavevmode\begin{itemize}
\item {} 
\sphinxstyleliteralstrong{id} (\sphinxstyleliteralemphasis{int}) -- the policy's unique id

\item {} 
\sphinxstyleliteralstrong{title} (\sphinxstyleliteralemphasis{string}) -- the policy's title

\item {} 
\sphinxstyleliteralstrong{domain\_id} (\sphinxstyleliteralemphasis{int}) -- the domain the policy is related to

\end{itemize}

\item[{Status Codes}] \leavevmode\begin{itemize}
\item {} 
\href{http://www.w3.org/Protocols/rfc2616/rfc2616-sec10.html\#sec10.2.1}{200 OK} -- Policy is retrieved

\item {} 
\href{http://www.w3.org/Protocols/rfc2616/rfc2616-sec10.html\#sec10.4.1}{400 Bad Request} -- Token is not well-formed

\item {} 
\href{http://www.w3.org/Protocols/rfc2616/rfc2616-sec10.html\#sec10.4.2}{401 Unauthorized} -- Token has expired

\item {} 
\href{http://www.w3.org/Protocols/rfc2616/rfc2616-sec10.html\#sec10.4.2}{401 Unauthorized} -- User is not authorized

\item {} 
\href{http://www.w3.org/Protocols/rfc2616/rfc2616-sec10.html\#sec10.4.5}{404 Not Found} -- Policy is not found

\end{itemize}

\end{description}\end{quote}

\end{fulllineitems}



\section{Role}
\label{\detokenize{resources/role:role}}\label{\detokenize{resources/role::doc}}\label{\detokenize{resources/role:id1}}
Represents a role. A role defines domains to control and accountabilities to perform. A {\hyperref[\detokenize{resources/partner:partner}]{\sphinxcrossref{\DUrole{std,std-ref}{Partner}}}} can be assigned to core roles (e.g. facilitator, secretary, lead link, rep link, cross link) as well as custom roles.

\noindent\begin{tabulary}{\linewidth}{|L|L|L|}
\hline
\sphinxstylethead{\relax 
Resource
\unskip}\relax &\sphinxstylethead{\relax 
Operation
\unskip}\relax &\sphinxstylethead{\relax 
Description
\unskip}\relax \\
\hline
Role
&
{\hyperref[\detokenize{resources/role:put--roles-(role_id)}]{\emph{PUT /roles/(role\_id)}}}
&
Update a role.
\\
\hline&
{\hyperref[\detokenize{resources/role:delete--roles-(role_id)}]{\emph{DELETE /roles/(role\_id)}}}
&
Delete a role.
\\
\hline&
{\hyperref[\detokenize{resources/role:get--roles-(role_id)}]{\emph{GET /roles/(role\_id)}}}
&
Retrieve a role.
\\
\hline\end{tabulary}


\begin{DUlineblock}{0em}
\item[] 
\end{DUlineblock}


\begin{fulllineitems}
\phantomsection\label{\detokenize{resources/role:put--roles-(role_id)}}\pysiglinewithargsret{\sphinxbfcode{PUT~}\sphinxbfcode{/roles/}}{\emph{role\_id}}{}~
Update a role.

\sphinxstylestrong{Example request}:

\begin{sphinxVerbatim}[commandchars=\\\{\}]
PUT /roles/5 HTTP/1.1
Host: example.com
Authorization: Bearer \PYGZlt{}token\PYGZgt{}
Content\PYGZhy{}Type: application/json

\PYGZob{}
    \PYGZsq{}name\PYGZsq{}: \PYGZsq{}My Role\PYGZsq{}s new name\PYGZsq{},
    \PYGZsq{}purpose\PYGZsq{}: \PYGZsq{}My Role\PYGZsq{}s new purpose\PYGZsq{}
\PYGZcb{}
\end{sphinxVerbatim}

\sphinxstylestrong{Example response}:

\begin{sphinxVerbatim}[commandchars=\\\{\}]
HTTP/1.1 200 OK
Content\PYGZhy{}Type: application/json

\PYGZob{}
    \PYGZsq{}id\PYGZsq{}: 5,
    \PYGZsq{}type\PYGZsq{}: \PYGZsq{}custom\PYGZsq{},
    \PYGZsq{}name\PYGZsq{}: \PYGZsq{}My Role\PYGZsq{}s new name\PYGZsq{},
    \PYGZsq{}purpose\PYGZsq{}: \PYGZsq{}My Role\PYGZsq{}s new purpose\PYGZsq{},
    \PYGZsq{}parent\PYGZus{}role\PYGZus{}id\PYGZsq{}: 1,
    \PYGZsq{}organization\PYGZus{}id\PYGZsq{}: 1
\PYGZcb{}
\end{sphinxVerbatim}
\begin{quote}\begin{description}
\item[{Parameters}] \leavevmode\begin{itemize}
\item {} 
\sphinxstyleliteralstrong{role\_id} (\sphinxstyleliteralemphasis{int}) -- the role to update

\end{itemize}

\item[{Request Headers}] \leavevmode\begin{itemize}
\item {} 
\href{http://tools.ietf.org/html/rfc7235\#section-4.2}{Authorization} -- JSON Web Token to authenticate

\item {} 
\href{http://tools.ietf.org/html/rfc7231\#section-3.1.1.5}{Content-Type} -- data is sent as application/json or
application/x-www-form-urlencoded

\end{itemize}

\item[{Request JSON Object}] \leavevmode\begin{itemize}
\item {} 
\sphinxstyleliteralstrong{name} (\sphinxstyleliteralemphasis{string}) -- the role's name

\item {} 
\sphinxstyleliteralstrong{purpose} (\sphinxstyleliteralemphasis{string}) -- the role's purpose

\end{itemize}

\item[{Response Headers}] \leavevmode\begin{itemize}
\item {} 
\href{http://tools.ietf.org/html/rfc7231\#section-3.1.1.5}{Content-Type} -- data is received as application/json

\end{itemize}

\item[{Response JSON Object}] \leavevmode\begin{itemize}
\item {} 
\sphinxstyleliteralstrong{id} (\sphinxstyleliteralemphasis{int}) -- the role's unique id

\item {} 
\sphinxstyleliteralstrong{type} (\sphinxstyleliteralemphasis{string}) -- the role's type

\item {} 
\sphinxstyleliteralstrong{name} (\sphinxstyleliteralemphasis{string}) -- the role's name

\item {} 
\sphinxstyleliteralstrong{purpose} (\sphinxstyleliteralemphasis{string}) -- the role's purpose

\item {} 
\sphinxstyleliteralstrong{parent\_role\_id} (\sphinxstyleliteralemphasis{int}) -- the parent role the role is related to

\item {} 
\sphinxstyleliteralstrong{organization\_id} (\sphinxstyleliteralemphasis{int}) -- the organization the role is related to

\end{itemize}

\item[{Status Codes}] \leavevmode\begin{itemize}
\item {} 
\href{http://www.w3.org/Protocols/rfc2616/rfc2616-sec10.html\#sec10.2.1}{200 OK} -- Role is updated

\item {} 
\href{http://www.w3.org/Protocols/rfc2616/rfc2616-sec10.html\#sec10.4.1}{400 Bad Request} -- Parameters are missing

\item {} 
\href{http://www.w3.org/Protocols/rfc2616/rfc2616-sec10.html\#sec10.4.1}{400 Bad Request} -- Token is not well-formed

\item {} 
\href{http://www.w3.org/Protocols/rfc2616/rfc2616-sec10.html\#sec10.4.2}{401 Unauthorized} -- Token has expired

\item {} 
\href{http://www.w3.org/Protocols/rfc2616/rfc2616-sec10.html\#sec10.4.2}{401 Unauthorized} -- User is not authorized

\item {} 
\href{http://www.w3.org/Protocols/rfc2616/rfc2616-sec10.html\#sec10.4.5}{404 Not Found} -- Role is not found

\end{itemize}

\end{description}\end{quote}

\end{fulllineitems}



\begin{fulllineitems}
\phantomsection\label{\detokenize{resources/role:delete--roles-(role_id)}}\pysiglinewithargsret{\sphinxbfcode{DELETE~}\sphinxbfcode{/roles/}}{\emph{role\_id}}{}~
Delete a role.

\sphinxstylestrong{Example request}:

\begin{sphinxVerbatim}[commandchars=\\\{\}]
\PYG{n+nf}{DELETE} \PYG{n+nn}{/roles/5} \PYG{k+kr}{HTTP}\PYG{o}{/}\PYG{l+m}{1.1}
\PYG{n+na}{Host}\PYG{o}{:} \PYG{l}{example.com}
\PYG{n+na}{Authorization}\PYG{o}{:} \PYG{l}{Bearer \PYGZlt{}token\PYGZgt{}}
\end{sphinxVerbatim}

\sphinxstylestrong{Example response}:

\begin{sphinxVerbatim}[commandchars=\\\{\}]
\PYG{k+kr}{HTTP}\PYG{o}{/}\PYG{l+m}{1.1} \PYG{l+m}{204} \PYG{n+ne}{No Content}
\end{sphinxVerbatim}
\begin{quote}\begin{description}
\item[{Parameters}] \leavevmode\begin{itemize}
\item {} 
\sphinxstyleliteralstrong{role\_id} (\sphinxstyleliteralemphasis{int}) -- the role to delete

\end{itemize}

\item[{Request Headers}] \leavevmode\begin{itemize}
\item {} 
\href{http://tools.ietf.org/html/rfc7235\#section-4.2}{Authorization} -- JSON Web Token to authenticate

\end{itemize}

\item[{Status Codes}] \leavevmode\begin{itemize}
\item {} 
\href{http://www.w3.org/Protocols/rfc2616/rfc2616-sec10.html\#sec10.2.5}{204 No Content} -- Role is deleted

\item {} 
\href{http://www.w3.org/Protocols/rfc2616/rfc2616-sec10.html\#sec10.4.1}{400 Bad Request} -- Token is not well-formed

\item {} 
\href{http://www.w3.org/Protocols/rfc2616/rfc2616-sec10.html\#sec10.4.2}{401 Unauthorized} -- Token has expired

\item {} 
\href{http://www.w3.org/Protocols/rfc2616/rfc2616-sec10.html\#sec10.4.2}{401 Unauthorized} -- User is not authorized

\item {} 
\href{http://www.w3.org/Protocols/rfc2616/rfc2616-sec10.html\#sec10.4.5}{404 Not Found} -- Role is not found

\item {} 
\href{http://www.w3.org/Protocols/rfc2616/rfc2616-sec10.html\#sec10.4.10}{409 Conflict} -- Role type is other than custom

\item {} 
\href{http://www.w3.org/Protocols/rfc2616/rfc2616-sec10.html\#sec10.4.10}{409 Conflict} -- Role is an anchor circle of an organization

\end{itemize}

\end{description}\end{quote}

\end{fulllineitems}



\begin{fulllineitems}
\phantomsection\label{\detokenize{resources/role:get--roles-(role_id)}}\pysiglinewithargsret{\sphinxbfcode{GET~}\sphinxbfcode{/roles/}}{\emph{role\_id}}{}~
Retrieve a role.

\sphinxstylestrong{Example request}:

\begin{sphinxVerbatim}[commandchars=\\\{\}]
\PYG{n+nf}{GET} \PYG{n+nn}{/roles/1} \PYG{k+kr}{HTTP}\PYG{o}{/}\PYG{l+m}{1.1}
\PYG{n+na}{Host}\PYG{o}{:} \PYG{l}{example.com}
\PYG{n+na}{Authorization}\PYG{o}{:} \PYG{l}{Bearer \PYGZlt{}token\PYGZgt{}}
\end{sphinxVerbatim}

\sphinxstylestrong{Example response}:

\begin{sphinxVerbatim}[commandchars=\\\{\}]
HTTP/1.1 200 OK
Content\PYGZhy{}Type: application/json

\PYGZob{}
    \PYGZsq{}id\PYGZsq{}: 5,
    \PYGZsq{}type\PYGZsq{}: \PYGZsq{}custom\PYGZsq{},
    \PYGZsq{}name\PYGZsq{}: \PYGZsq{}My Role\PYGZsq{}s name\PYGZsq{},
    \PYGZsq{}purpose\PYGZsq{}: \PYGZsq{}My Role\PYGZsq{}s purpose\PYGZsq{},
    \PYGZsq{}parent\PYGZus{}role\PYGZus{}id\PYGZsq{}: 1,
    \PYGZsq{}organization\PYGZus{}id\PYGZsq{}: 1
\PYGZcb{}
\end{sphinxVerbatim}
\begin{quote}\begin{description}
\item[{Parameters}] \leavevmode\begin{itemize}
\item {} 
\sphinxstyleliteralstrong{role\_id} (\sphinxstyleliteralemphasis{int}) -- the role to retrieve

\end{itemize}

\item[{Request Headers}] \leavevmode\begin{itemize}
\item {} 
\href{http://tools.ietf.org/html/rfc7235\#section-4.2}{Authorization} -- JSON Web Token to authenticate

\end{itemize}

\item[{Response Headers}] \leavevmode\begin{itemize}
\item {} 
\href{http://tools.ietf.org/html/rfc7231\#section-3.1.1.5}{Content-Type} -- data is received as application/json

\end{itemize}

\item[{Response JSON Object}] \leavevmode\begin{itemize}
\item {} 
\sphinxstyleliteralstrong{id} (\sphinxstyleliteralemphasis{int}) -- the role's unique id

\item {} 
\sphinxstyleliteralstrong{type} (\sphinxstyleliteralemphasis{string}) -- the role's type

\item {} 
\sphinxstyleliteralstrong{name} (\sphinxstyleliteralemphasis{string}) -- the role's name

\item {} 
\sphinxstyleliteralstrong{purpose} (\sphinxstyleliteralemphasis{string}) -- the role's purpose

\item {} 
\sphinxstyleliteralstrong{parent\_role\_id} (\sphinxstyleliteralemphasis{int}) -- the parent role the role is related to

\item {} 
\sphinxstyleliteralstrong{organization\_id} (\sphinxstyleliteralemphasis{int}) -- the organization the role is related to

\end{itemize}

\item[{Status Codes}] \leavevmode\begin{itemize}
\item {} 
\href{http://www.w3.org/Protocols/rfc2616/rfc2616-sec10.html\#sec10.2.1}{200 OK} -- Role is retrieved

\item {} 
\href{http://www.w3.org/Protocols/rfc2616/rfc2616-sec10.html\#sec10.4.1}{400 Bad Request} -- Token is not well-formed

\item {} 
\href{http://www.w3.org/Protocols/rfc2616/rfc2616-sec10.html\#sec10.4.2}{401 Unauthorized} -- Token has expired

\item {} 
\href{http://www.w3.org/Protocols/rfc2616/rfc2616-sec10.html\#sec10.4.2}{401 Unauthorized} -- User is not authorized

\item {} 
\href{http://www.w3.org/Protocols/rfc2616/rfc2616-sec10.html\#sec10.4.5}{404 Not Found} -- Role is not found

\end{itemize}

\end{description}\end{quote}

\end{fulllineitems}



\subsection{Accountabilities}
\label{\detokenize{resources/role:accountabilities}}
Represents the accountabilities of a role. See {\hyperref[\detokenize{resources/accountability:accountability}]{\sphinxcrossref{\DUrole{std,std-ref}{Accountability}}}} for a description of a single accountability.

\noindent\begin{tabulary}{\linewidth}{|L|L|L|}
\hline
\sphinxstylethead{\relax 
Resource
\unskip}\relax &\sphinxstylethead{\relax 
Operation
\unskip}\relax &\sphinxstylethead{\relax 
Description
\unskip}\relax \\
\hline
Role Accountabilities
&
{\hyperref[\detokenize{resources/role:post--roles-(role_id)-accountabilities}]{\emph{POST /roles/(role\_id)/accountabilities}}}
&
Add an accountability to a role.
\\
\hline&
{\hyperref[\detokenize{resources/role:get--roles-(role_id)-accountabilities}]{\emph{GET /roles/(role\_id)/accountabilities}}}
&
List accountabilities of a role.
\\
\hline\end{tabulary}


\begin{DUlineblock}{0em}
\item[] 
\end{DUlineblock}


\begin{fulllineitems}
\phantomsection\label{\detokenize{resources/role:post--roles-(role_id)-accountabilities}}\pysiglinewithargsret{\sphinxbfcode{POST~}\sphinxbfcode{/roles/}}{\emph{role\_id}}{\sphinxbfcode{/accountabilities}}~
Add an accountability to a role.

\sphinxstylestrong{Example request}:

\begin{sphinxVerbatim}[commandchars=\\\{\}]
POST /roles/1/accountabilities HTTP/1.1
Host: example.com
Authorization: Bearer \PYGZlt{}token\PYGZgt{}
Content\PYGZhy{}Type: application/json

\PYGZob{}
    \PYGZsq{}title\PYGZsq{}: \PYGZsq{}Accountability\PYGZsq{}s title\PYGZsq{}
\PYGZcb{}
\end{sphinxVerbatim}

\sphinxstylestrong{Example response}:

\begin{sphinxVerbatim}[commandchars=\\\{\}]
HTTP/1.1 201 Created
Content\PYGZhy{}Type: application/json

\PYGZob{}
    \PYGZsq{}id\PYGZsq{}: 1,
    \PYGZsq{}title\PYGZsq{}: \PYGZsq{}Accountability\PYGZsq{}s title\PYGZsq{},
    \PYGZsq{}role\PYGZus{}id\PYGZsq{}: 1
\PYGZcb{}
\end{sphinxVerbatim}
\begin{quote}\begin{description}
\item[{Parameters}] \leavevmode\begin{itemize}
\item {} 
\sphinxstyleliteralstrong{role\_id} (\sphinxstyleliteralemphasis{int}) -- the role the accountability is added to

\end{itemize}

\item[{Request Headers}] \leavevmode\begin{itemize}
\item {} 
\href{http://tools.ietf.org/html/rfc7235\#section-4.2}{Authorization} -- JSON Web Token to authenticate

\item {} 
\href{http://tools.ietf.org/html/rfc7231\#section-3.1.1.5}{Content-Type} -- data is sent as application/json or
application/x-www-form-urlencoded

\end{itemize}

\item[{Request JSON Object}] \leavevmode\begin{itemize}
\item {} 
\sphinxstyleliteralstrong{title} (\sphinxstyleliteralemphasis{string}) -- the accountability's title

\end{itemize}

\item[{Response Headers}] \leavevmode\begin{itemize}
\item {} 
\href{http://tools.ietf.org/html/rfc7231\#section-3.1.1.5}{Content-Type} -- data is received as application/json

\end{itemize}

\item[{Response JSON Object}] \leavevmode\begin{itemize}
\item {} 
\sphinxstyleliteralstrong{id} (\sphinxstyleliteralemphasis{int}) -- the accountability's unique id

\item {} 
\sphinxstyleliteralstrong{title} (\sphinxstyleliteralemphasis{string}) -- the accountability's title

\item {} 
\sphinxstyleliteralstrong{role\_id} (\sphinxstyleliteralemphasis{int}) -- the role the accountability is related to

\end{itemize}

\item[{Status Codes}] \leavevmode\begin{itemize}
\item {} 
\href{http://www.w3.org/Protocols/rfc2616/rfc2616-sec10.html\#sec10.2.2}{201 Created} -- Accountability is added

\item {} 
\href{http://www.w3.org/Protocols/rfc2616/rfc2616-sec10.html\#sec10.4.1}{400 Bad Request} -- Parameters are missing

\item {} 
\href{http://www.w3.org/Protocols/rfc2616/rfc2616-sec10.html\#sec10.4.1}{400 Bad Request} -- Token is not well-formed

\item {} 
\href{http://www.w3.org/Protocols/rfc2616/rfc2616-sec10.html\#sec10.4.2}{401 Unauthorized} -- Token has expired

\item {} 
\href{http://www.w3.org/Protocols/rfc2616/rfc2616-sec10.html\#sec10.4.2}{401 Unauthorized} -- User is not authorized

\item {} 
\href{http://www.w3.org/Protocols/rfc2616/rfc2616-sec10.html\#sec10.4.5}{404 Not Found} -- Role is not found

\end{itemize}

\end{description}\end{quote}

\end{fulllineitems}



\begin{fulllineitems}
\phantomsection\label{\detokenize{resources/role:get--roles-(role_id)-accountabilities}}\pysiglinewithargsret{\sphinxbfcode{GET~}\sphinxbfcode{/roles/}}{\emph{role\_id}}{\sphinxbfcode{/accountabilities}}~
List accountabilities of a role.

\sphinxstylestrong{Example request}:

\begin{sphinxVerbatim}[commandchars=\\\{\}]
\PYG{n+nf}{GET} \PYG{n+nn}{/roles/1/accountabilities} \PYG{k+kr}{HTTP}\PYG{o}{/}\PYG{l+m}{1.1}
\PYG{n+na}{Host}\PYG{o}{:} \PYG{l}{example.com}
\PYG{n+na}{Authorization}\PYG{o}{:} \PYG{l}{Bearer \PYGZlt{}token\PYGZgt{}}
\end{sphinxVerbatim}

\sphinxstylestrong{Example response}:

\begin{sphinxVerbatim}[commandchars=\\\{\}]
HTTP/1.1 200 OK
Content\PYGZhy{}Type: application/json

[
    \PYGZob{}
        \PYGZsq{}id\PYGZsq{}: 1,
        \PYGZsq{}title\PYGZsq{}: \PYGZsq{}Accountability\PYGZsq{}s title\PYGZsq{},
        \PYGZsq{}role\PYGZus{}id\PYGZsq{}: 1
    \PYGZcb{}
]
\end{sphinxVerbatim}
\begin{quote}\begin{description}
\item[{Parameters}] \leavevmode\begin{itemize}
\item {} 
\sphinxstyleliteralstrong{role\_id} (\sphinxstyleliteralemphasis{int}) -- the role the accountabilities are listed for

\end{itemize}

\item[{Request Headers}] \leavevmode\begin{itemize}
\item {} 
\href{http://tools.ietf.org/html/rfc7235\#section-4.2}{Authorization} -- JSON Web Token to authenticate

\end{itemize}

\item[{Response Headers}] \leavevmode\begin{itemize}
\item {} 
\href{http://tools.ietf.org/html/rfc7231\#section-3.1.1.5}{Content-Type} -- data is received as application/json

\end{itemize}

\item[{Response JSON Array of Objects}] \leavevmode\begin{itemize}
\item {} 
\sphinxstyleliteralstrong{id} (\sphinxstyleliteralemphasis{int}) -- the accountability's unique id

\item {} 
\sphinxstyleliteralstrong{title} (\sphinxstyleliteralemphasis{string}) -- the accountability's title

\item {} 
\sphinxstyleliteralstrong{role\_id} (\sphinxstyleliteralemphasis{int}) -- the role the accountability is related to

\end{itemize}

\item[{Status Codes}] \leavevmode\begin{itemize}
\item {} 
\href{http://www.w3.org/Protocols/rfc2616/rfc2616-sec10.html\#sec10.2.1}{200 OK} -- Accountabilities are listed

\item {} 
\href{http://www.w3.org/Protocols/rfc2616/rfc2616-sec10.html\#sec10.4.1}{400 Bad Request} -- Token is not well-formed

\item {} 
\href{http://www.w3.org/Protocols/rfc2616/rfc2616-sec10.html\#sec10.4.2}{401 Unauthorized} -- Token has expired

\item {} 
\href{http://www.w3.org/Protocols/rfc2616/rfc2616-sec10.html\#sec10.4.2}{401 Unauthorized} -- User is not authorized

\item {} 
\href{http://www.w3.org/Protocols/rfc2616/rfc2616-sec10.html\#sec10.4.5}{404 Not Found} -- Role is not found

\end{itemize}

\end{description}\end{quote}

\end{fulllineitems}



\subsection{Circle}
\label{\detokenize{resources/role:circle}}
Converts a role to a {\hyperref[\detokenize{resources/circle:circle}]{\sphinxcrossref{\DUrole{std,std-ref}{Circle}}}} and vice versa.

\noindent\begin{tabulary}{\linewidth}{|L|L|L|}
\hline
\sphinxstylethead{\relax 
Resource
\unskip}\relax &\sphinxstylethead{\relax 
Operation
\unskip}\relax &\sphinxstylethead{\relax 
Description
\unskip}\relax \\
\hline
Role Circle
&
{\hyperref[\detokenize{resources/role:put--roles-(role_id)-circle}]{\emph{PUT /roles/(role\_id)/circle}}}
&
Add circle properties to a role.
\\
\hline&
{\hyperref[\detokenize{resources/role:delete--roles-(role_id)-circle}]{\emph{DELETE /roles/(role\_id)/circle}}}
&
Remove circle properties from a role.
\\
\hline\end{tabulary}


\begin{DUlineblock}{0em}
\item[] 
\end{DUlineblock}


\begin{fulllineitems}
\phantomsection\label{\detokenize{resources/role:put--roles-(role_id)-circle}}\pysiglinewithargsret{\sphinxbfcode{PUT~}\sphinxbfcode{/roles/}}{\emph{role\_id}}{\sphinxbfcode{/circle}}~
Add circle properties to a role.

\sphinxstylestrong{Example request}:

\begin{sphinxVerbatim}[commandchars=\\\{\}]
\PYG{n+nf}{PUT} \PYG{n+nn}{/roles/5/circle} \PYG{k+kr}{HTTP}\PYG{o}{/}\PYG{l+m}{1.1}
\PYG{n+na}{Host}\PYG{o}{:} \PYG{l}{example.com}
\PYG{n+na}{Authorization}\PYG{o}{:} \PYG{l}{Bearer \PYGZlt{}token\PYGZgt{}}
\end{sphinxVerbatim}

\sphinxstylestrong{Example response}:

\begin{sphinxVerbatim}[commandchars=\\\{\}]
\PYG{k+kr}{HTTP}\PYG{o}{/}\PYG{l+m}{1.1} \PYG{l+m}{204} \PYG{n+ne}{No Content}
\end{sphinxVerbatim}
\begin{quote}\begin{description}
\item[{Parameters}] \leavevmode\begin{itemize}
\item {} 
\sphinxstyleliteralstrong{role\_id} (\sphinxstyleliteralemphasis{int}) -- the role to add circle properties to

\end{itemize}

\item[{Request Headers}] \leavevmode\begin{itemize}
\item {} 
\href{http://tools.ietf.org/html/rfc7235\#section-4.2}{Authorization} -- JSON Web Token to authenticate

\end{itemize}

\item[{Status Codes}] \leavevmode\begin{itemize}
\item {} 
\href{http://www.w3.org/Protocols/rfc2616/rfc2616-sec10.html\#sec10.2.5}{204 No Content} -- Circle properties are added to role

\item {} 
\href{http://www.w3.org/Protocols/rfc2616/rfc2616-sec10.html\#sec10.4.1}{400 Bad Request} -- Token is not well-formed

\item {} 
\href{http://www.w3.org/Protocols/rfc2616/rfc2616-sec10.html\#sec10.4.2}{401 Unauthorized} -- Token has expired

\item {} 
\href{http://www.w3.org/Protocols/rfc2616/rfc2616-sec10.html\#sec10.4.2}{401 Unauthorized} -- User is not authorized

\item {} 
\href{http://www.w3.org/Protocols/rfc2616/rfc2616-sec10.html\#sec10.4.5}{404 Not Found} -- Role is not found

\item {} 
\href{http://www.w3.org/Protocols/rfc2616/rfc2616-sec10.html\#sec10.4.10}{409 Conflict} -- Role type is other than custom

\end{itemize}

\end{description}\end{quote}

\end{fulllineitems}



\begin{fulllineitems}
\phantomsection\label{\detokenize{resources/role:delete--roles-(role_id)-circle}}\pysiglinewithargsret{\sphinxbfcode{DELETE~}\sphinxbfcode{/roles/}}{\emph{role\_id}}{\sphinxbfcode{/circle}}~
Remove circle properties from a role.

\sphinxstylestrong{Example request}:

\begin{sphinxVerbatim}[commandchars=\\\{\}]
\PYG{n+nf}{DELETE} \PYG{n+nn}{/roles/5/circle} \PYG{k+kr}{HTTP}\PYG{o}{/}\PYG{l+m}{1.1}
\PYG{n+na}{Host}\PYG{o}{:} \PYG{l}{example.com}
\PYG{n+na}{Authorization}\PYG{o}{:} \PYG{l}{Bearer \PYGZlt{}token\PYGZgt{}}
\end{sphinxVerbatim}

\sphinxstylestrong{Example response}:

\begin{sphinxVerbatim}[commandchars=\\\{\}]
\PYG{k+kr}{HTTP}\PYG{o}{/}\PYG{l+m}{1.1} \PYG{l+m}{204} \PYG{n+ne}{No Content}
\end{sphinxVerbatim}
\begin{quote}\begin{description}
\item[{Parameters}] \leavevmode\begin{itemize}
\item {} 
\sphinxstyleliteralstrong{role\_id} (\sphinxstyleliteralemphasis{int}) -- the role to remove circle properties from

\end{itemize}

\item[{Request Headers}] \leavevmode\begin{itemize}
\item {} 
\href{http://tools.ietf.org/html/rfc7235\#section-4.2}{Authorization} -- JSON Web Token to authenticate

\end{itemize}

\item[{Status Codes}] \leavevmode\begin{itemize}
\item {} 
\href{http://www.w3.org/Protocols/rfc2616/rfc2616-sec10.html\#sec10.2.5}{204 No Content} -- Circle properties are removed from role

\item {} 
\href{http://www.w3.org/Protocols/rfc2616/rfc2616-sec10.html\#sec10.4.1}{400 Bad Request} -- Token is not well-formed

\item {} 
\href{http://www.w3.org/Protocols/rfc2616/rfc2616-sec10.html\#sec10.4.2}{401 Unauthorized} -- Token has expired

\item {} 
\href{http://www.w3.org/Protocols/rfc2616/rfc2616-sec10.html\#sec10.4.2}{401 Unauthorized} -- User is not authorized

\item {} 
\href{http://www.w3.org/Protocols/rfc2616/rfc2616-sec10.html\#sec10.4.5}{404 Not Found} -- Role is not found

\item {} 
\href{http://www.w3.org/Protocols/rfc2616/rfc2616-sec10.html\#sec10.4.10}{409 Conflict} -- Role is other than circle

\item {} 
\href{http://www.w3.org/Protocols/rfc2616/rfc2616-sec10.html\#sec10.4.10}{409 Conflict} -- Role is an anchor circle of an organization

\end{itemize}

\end{description}\end{quote}

\end{fulllineitems}



\subsection{Domains}
\label{\detokenize{resources/role:domains}}
Represents the domains of a role. See {\hyperref[\detokenize{resources/domain:domain}]{\sphinxcrossref{\DUrole{std,std-ref}{Domain}}}} for a description of a single domain.

\noindent\begin{tabulary}{\linewidth}{|L|L|L|}
\hline
\sphinxstylethead{\relax 
Resource
\unskip}\relax &\sphinxstylethead{\relax 
Operation
\unskip}\relax &\sphinxstylethead{\relax 
Description
\unskip}\relax \\
\hline
Role Domains
&
{\hyperref[\detokenize{resources/role:post--roles-(role_id)-domains}]{\emph{POST /roles/(role\_id)/domains}}}
&
Add a domain to a role.
\\
\hline&
{\hyperref[\detokenize{resources/role:get--roles-(role_id)-domains}]{\emph{GET /roles/(role\_id)/domains}}}
&
List domains of a role.
\\
\hline\end{tabulary}


\begin{DUlineblock}{0em}
\item[] 
\end{DUlineblock}


\begin{fulllineitems}
\phantomsection\label{\detokenize{resources/role:post--roles-(role_id)-domains}}\pysiglinewithargsret{\sphinxbfcode{POST~}\sphinxbfcode{/roles/}}{\emph{role\_id}}{\sphinxbfcode{/domains}}~
Add a domain to a role.

\sphinxstylestrong{Example request}:

\begin{sphinxVerbatim}[commandchars=\\\{\}]
POST /roles/1/domains HTTP/1.1
Host: example.com
Authorization: Bearer \PYGZlt{}token\PYGZgt{}
Content\PYGZhy{}Type: application/json

\PYGZob{}
    \PYGZsq{}title\PYGZsq{}: \PYGZsq{}Domain\PYGZsq{}s title\PYGZsq{}
\PYGZcb{}
\end{sphinxVerbatim}

\sphinxstylestrong{Example response}:

\begin{sphinxVerbatim}[commandchars=\\\{\}]
HTTP/1.1 201 Created
Content\PYGZhy{}Type: application/json

\PYGZob{}
    \PYGZsq{}id\PYGZsq{}: 1,
    \PYGZsq{}title\PYGZsq{}: \PYGZsq{}Domain\PYGZsq{}s title\PYGZsq{},
    \PYGZsq{}role\PYGZus{}id\PYGZsq{}: 1
\PYGZcb{}
\end{sphinxVerbatim}
\begin{quote}\begin{description}
\item[{Parameters}] \leavevmode\begin{itemize}
\item {} 
\sphinxstyleliteralstrong{role\_id} (\sphinxstyleliteralemphasis{int}) -- the role the domain is added to

\end{itemize}

\item[{Request Headers}] \leavevmode\begin{itemize}
\item {} 
\href{http://tools.ietf.org/html/rfc7235\#section-4.2}{Authorization} -- JSON Web Token to authenticate

\item {} 
\href{http://tools.ietf.org/html/rfc7231\#section-3.1.1.5}{Content-Type} -- data is sent as application/json or
application/x-www-form-urlencoded

\end{itemize}

\item[{Request JSON Object}] \leavevmode\begin{itemize}
\item {} 
\sphinxstyleliteralstrong{title} (\sphinxstyleliteralemphasis{string}) -- the domain's title

\end{itemize}

\item[{Response Headers}] \leavevmode\begin{itemize}
\item {} 
\href{http://tools.ietf.org/html/rfc7231\#section-3.1.1.5}{Content-Type} -- data is received as application/json

\end{itemize}

\item[{Response JSON Object}] \leavevmode\begin{itemize}
\item {} 
\sphinxstyleliteralstrong{id} (\sphinxstyleliteralemphasis{int}) -- the domain's unique id

\item {} 
\sphinxstyleliteralstrong{title} (\sphinxstyleliteralemphasis{string}) -- the domain's title

\item {} 
\sphinxstyleliteralstrong{role\_id} (\sphinxstyleliteralemphasis{int}) -- the role the domain is related to

\end{itemize}

\item[{Status Codes}] \leavevmode\begin{itemize}
\item {} 
\href{http://www.w3.org/Protocols/rfc2616/rfc2616-sec10.html\#sec10.2.2}{201 Created} -- Domain is added

\item {} 
\href{http://www.w3.org/Protocols/rfc2616/rfc2616-sec10.html\#sec10.4.1}{400 Bad Request} -- Parameters are missing

\item {} 
\href{http://www.w3.org/Protocols/rfc2616/rfc2616-sec10.html\#sec10.4.1}{400 Bad Request} -- Token is not well-formed

\item {} 
\href{http://www.w3.org/Protocols/rfc2616/rfc2616-sec10.html\#sec10.4.2}{401 Unauthorized} -- Token has expired

\item {} 
\href{http://www.w3.org/Protocols/rfc2616/rfc2616-sec10.html\#sec10.4.2}{401 Unauthorized} -- User is not authorized

\item {} 
\href{http://www.w3.org/Protocols/rfc2616/rfc2616-sec10.html\#sec10.4.5}{404 Not Found} -- Role is not found

\end{itemize}

\end{description}\end{quote}

\end{fulllineitems}



\begin{fulllineitems}
\phantomsection\label{\detokenize{resources/role:get--roles-(role_id)-domains}}\pysiglinewithargsret{\sphinxbfcode{GET~}\sphinxbfcode{/roles/}}{\emph{role\_id}}{\sphinxbfcode{/domains}}~
List domains of a role.

\sphinxstylestrong{Example request}:

\begin{sphinxVerbatim}[commandchars=\\\{\}]
\PYG{n+nf}{GET} \PYG{n+nn}{/roles/1/domains} \PYG{k+kr}{HTTP}\PYG{o}{/}\PYG{l+m}{1.1}
\PYG{n+na}{Host}\PYG{o}{:} \PYG{l}{example.com}
\PYG{n+na}{Authorization}\PYG{o}{:} \PYG{l}{Bearer \PYGZlt{}token\PYGZgt{}}
\end{sphinxVerbatim}

\sphinxstylestrong{Example response}:

\begin{sphinxVerbatim}[commandchars=\\\{\}]
HTTP/1.1 200 OK
Content\PYGZhy{}Type: application/json

[
    \PYGZob{}
        \PYGZsq{}id\PYGZsq{}: 1,
        \PYGZsq{}title\PYGZsq{}: \PYGZsq{}Domain\PYGZsq{}s title\PYGZsq{},
        \PYGZsq{}role\PYGZus{}id\PYGZsq{}: 1
    \PYGZcb{}
]
\end{sphinxVerbatim}
\begin{quote}\begin{description}
\item[{Parameters}] \leavevmode\begin{itemize}
\item {} 
\sphinxstyleliteralstrong{role\_id} (\sphinxstyleliteralemphasis{int}) -- the role the domains are listed for

\end{itemize}

\item[{Request Headers}] \leavevmode\begin{itemize}
\item {} 
\href{http://tools.ietf.org/html/rfc7235\#section-4.2}{Authorization} -- JSON Web Token to authenticate

\end{itemize}

\item[{Response Headers}] \leavevmode\begin{itemize}
\item {} 
\href{http://tools.ietf.org/html/rfc7231\#section-3.1.1.5}{Content-Type} -- data is received as application/json

\end{itemize}

\item[{Response JSON Array of Objects}] \leavevmode\begin{itemize}
\item {} 
\sphinxstyleliteralstrong{id} (\sphinxstyleliteralemphasis{int}) -- the domain's unique id

\item {} 
\sphinxstyleliteralstrong{title} (\sphinxstyleliteralemphasis{string}) -- the domain's title

\item {} 
\sphinxstyleliteralstrong{role\_id} (\sphinxstyleliteralemphasis{int}) -- the role the domain is related to

\end{itemize}

\item[{Status Codes}] \leavevmode\begin{itemize}
\item {} 
\href{http://www.w3.org/Protocols/rfc2616/rfc2616-sec10.html\#sec10.2.1}{200 OK} -- Domains are listed

\item {} 
\href{http://www.w3.org/Protocols/rfc2616/rfc2616-sec10.html\#sec10.4.1}{400 Bad Request} -- Token is not well-formed

\item {} 
\href{http://www.w3.org/Protocols/rfc2616/rfc2616-sec10.html\#sec10.4.2}{401 Unauthorized} -- Token has expired

\item {} 
\href{http://www.w3.org/Protocols/rfc2616/rfc2616-sec10.html\#sec10.4.2}{401 Unauthorized} -- User is not authorized

\item {} 
\href{http://www.w3.org/Protocols/rfc2616/rfc2616-sec10.html\#sec10.4.5}{404 Not Found} -- Role is not found

\end{itemize}

\end{description}\end{quote}

\end{fulllineitems}



\subsection{Members}
\label{\detokenize{resources/role:members}}
Represents the members of a role. See {\hyperref[\detokenize{resources/partner:partner}]{\sphinxcrossref{\DUrole{std,std-ref}{Partner}}}} for a description of a single member.

\noindent\begin{tabulary}{\linewidth}{|L|L|L|}
\hline
\sphinxstylethead{\relax 
Resource
\unskip}\relax &\sphinxstylethead{\relax 
Operation
\unskip}\relax &\sphinxstylethead{\relax 
Description
\unskip}\relax \\
\hline
Role Members
&
{\hyperref[\detokenize{resources/role:get--roles-(role_id)-members}]{\emph{GET /roles/(role\_id)/members}}}
&
List members of a role.
\\
\hline&
{\hyperref[\detokenize{resources/role:put--roles-(role_id)-members-(partner_id)}]{\emph{PUT /roles/(role\_id)/members/(partner\_id)}}}
&
Assign a partner to a role.
\\
\hline&
{\hyperref[\detokenize{resources/role:delete--roles-(role_id)-members-(partner_id)}]{\emph{DELETE /roles/(role\_id)/members/(partner\_id)}}}
&
Unassign a partner from a role.
\\
\hline\end{tabulary}


\begin{DUlineblock}{0em}
\item[] 
\end{DUlineblock}


\begin{fulllineitems}
\phantomsection\label{\detokenize{resources/role:get--roles-(role_id)-members}}\pysiglinewithargsret{\sphinxbfcode{GET~}\sphinxbfcode{/roles/}}{\emph{role\_id}}{\sphinxbfcode{/members}}~
List members of a role.

\sphinxstylestrong{Example request}:

\begin{sphinxVerbatim}[commandchars=\\\{\}]
\PYG{n+nf}{GET} \PYG{n+nn}{/roles/1/members} \PYG{k+kr}{HTTP}\PYG{o}{/}\PYG{l+m}{1.1}
\PYG{n+na}{Host}\PYG{o}{:} \PYG{l}{example.com}
\PYG{n+na}{Authorization}\PYG{o}{:} \PYG{l}{Bearer \PYGZlt{}token\PYGZgt{}}
\end{sphinxVerbatim}

\sphinxstylestrong{Example response}:

\begin{sphinxVerbatim}[commandchars=\\\{\}]
HTTP/1.1 200 OK
Content\PYGZhy{}Type: application/json

[
    \PYGZob{}
        \PYGZsq{}id\PYGZsq{}: 1,
        \PYGZsq{}type\PYGZsq{}: \PYGZsq{}admin\PYGZsq{},
        \PYGZsq{}firstname\PYGZsq{}: \PYGZsq{}John\PYGZsq{},
        \PYGZsq{}lastname\PYGZsq{}: \PYGZsq{}Doe\PYGZsq{},
        \PYGZsq{}email\PYGZsq{}: \PYGZsq{}john@example.org\PYGZsq{},
        \PYGZsq{}is\PYGZus{}active\PYGZsq{}: True,
        \PYGZsq{}user\PYGZus{}id\PYGZsq{}: 1,
        \PYGZsq{}organization\PYGZus{}id\PYGZsq{}: 1,
        \PYGZsq{}invitation\PYGZus{}id\PYGZsq{}: null
    \PYGZcb{}
]
\end{sphinxVerbatim}
\begin{quote}\begin{description}
\item[{Parameters}] \leavevmode\begin{itemize}
\item {} 
\sphinxstyleliteralstrong{role\_id} (\sphinxstyleliteralemphasis{int}) -- the role the members are listed for

\end{itemize}

\item[{Request Headers}] \leavevmode\begin{itemize}
\item {} 
\href{http://tools.ietf.org/html/rfc7235\#section-4.2}{Authorization} -- JSON Web Token to authenticate

\end{itemize}

\item[{Response Headers}] \leavevmode\begin{itemize}
\item {} 
\href{http://tools.ietf.org/html/rfc7231\#section-3.1.1.5}{Content-Type} -- data is received as application/json

\end{itemize}

\item[{Response JSON Array of Objects}] \leavevmode\begin{itemize}
\item {} 
\sphinxstyleliteralstrong{id} (\sphinxstyleliteralemphasis{int}) -- the partner's unique id

\item {} 
\sphinxstyleliteralstrong{type} (\sphinxstyleliteralemphasis{string}) -- the partner's type

\item {} 
\sphinxstyleliteralstrong{firstname} (\sphinxstyleliteralemphasis{string}) -- the partner's firstname

\item {} 
\sphinxstyleliteralstrong{lastname} (\sphinxstyleliteralemphasis{string}) -- the partner's lastname

\item {} 
\sphinxstyleliteralstrong{email} (\sphinxstyleliteralemphasis{string}) -- the partner's email address

\item {} 
\sphinxstyleliteralstrong{is\_active} (\sphinxstyleliteralemphasis{boolean}) -- the partner's status

\item {} 
\sphinxstyleliteralstrong{user\_id} (\sphinxstyleliteralemphasis{int}) -- the user account the partner is related to

\item {} 
\sphinxstyleliteralstrong{organization\_id} (\sphinxstyleliteralemphasis{int}) -- the organization the partner is
related to

\item {} 
\sphinxstyleliteralstrong{invitation\_id} (\sphinxstyleliteralemphasis{int}) -- the invitation the partner is related to

\end{itemize}

\item[{Status Codes}] \leavevmode\begin{itemize}
\item {} 
\href{http://www.w3.org/Protocols/rfc2616/rfc2616-sec10.html\#sec10.2.1}{200 OK} -- Members are listed

\item {} 
\href{http://www.w3.org/Protocols/rfc2616/rfc2616-sec10.html\#sec10.4.1}{400 Bad Request} -- Token is not well-formed

\item {} 
\href{http://www.w3.org/Protocols/rfc2616/rfc2616-sec10.html\#sec10.4.2}{401 Unauthorized} -- Token has expired

\item {} 
\href{http://www.w3.org/Protocols/rfc2616/rfc2616-sec10.html\#sec10.4.2}{401 Unauthorized} -- User is not authorized

\item {} 
\href{http://www.w3.org/Protocols/rfc2616/rfc2616-sec10.html\#sec10.4.5}{404 Not Found} -- Role is not found

\end{itemize}

\end{description}\end{quote}

\end{fulllineitems}



\begin{fulllineitems}
\phantomsection\label{\detokenize{resources/role:put--roles-(role_id)-members-(partner_id)}}\pysiglinewithargsret{\sphinxbfcode{PUT~}\sphinxbfcode{/roles/}}{\emph{role\_id}}{\sphinxbfcode{/members/}}{\emph{partner\_id}}{}~
Assign a partner to a role.

\sphinxstylestrong{Example request}:

\begin{sphinxVerbatim}[commandchars=\\\{\}]
\PYG{n+nf}{PUT} \PYG{n+nn}{/roles/5/members/1} \PYG{k+kr}{HTTP}\PYG{o}{/}\PYG{l+m}{1.1}
\PYG{n+na}{Host}\PYG{o}{:} \PYG{l}{example.com}
\PYG{n+na}{Authorization}\PYG{o}{:} \PYG{l}{Bearer \PYGZlt{}token\PYGZgt{}}
\end{sphinxVerbatim}

\sphinxstylestrong{Example response}:

\begin{sphinxVerbatim}[commandchars=\\\{\}]
\PYG{k+kr}{HTTP}\PYG{o}{/}\PYG{l+m}{1.1} \PYG{l+m}{204} \PYG{n+ne}{No Content}
\end{sphinxVerbatim}
\begin{quote}\begin{description}
\item[{Parameters}] \leavevmode\begin{itemize}
\item {} 
\sphinxstyleliteralstrong{role\_id} (\sphinxstyleliteralemphasis{int}) -- the role the partner is assigned to

\item {} 
\sphinxstyleliteralstrong{partner\_id} (\sphinxstyleliteralemphasis{int}) -- the partner who is assigned to the role

\end{itemize}

\item[{Request Headers}] \leavevmode\begin{itemize}
\item {} 
\href{http://tools.ietf.org/html/rfc7235\#section-4.2}{Authorization} -- JSON Web Token to authenticate

\end{itemize}

\item[{Status Codes}] \leavevmode\begin{itemize}
\item {} 
\href{http://www.w3.org/Protocols/rfc2616/rfc2616-sec10.html\#sec10.2.5}{204 No Content} -- Partner is assigned to role

\item {} 
\href{http://www.w3.org/Protocols/rfc2616/rfc2616-sec10.html\#sec10.4.1}{400 Bad Request} -- Token is not well-formed

\item {} 
\href{http://www.w3.org/Protocols/rfc2616/rfc2616-sec10.html\#sec10.4.2}{401 Unauthorized} -- Token has expired

\item {} 
\href{http://www.w3.org/Protocols/rfc2616/rfc2616-sec10.html\#sec10.4.2}{401 Unauthorized} -- User is not authorized

\item {} 
\href{http://www.w3.org/Protocols/rfc2616/rfc2616-sec10.html\#sec10.4.5}{404 Not Found} -- Role is not found

\item {} 
\href{http://www.w3.org/Protocols/rfc2616/rfc2616-sec10.html\#sec10.4.5}{404 Not Found} -- Partner is not found

\item {} 
\href{http://www.w3.org/Protocols/rfc2616/rfc2616-sec10.html\#sec10.4.10}{409 Conflict} -- Role is not associated with partner's organization

\end{itemize}

\end{description}\end{quote}

\end{fulllineitems}



\begin{fulllineitems}
\phantomsection\label{\detokenize{resources/role:delete--roles-(role_id)-members-(partner_id)}}\pysiglinewithargsret{\sphinxbfcode{DELETE~}\sphinxbfcode{/roles/}}{\emph{role\_id}}{\sphinxbfcode{/members/}}{\emph{partner\_id}}{}~
Unassign a partner from a role.

\sphinxstylestrong{Example request}:

\begin{sphinxVerbatim}[commandchars=\\\{\}]
\PYG{n+nf}{DELETE} \PYG{n+nn}{/roles/5/members/1} \PYG{k+kr}{HTTP}\PYG{o}{/}\PYG{l+m}{1.1}
\PYG{n+na}{Host}\PYG{o}{:} \PYG{l}{example.com}
\PYG{n+na}{Authorization}\PYG{o}{:} \PYG{l}{Bearer \PYGZlt{}token\PYGZgt{}}
\end{sphinxVerbatim}

\sphinxstylestrong{Example response}:

\begin{sphinxVerbatim}[commandchars=\\\{\}]
\PYG{k+kr}{HTTP}\PYG{o}{/}\PYG{l+m}{1.1} \PYG{l+m}{204} \PYG{n+ne}{No Content}
\end{sphinxVerbatim}
\begin{quote}\begin{description}
\item[{Parameters}] \leavevmode\begin{itemize}
\item {} 
\sphinxstyleliteralstrong{role\_id} (\sphinxstyleliteralemphasis{int}) -- the role the partner is unassigned from

\item {} 
\sphinxstyleliteralstrong{partner\_id} (\sphinxstyleliteralemphasis{int}) -- the partner who is unassigned from the role

\end{itemize}

\item[{Request Headers}] \leavevmode\begin{itemize}
\item {} 
\href{http://tools.ietf.org/html/rfc7235\#section-4.2}{Authorization} -- JSON Web Token to authenticate

\end{itemize}

\item[{Status Codes}] \leavevmode\begin{itemize}
\item {} 
\href{http://www.w3.org/Protocols/rfc2616/rfc2616-sec10.html\#sec10.2.5}{204 No Content} -- Partner is unassigned from role

\item {} 
\href{http://www.w3.org/Protocols/rfc2616/rfc2616-sec10.html\#sec10.4.1}{400 Bad Request} -- Token is not well-formed

\item {} 
\href{http://www.w3.org/Protocols/rfc2616/rfc2616-sec10.html\#sec10.4.2}{401 Unauthorized} -- Token has expired

\item {} 
\href{http://www.w3.org/Protocols/rfc2616/rfc2616-sec10.html\#sec10.4.2}{401 Unauthorized} -- User is not authorized

\item {} 
\href{http://www.w3.org/Protocols/rfc2616/rfc2616-sec10.html\#sec10.4.5}{404 Not Found} -- Role is not found

\item {} 
\href{http://www.w3.org/Protocols/rfc2616/rfc2616-sec10.html\#sec10.4.5}{404 Not Found} -- Partner is not found

\end{itemize}

\end{description}\end{quote}

\end{fulllineitems}



\section{User}
\label{\detokenize{resources/user:user}}\label{\detokenize{resources/user::doc}}\label{\detokenize{resources/user:id1}}
Represents a user. A user is related to an {\hyperref[\detokenize{resources/organization:organization}]{\sphinxcrossref{\DUrole{std,std-ref}{Organization}}}} through a {\hyperref[\detokenize{resources/partner:partner}]{\sphinxcrossref{\DUrole{std,std-ref}{Partner}}}}.

\noindent\begin{tabulary}{\linewidth}{|L|L|L|}
\hline
\sphinxstylethead{\relax 
Resource
\unskip}\relax &\sphinxstylethead{\relax 
Operation
\unskip}\relax &\sphinxstylethead{\relax 
Description
\unskip}\relax \\
\hline
User
&
{\hyperref[\detokenize{resources/user:put--me}]{\emph{PUT /me}}}
&
Update the authenticated user.
\\
\hline&
{\hyperref[\detokenize{resources/user:delete--me}]{\emph{DELETE /me}}}
&
Delete the authenticated user.
\\
\hline&
{\hyperref[\detokenize{resources/user:get--me}]{\emph{GET /me}}}
&
Retrieve the authenticated user.
\\
\hline\end{tabulary}


\begin{DUlineblock}{0em}
\item[] 
\end{DUlineblock}


\begin{fulllineitems}
\phantomsection\label{\detokenize{resources/user:put--me}}\pysigline{\sphinxbfcode{PUT~}\sphinxbfcode{/me}}~
Update the authenticated user.

\sphinxstylestrong{Example request}:

\begin{sphinxVerbatim}[commandchars=\\\{\}]
PUT /me HTTP/1.1
Host: example.com
Authorization: Bearer \PYGZlt{}token\PYGZgt{}
Content\PYGZhy{}Type: application/json

\PYGZob{}
    \PYGZsq{}firstname\PYGZsq{}: \PYGZsq{}John\PYGZsq{},
    \PYGZsq{}lastname\PYGZsq{}: \PYGZsq{}Doe\PYGZsq{},
    \PYGZsq{}email\PYGZsq{}: \PYGZsq{}john@example.org\PYGZsq{}
\PYGZcb{}
\end{sphinxVerbatim}

\sphinxstylestrong{Example response}:

\begin{sphinxVerbatim}[commandchars=\\\{\}]
HTTP/1.1 200 OK
Content\PYGZhy{}Type: application/json

\PYGZob{}
    \PYGZsq{}id\PYGZsq{}: 1,
    \PYGZsq{}google\PYGZus{}id\PYGZsq{}: \PYGZsq{}123456789\PYGZsq{},
    \PYGZsq{}firstname\PYGZsq{}: \PYGZsq{}John\PYGZsq{},
    \PYGZsq{}lastname\PYGZsq{}: \PYGZsq{}Doe\PYGZsq{},
    \PYGZsq{}email\PYGZsq{}: \PYGZsq{}john@example.org\PYGZsq{},
    \PYGZsq{}is\PYGZus{}active\PYGZsq{}: True
\PYGZcb{}
\end{sphinxVerbatim}
\begin{quote}\begin{description}
\item[{Request Headers}] \leavevmode\begin{itemize}
\item {} 
\href{http://tools.ietf.org/html/rfc7235\#section-4.2}{Authorization} -- JSON Web Token to authenticate

\item {} 
\href{http://tools.ietf.org/html/rfc7231\#section-3.1.1.5}{Content-Type} -- data is sent as application/json or
application/x-www-form-urlencoded

\end{itemize}

\item[{Request JSON Object}] \leavevmode\begin{itemize}
\item {} 
\sphinxstyleliteralstrong{firstname} (\sphinxstyleliteralemphasis{string}) -- the user's firstname

\item {} 
\sphinxstyleliteralstrong{lastname} (\sphinxstyleliteralemphasis{string}) -- the user's lastname

\item {} 
\sphinxstyleliteralstrong{email} (\sphinxstyleliteralemphasis{string}) -- the user's email

\end{itemize}

\item[{Response Headers}] \leavevmode\begin{itemize}
\item {} 
\href{http://tools.ietf.org/html/rfc7231\#section-3.1.1.5}{Content-Type} -- data is received as application/json

\end{itemize}

\item[{Response JSON Object}] \leavevmode\begin{itemize}
\item {} 
\sphinxstyleliteralstrong{id} (\sphinxstyleliteralemphasis{int}) -- the user's id

\item {} 
\sphinxstyleliteralstrong{google\_id} (\sphinxstyleliteralemphasis{string}) -- the user's google id

\item {} 
\sphinxstyleliteralstrong{firstname} (\sphinxstyleliteralemphasis{string}) -- the user's firstname

\item {} 
\sphinxstyleliteralstrong{lastname} (\sphinxstyleliteralemphasis{string}) -- the user's lastname

\item {} 
\sphinxstyleliteralstrong{email} (\sphinxstyleliteralemphasis{string}) -- the user's email

\item {} 
\sphinxstyleliteralstrong{is\_active} (\sphinxstyleliteralemphasis{boolean}) -- the user's status

\end{itemize}

\item[{Status Codes}] \leavevmode\begin{itemize}
\item {} 
\href{http://www.w3.org/Protocols/rfc2616/rfc2616-sec10.html\#sec10.2.1}{200 OK} -- User is updated

\item {} 
\href{http://www.w3.org/Protocols/rfc2616/rfc2616-sec10.html\#sec10.4.1}{400 Bad Request} -- Parameters are missing

\item {} 
\href{http://www.w3.org/Protocols/rfc2616/rfc2616-sec10.html\#sec10.4.1}{400 Bad Request} -- Token is not well-formed

\item {} 
\href{http://www.w3.org/Protocols/rfc2616/rfc2616-sec10.html\#sec10.4.2}{401 Unauthorized} -- Token has expired

\item {} 
\href{http://www.w3.org/Protocols/rfc2616/rfc2616-sec10.html\#sec10.4.2}{401 Unauthorized} -- User is not authorized

\end{itemize}

\end{description}\end{quote}

\end{fulllineitems}



\begin{fulllineitems}
\phantomsection\label{\detokenize{resources/user:delete--me}}\pysigline{\sphinxbfcode{DELETE~}\sphinxbfcode{/me}}~
Delete the authenticated user.

This endpoint sets the authenticated user's account and partnerships
with organizations to `inactive'. By signin-up again with the same
google account, the user's account is reactivated. To rejoin an
organization, a new invitation is needed.

\sphinxstylestrong{Example request}:

\begin{sphinxVerbatim}[commandchars=\\\{\}]
\PYG{n+nf}{DELETE} \PYG{n+nn}{/me} \PYG{k+kr}{HTTP}\PYG{o}{/}\PYG{l+m}{1.1}
\PYG{n+na}{Host}\PYG{o}{:} \PYG{l}{example.com}
\PYG{n+na}{Authorization}\PYG{o}{:} \PYG{l}{Bearer \PYGZlt{}token\PYGZgt{}}
\end{sphinxVerbatim}

\sphinxstylestrong{Example response}:

\begin{sphinxVerbatim}[commandchars=\\\{\}]
\PYG{k+kr}{HTTP}\PYG{o}{/}\PYG{l+m}{1.1} \PYG{l+m}{204} \PYG{n+ne}{No Content}
\end{sphinxVerbatim}
\begin{quote}\begin{description}
\item[{Request Headers}] \leavevmode\begin{itemize}
\item {} 
\href{http://tools.ietf.org/html/rfc7235\#section-4.2}{Authorization} -- JSON Web Token to authenticate

\end{itemize}

\item[{Status Codes}] \leavevmode\begin{itemize}
\item {} 
\href{http://www.w3.org/Protocols/rfc2616/rfc2616-sec10.html\#sec10.2.5}{204 No Content} -- User is deleted

\item {} 
\href{http://www.w3.org/Protocols/rfc2616/rfc2616-sec10.html\#sec10.4.1}{400 Bad Request} -- Token is not well-formed

\item {} 
\href{http://www.w3.org/Protocols/rfc2616/rfc2616-sec10.html\#sec10.4.2}{401 Unauthorized} -- Token has expired

\item {} 
\href{http://www.w3.org/Protocols/rfc2616/rfc2616-sec10.html\#sec10.4.2}{401 Unauthorized} -- User is not authorized

\end{itemize}

\end{description}\end{quote}

\end{fulllineitems}



\begin{fulllineitems}
\phantomsection\label{\detokenize{resources/user:get--me}}\pysigline{\sphinxbfcode{GET~}\sphinxbfcode{/me}}~
Retrieve the authenticated user.

\sphinxstylestrong{Example request}:

\begin{sphinxVerbatim}[commandchars=\\\{\}]
\PYG{n+nf}{GET} \PYG{n+nn}{/me} \PYG{k+kr}{HTTP}\PYG{o}{/}\PYG{l+m}{1.1}
\PYG{n+na}{Host}\PYG{o}{:} \PYG{l}{example.com}
\PYG{n+na}{Authorization}\PYG{o}{:} \PYG{l}{Bearer \PYGZlt{}token\PYGZgt{}}
\end{sphinxVerbatim}

\sphinxstylestrong{Example response}:

\begin{sphinxVerbatim}[commandchars=\\\{\}]
HTTP/1.1 200 OK
Content\PYGZhy{}Type: application/json

\PYGZob{}
    \PYGZsq{}id\PYGZsq{}: 1,
    \PYGZsq{}google\PYGZus{}id\PYGZsq{}: \PYGZsq{}123456789\PYGZsq{},
    \PYGZsq{}firstname\PYGZsq{}: \PYGZsq{}John\PYGZsq{},
    \PYGZsq{}lastname\PYGZsq{}: \PYGZsq{}Doe\PYGZsq{},
    \PYGZsq{}email\PYGZsq{}: \PYGZsq{}john@example.org\PYGZsq{},
    \PYGZsq{}is\PYGZus{}active\PYGZsq{}: True
\PYGZcb{}
\end{sphinxVerbatim}
\begin{quote}\begin{description}
\item[{Request Headers}] \leavevmode\begin{itemize}
\item {} 
\href{http://tools.ietf.org/html/rfc7235\#section-4.2}{Authorization} -- JSON Web Token to authenticate

\end{itemize}

\item[{Response Headers}] \leavevmode\begin{itemize}
\item {} 
\href{http://tools.ietf.org/html/rfc7231\#section-3.1.1.5}{Content-Type} -- data is received as application/json

\end{itemize}

\item[{Response JSON Object}] \leavevmode\begin{itemize}
\item {} 
\sphinxstyleliteralstrong{id} (\sphinxstyleliteralemphasis{int}) -- the user's id

\item {} 
\sphinxstyleliteralstrong{google\_id} (\sphinxstyleliteralemphasis{string}) -- the user's google id

\item {} 
\sphinxstyleliteralstrong{firstname} (\sphinxstyleliteralemphasis{string}) -- the user's firstname

\item {} 
\sphinxstyleliteralstrong{lastname} (\sphinxstyleliteralemphasis{string}) -- the user's lastname

\item {} 
\sphinxstyleliteralstrong{email} (\sphinxstyleliteralemphasis{string}) -- the user's email

\item {} 
\sphinxstyleliteralstrong{is\_active} (\sphinxstyleliteralemphasis{boolean}) -- the user's status

\end{itemize}

\item[{Status Codes}] \leavevmode\begin{itemize}
\item {} 
\href{http://www.w3.org/Protocols/rfc2616/rfc2616-sec10.html\#sec10.2.1}{200 OK} -- User is retrieved

\item {} 
\href{http://www.w3.org/Protocols/rfc2616/rfc2616-sec10.html\#sec10.4.1}{400 Bad Request} -- Token is not well-formed

\item {} 
\href{http://www.w3.org/Protocols/rfc2616/rfc2616-sec10.html\#sec10.4.2}{401 Unauthorized} -- Token has expired

\item {} 
\href{http://www.w3.org/Protocols/rfc2616/rfc2616-sec10.html\#sec10.4.2}{401 Unauthorized} -- User is not authorized

\end{itemize}

\end{description}\end{quote}

\end{fulllineitems}



\subsection{Organizations}
\label{\detokenize{resources/user:organizations}}
Represents the organizations of the authenticated user. See {\hyperref[\detokenize{resources/organization:organization}]{\sphinxcrossref{\DUrole{std,std-ref}{Organization}}}} for a description of a single organization.

\noindent\begin{tabulary}{\linewidth}{|L|L|L|}
\hline
\sphinxstylethead{\relax 
Resource
\unskip}\relax &\sphinxstylethead{\relax 
Operation
\unskip}\relax &\sphinxstylethead{\relax 
Description
\unskip}\relax \\
\hline
User Organizations
&
{\hyperref[\detokenize{resources/user:post--me-organizations}]{\emph{POST /me/organizations}}}
&
Create an organization.
\\
\hline&
{\hyperref[\detokenize{resources/user:get--me-organizations}]{\emph{GET /me/organizations}}}
&
List the user's organizations.
\\
\hline\end{tabulary}


\begin{DUlineblock}{0em}
\item[] 
\end{DUlineblock}


\begin{fulllineitems}
\phantomsection\label{\detokenize{resources/user:post--me-organizations}}\pysigline{\sphinxbfcode{POST~}\sphinxbfcode{/me/organizations}}~
Create an organization.

This endpoint creates a new organization with an anchor circle and
adds the authenticated user as an admin to the organization.

\sphinxstylestrong{Example request}:

\begin{sphinxVerbatim}[commandchars=\\\{\}]
POST /me/organizations HTTP/1.1
Host: example.com
Authorization: Bearer \PYGZlt{}token\PYGZgt{}
Content\PYGZhy{}Type: application/json

\PYGZob{}
    \PYGZsq{}name\PYGZsq{}: \PYGZsq{}My Organization\PYGZsq{}
\PYGZcb{}
\end{sphinxVerbatim}

\sphinxstylestrong{Example response}:

\begin{sphinxVerbatim}[commandchars=\\\{\}]
HTTP/1.1 201 Created
Content\PYGZhy{}Type: application/json

\PYGZob{}
    \PYGZsq{}id\PYGZsq{}: 1,
    \PYGZsq{}name\PYGZsq{}: \PYGZsq{}My Organization\PYGZsq{}
\PYGZcb{}
\end{sphinxVerbatim}
\begin{quote}\begin{description}
\item[{Request Headers}] \leavevmode\begin{itemize}
\item {} 
\href{http://tools.ietf.org/html/rfc7235\#section-4.2}{Authorization} -- JSON Web Token to authenticate

\item {} 
\href{http://tools.ietf.org/html/rfc7231\#section-3.1.1.5}{Content-Type} -- data is sent as application/json or
application/x-www-form-urlencoded

\end{itemize}

\item[{Request JSON Object}] \leavevmode\begin{itemize}
\item {} 
\sphinxstyleliteralstrong{name} (\sphinxstyleliteralemphasis{string}) -- the organization's name

\end{itemize}

\item[{Response Headers}] \leavevmode\begin{itemize}
\item {} 
\href{http://tools.ietf.org/html/rfc7231\#section-3.1.1.5}{Content-Type} -- data is received as application/json

\end{itemize}

\item[{Response JSON Object}] \leavevmode\begin{itemize}
\item {} 
\sphinxstyleliteralstrong{id} (\sphinxstyleliteralemphasis{int}) -- the organization's id

\item {} 
\sphinxstyleliteralstrong{name} (\sphinxstyleliteralemphasis{string}) -- the organization's name

\end{itemize}

\item[{Status Codes}] \leavevmode\begin{itemize}
\item {} 
\href{http://www.w3.org/Protocols/rfc2616/rfc2616-sec10.html\#sec10.2.2}{201 Created} -- Organization is created

\item {} 
\href{http://www.w3.org/Protocols/rfc2616/rfc2616-sec10.html\#sec10.4.1}{400 Bad Request} -- Parameters are missing

\item {} 
\href{http://www.w3.org/Protocols/rfc2616/rfc2616-sec10.html\#sec10.4.1}{400 Bad Request} -- Token is not well-formed

\item {} 
\href{http://www.w3.org/Protocols/rfc2616/rfc2616-sec10.html\#sec10.4.2}{401 Unauthorized} -- Token has expired

\item {} 
\href{http://www.w3.org/Protocols/rfc2616/rfc2616-sec10.html\#sec10.4.2}{401 Unauthorized} -- User is not authorized

\item {} 
\href{http://www.w3.org/Protocols/rfc2616/rfc2616-sec10.html\#sec10.4.10}{409 Conflict} -- Organization cannot be created

\end{itemize}

\end{description}\end{quote}

\end{fulllineitems}



\begin{fulllineitems}
\phantomsection\label{\detokenize{resources/user:get--me-organizations}}\pysigline{\sphinxbfcode{GET~}\sphinxbfcode{/me/organizations}}~
List organizations for the authenticated user.

This endpoint only lists organizations that the authenticated user is
allowed to operate on as a member or an admin.

\sphinxstylestrong{Example request}:

\begin{sphinxVerbatim}[commandchars=\\\{\}]
\PYG{n+nf}{GET} \PYG{n+nn}{/me/organizations} \PYG{k+kr}{HTTP}\PYG{o}{/}\PYG{l+m}{1.1}
\PYG{n+na}{Host}\PYG{o}{:} \PYG{l}{example.com}
\PYG{n+na}{Authorization}\PYG{o}{:} \PYG{l}{Bearer \PYGZlt{}token\PYGZgt{}}
\end{sphinxVerbatim}

\sphinxstylestrong{Example response}:

\begin{sphinxVerbatim}[commandchars=\\\{\}]
HTTP/1.1 200 OK
Content\PYGZhy{}Type: application/json

[
    \PYGZob{}
        \PYGZsq{}id\PYGZsq{}: 1,
        \PYGZsq{}name\PYGZsq{}: \PYGZsq{}My Organization\PYGZsq{}
    \PYGZcb{}
]
\end{sphinxVerbatim}
\begin{quote}\begin{description}
\item[{Request Headers}] \leavevmode\begin{itemize}
\item {} 
\href{http://tools.ietf.org/html/rfc7235\#section-4.2}{Authorization} -- JSON Web Token to authenticate

\end{itemize}

\item[{Response Headers}] \leavevmode\begin{itemize}
\item {} 
\href{http://tools.ietf.org/html/rfc7231\#section-3.1.1.5}{Content-Type} -- data is received as application/json

\end{itemize}

\item[{Response JSON Array of Objects}] \leavevmode\begin{itemize}
\item {} 
\sphinxstyleliteralstrong{id} (\sphinxstyleliteralemphasis{int}) -- the organization's id

\item {} 
\sphinxstyleliteralstrong{name} (\sphinxstyleliteralemphasis{string}) -- the organization's name

\end{itemize}

\item[{Status Codes}] \leavevmode\begin{itemize}
\item {} 
\href{http://www.w3.org/Protocols/rfc2616/rfc2616-sec10.html\#sec10.2.1}{200 OK} -- Organizations are listed

\item {} 
\href{http://www.w3.org/Protocols/rfc2616/rfc2616-sec10.html\#sec10.4.1}{400 Bad Request} -- Token is not well-formed

\item {} 
\href{http://www.w3.org/Protocols/rfc2616/rfc2616-sec10.html\#sec10.4.2}{401 Unauthorized} -- Token has expired

\item {} 
\href{http://www.w3.org/Protocols/rfc2616/rfc2616-sec10.html\#sec10.4.2}{401 Unauthorized} -- User is not authorized

\end{itemize}

\end{description}\end{quote}

\end{fulllineitems}



\renewcommand{\indexname}{HTTP Routing Table}
\begin{sphinxtheindex}
\def\bigletter#1{{\Large\sffamily#1}\nopagebreak\vspace{1mm}}
\bigletter{/accountabilities}
\item {\sphinxstyleindexentry{GET /accountabilities/(accountability\_id)}}\sphinxstyleindexpageref{resources/accountability:\detokenize{get--accountabilities-(accountability_id)}}
\item {\sphinxstyleindexentry{PUT /accountabilities/(accountability\_id)}}\sphinxstyleindexpageref{resources/accountability:\detokenize{put--accountabilities-(accountability_id)}}
\item {\sphinxstyleindexentry{DELETE /accountabilities/(accountability\_id)}}\sphinxstyleindexpageref{resources/accountability:\detokenize{delete--accountabilities-(accountability_id)}}
\indexspace
\bigletter{/circles}
\item {\sphinxstyleindexentry{GET /circles/(circle\_id)}}\sphinxstyleindexpageref{resources/circle:\detokenize{get--circles-(circle_id)}}
\item {\sphinxstyleindexentry{GET /circles/(circle\_id)/members}}\sphinxstyleindexpageref{resources/circle:\detokenize{get--circles-(circle_id)-members}}
\item {\sphinxstyleindexentry{GET /circles/(circle\_id)/roles}}\sphinxstyleindexpageref{resources/circle:\detokenize{get--circles-(circle_id)-roles}}
\item {\sphinxstyleindexentry{POST /circles/(circle\_id)/roles}}\sphinxstyleindexpageref{resources/circle:\detokenize{post--circles-(circle_id)-roles}}
\item {\sphinxstyleindexentry{PUT /circles/(circle\_id)}}\sphinxstyleindexpageref{resources/circle:\detokenize{put--circles-(circle_id)}}
\item {\sphinxstyleindexentry{PUT /circles/(circle\_id)/members/(partner\_id)}}\sphinxstyleindexpageref{resources/circle:\detokenize{put--circles-(circle_id)-members-(partner_id)}}
\item {\sphinxstyleindexentry{DELETE /circles/(circle\_id)/members/(partner\_id)}}\sphinxstyleindexpageref{resources/circle:\detokenize{delete--circles-(circle_id)-members-(partner_id)}}
\indexspace
\bigletter{/domains}
\item {\sphinxstyleindexentry{GET /domains/(domain\_id)}}\sphinxstyleindexpageref{resources/domain:\detokenize{get--domains-(domain_id)}}
\item {\sphinxstyleindexentry{GET /domains/(domain\_id)/policies}}\sphinxstyleindexpageref{resources/domain:\detokenize{get--domains-(domain_id)-policies}}
\item {\sphinxstyleindexentry{POST /domains/(domain\_id)/policies}}\sphinxstyleindexpageref{resources/domain:\detokenize{post--domains-(domain_id)-policies}}
\item {\sphinxstyleindexentry{PUT /domains/(domain\_id)}}\sphinxstyleindexpageref{resources/domain:\detokenize{put--domains-(domain_id)}}
\item {\sphinxstyleindexentry{DELETE /domains/(domain\_id)}}\sphinxstyleindexpageref{resources/domain:\detokenize{delete--domains-(domain_id)}}
\indexspace
\bigletter{/invitations}
\item {\sphinxstyleindexentry{GET /invitations/(code)/accept}}\sphinxstyleindexpageref{resources/invitation:\detokenize{get--invitations-(code)-accept}}
\item {\sphinxstyleindexentry{GET /invitations/(invitation\_id)}}\sphinxstyleindexpageref{resources/invitation:\detokenize{get--invitations-(invitation_id)}}
\item {\sphinxstyleindexentry{PUT /invitations/(invitation\_id)/cancel}}\sphinxstyleindexpageref{resources/invitation:\detokenize{put--invitations-(invitation_id)-cancel}}
\indexspace
\bigletter{/me}
\item {\sphinxstyleindexentry{GET /me}}\sphinxstyleindexpageref{resources/user:\detokenize{get--me}}
\item {\sphinxstyleindexentry{GET /me/organizations}}\sphinxstyleindexpageref{resources/user:\detokenize{get--me-organizations}}
\item {\sphinxstyleindexentry{POST /me/organizations}}\sphinxstyleindexpageref{resources/user:\detokenize{post--me-organizations}}
\item {\sphinxstyleindexentry{PUT /me}}\sphinxstyleindexpageref{resources/user:\detokenize{put--me}}
\item {\sphinxstyleindexentry{DELETE /me}}\sphinxstyleindexpageref{resources/user:\detokenize{delete--me}}
\indexspace
\bigletter{/organizations}
\item {\sphinxstyleindexentry{GET /organizations/(organization\_id)}}\sphinxstyleindexpageref{resources/organization:\detokenize{get--organizations-(organization_id)}}
\item {\sphinxstyleindexentry{GET /organizations/(organization\_id)/anchor\_circle}}\sphinxstyleindexpageref{resources/organization:\detokenize{get--organizations-(organization_id)-anchor_circle}}
\item {\sphinxstyleindexentry{GET /organizations/(organization\_id)/invitations}}\sphinxstyleindexpageref{resources/organization:\detokenize{get--organizations-(organization_id)-invitations}}
\item {\sphinxstyleindexentry{GET /organizations/(organization\_id)/members}}\sphinxstyleindexpageref{resources/organization:\detokenize{get--organizations-(organization_id)-members}}
\item {\sphinxstyleindexentry{POST /organizations/(organization\_id)/invitations}}\sphinxstyleindexpageref{resources/organization:\detokenize{post--organizations-(organization_id)-invitations}}
\item {\sphinxstyleindexentry{PUT /organizations/(organization\_id)}}\sphinxstyleindexpageref{resources/organization:\detokenize{put--organizations-(organization_id)}}
\item {\sphinxstyleindexentry{DELETE /organizations/(organization\_id)}}\sphinxstyleindexpageref{resources/organization:\detokenize{delete--organizations-(organization_id)}}
\indexspace
\bigletter{/partners}
\item {\sphinxstyleindexentry{GET /partners/(partner\_id)}}\sphinxstyleindexpageref{resources/partner:\detokenize{get--partners-(partner_id)}}
\item {\sphinxstyleindexentry{GET /partners/(partner\_id)/memberships}}\sphinxstyleindexpageref{resources/partner:\detokenize{get--partners-(partner_id)-memberships}}
\item {\sphinxstyleindexentry{PUT /partners/(partner\_id)}}\sphinxstyleindexpageref{resources/partner:\detokenize{put--partners-(partner_id)}}
\item {\sphinxstyleindexentry{DELETE /partners/(partner\_id)}}\sphinxstyleindexpageref{resources/partner:\detokenize{delete--partners-(partner_id)}}
\indexspace
\bigletter{/policies}
\item {\sphinxstyleindexentry{GET /policies/(policy\_id)}}\sphinxstyleindexpageref{resources/policy:\detokenize{get--policies-(policy_id)}}
\item {\sphinxstyleindexentry{PUT /policies/(policy\_id)}}\sphinxstyleindexpageref{resources/policy:\detokenize{put--policies-(policy_id)}}
\item {\sphinxstyleindexentry{DELETE /policies/(policy\_id)}}\sphinxstyleindexpageref{resources/policy:\detokenize{delete--policies-(policy_id)}}
\indexspace
\bigletter{/roles}
\item {\sphinxstyleindexentry{GET /roles/(role\_id)}}\sphinxstyleindexpageref{resources/role:\detokenize{get--roles-(role_id)}}
\item {\sphinxstyleindexentry{GET /roles/(role\_id)/accountabilities}}\sphinxstyleindexpageref{resources/role:\detokenize{get--roles-(role_id)-accountabilities}}
\item {\sphinxstyleindexentry{GET /roles/(role\_id)/domains}}\sphinxstyleindexpageref{resources/role:\detokenize{get--roles-(role_id)-domains}}
\item {\sphinxstyleindexentry{GET /roles/(role\_id)/members}}\sphinxstyleindexpageref{resources/role:\detokenize{get--roles-(role_id)-members}}
\item {\sphinxstyleindexentry{POST /roles/(role\_id)/accountabilities}}\sphinxstyleindexpageref{resources/role:\detokenize{post--roles-(role_id)-accountabilities}}
\item {\sphinxstyleindexentry{POST /roles/(role\_id)/domains}}\sphinxstyleindexpageref{resources/role:\detokenize{post--roles-(role_id)-domains}}
\item {\sphinxstyleindexentry{PUT /roles/(role\_id)}}\sphinxstyleindexpageref{resources/role:\detokenize{put--roles-(role_id)}}
\item {\sphinxstyleindexentry{PUT /roles/(role\_id)/circle}}\sphinxstyleindexpageref{resources/role:\detokenize{put--roles-(role_id)-circle}}
\item {\sphinxstyleindexentry{PUT /roles/(role\_id)/members/(partner\_id)}}\sphinxstyleindexpageref{resources/role:\detokenize{put--roles-(role_id)-members-(partner_id)}}
\item {\sphinxstyleindexentry{DELETE /roles/(role\_id)}}\sphinxstyleindexpageref{resources/role:\detokenize{delete--roles-(role_id)}}
\item {\sphinxstyleindexentry{DELETE /roles/(role\_id)/circle}}\sphinxstyleindexpageref{resources/role:\detokenize{delete--roles-(role_id)-circle}}
\item {\sphinxstyleindexentry{DELETE /roles/(role\_id)/members/(partner\_id)}}\sphinxstyleindexpageref{resources/role:\detokenize{delete--roles-(role_id)-members-(partner_id)}}
\end{sphinxtheindex}

\renewcommand{\indexname}{Index}
\printindex
\end{document}